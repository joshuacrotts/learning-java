\section{Exceptions}

Exceptions, at their code, are effect handlers. We use exceptions to identify and respond to events that occur at runtime. Java uses objects to implement an exception type hierarchy, with \ttt{Throwable} being the highest class in the chain. Any subclass or instance of \ttt{Throwable} can be thrown by Java. We will discuss several different exception types by categorizing them into one of two categories: unchecked versus checked checked exceptions.

\subsection{Unchecked Exceptions}
In general, we handle exceptions at either compile time or runtime. The exceptions, themselves, are thrown at runtime, but certain exceptions must be explicitly handled and referenced by the program. An \emph{unchecked exception} is a form of exception whose behavior is dictated by the runtime system, or is caught by the programmer manually. A convenience factor of unchecked exceptions is that we do not \emph{have} to explicitly state what happens when one is thrown. It should also note that the \ttt{RuntimeException} class serves as the superclass of all unchecked exceptions.

\example{Consider what happens when a program contains code that may or may not invoke a division-by-zero. If the divide-by-zero operation occurs, Java automatically throws an \ttt{ArithmeticException} with a relevant explanation of the problem. The exception halts the program at the point thereof, but what's interesting is that we can control the behavior of an unchecked exception via the \ttt{try/catch} combination. Inside a \ttt{try} block we place the code that might throw the exception. Inside the corresponding \ttt{catch} block, we initialize an exception variable to whatever the desired caught exception is, e.g., \ttt{ArithmeticException}, and we handle it inside of the block. Let's write a method that does nothing more than divides the sum of two numbers by the third.}

\begin{cl}[]{}
\begin{lstlisting}[language=MyJava]
import java.lang.ArithmeticException;

class ArithmeticExceptionExample {
  
  public double div(int a, int b, int c) {
    return (a + b) / c;
  }

  public double div2(int a, int b, int c) {
    try {
      return (a + b) / c;
    } catch (ArithmeticException ex) {
      System.err.println("div2: / by zero!");
      return 0;
    }
  }
} 
\end{lstlisting}
\end{cl}

We define two versions of \ttt{div}, where the first does not perform an explicit check for the exception and the second does. In the latter we print a message to the standard error stream and return zero. The preferable resolution is certainly up to the programmer, but in general it makes more sense to throw the exception and halt program execution, in this instance, rather than propagating a zero up to the caller. Another solution might be to return an \ttt{Optional} from the method, but \ttt{Optional} is more about compositionality rather than exceptions.

\example{In the above example, we catch the \ttt{ArithmeticException} handled by Java. Though, suppose we have a situation in which \emph{we} want to throw the exception. Because the problem arises from a bad parameter, it would be wise to throw an \ttt{IllegalArgumentException}, which designates exactly what its name suggests. We manually check to see if the divisor, namely $c$, is zero and, if so, we throw a new \ttt{IllegalArgumentException}. Because \ttt{IllegalArgumentException} is an unchecked exception, the caller needs not to handle nor necessarily know that it may invoke the exception. Should we want to signal that as a hint, in the method signature we can specify that the method potentially throws an \ttt{IllegalArgumentException}. As the callee that defines the location of an exception invocation, we \emph{only} throw the exception; it is not our responsibility to control the outcome. We can test a new version of \ttt{div} by testing whether or not it throws an exception through the \ttt{assertThrows} and \ttt{assertDoesNotThrow} assertion methods. The thing is, though, neither \ttt{assertThrows} nor \ttt{assertDoesNotThrow} are not as simple as they appear on the surface; we need to specify \emph{which} exception the code might throw as a reference to the class definition.\footnote{To reference a class definition as an object, we access \ttt{.class} on the class as if it were a static method.} Additionally, it must be passed as something that is executable. Fortunately, we have worked with executable constructs before: lambda/anonymous functions! Simply wrap the code that might raise an exception inside a lambda, and it works as expected.}

\begin{cl}[]{Throwing Exceptions Tester}
\begin{lstlisting}[language=MyJava]
import static Assertions.assertAll;
import static Assertions.assertEquals;
import static IllegalArgumentExceptionExample.div;

class IllegalArgumentExceptionExampleTester {
  
  @Test
  void testIaeException() {
    assertAll(
      () -> assertThrows(IllegalArgumentException.class, () -> div(5, 3, 0)),
      () -> assertDoesNotThrow(div(5, 3, 1)));
  }
}
\end{lstlisting}
\end{cl}

\begin{cl}[]{Throwing Exceptions} 
\begin{lstlisting}[language=MyJava]
class IllegalArgumentExceptionExample {
  
  public void div(int a, int b, int c) throws IllegalArgumentException {
    if (c == 0) { throw new IllegalArgumentException("div: / by zero"); }
    else { return (a + b) / c; }
  }
}
\end{lstlisting}
\end{cl}

What if we wanted to call \ttt{div} from a separate method and process the exception ourselves? Indeed, this is doable. Should we wish to retrieve the exception message (i.e., the message passed to the exception constructor), we can via the \ttt{.getMessage} method, which is helpful for producing custom error messages/responses, or redirecting the message to a different destination.

\begin{cl}[]{Catching Thrown Exceptions}
\begin{lstlisting}[language=MyJava]
class IllegalArgumentExceptionExample {
  
  public void div(int a, int b, int c) throws IllegalArgumentException {
    if (c == 0) { throw new IllegalArgumentException("div: / by zero"); }
    else { return (a + b) / c; }
  }

  public static void main(String[] args) {
    try {
      double res = div(2, 3, 0);
    } catch (IllegalArgumentException ex) {
      System.err.printf("main: %s\n" ex.getMessage());
    }
  }
}
\end{lstlisting}
\end{cl}

\example{Arrays and strings both produce unchecked exceptions, resulting from indexing errors, via \ttt{ArrayIndexOutOfBoundsException} and \ttt{StringIndexOutOfBounds\-Exception} respectively, both of which inherit from the \ttt{IndexOutOfBoundsException} class. We imagine that these have both been received, with great frustration, from the readers a indeterminate number of times. Of course, an index out of bounds exception stems from accessing data beyond the permissible bounds of some collection or structure.}

\begin{cl}[]{Out-of-bounds Access Examples}
\begin{lstlisting}[language=MyJava]
import static Assertions.assertAll;
import static Assertions.assertEquals;

class IndexOutOfBoundsExceptionExampleTester {

  @Test
  void testOobException() {
    String ex1 = "String";
    int[] ex2 = new int[]{5, 3, 1, 2, 4, 6}; 
    assertAll(
      () -> assertThrows(StringIndexOutOfBoundsException.class, () -> ex1.charAt(17)),
      () -> assertThrows(StringIndexOutOfBoundsException.class, () -> ex1.charAt(-1)),
      () -> assertDoesNotThrow(() -> ex1.charAt(0)),
      () -> assertDoesNotThrow(() -> ex1.charAt(ex1.length() - 1)),
      () -> assertThrows(ArrayIndexOutOfBoundsException.class, () -> ex2[17]);
      () -> assertThrows(ArrayIndexOutOfBoundsException.class, () -> ex2[-1]);
      () -> assertDoesNotThrow(() -> ex2[0]),
      () -> assertDoesNotThrow(() -> ex2[ex2.length - 1]));
  } 
}
\end{lstlisting}
\end{cl} 

Another uncomfortably common unchecked exception that many Java programmers encounter is the \ttt{NullPointerException}, most often discovered when referencing an object that has yet to be instantiated. 

\example{Casting an object to a type that it is not results in an unchecked \ttt{ClassCastException}. By ``to a type that it is not,'' we mean to say that the object is either not an instance of the type or is not a subtype of the type. Primitive datatypes are not subject to this exception, as they are not objects.\footnote{No pun intended.} All primitive datatypes, except for booleans, can be cast into one another. E.g., \ttt{int x = (int) `A';} is valid, as is \ttt{char c = (char) 65;}. On the other hand, \ttt{String x = (String) new Integer(5);} is not valid, as \ttt{Integer} is not a subclass of \ttt{String}. We can treat \ttt{List<T>} as an \ttt{AbstractList<T>} by performing a runtime cast, e.g., \ttt{AbstractList<T> x = (AbstractList<T>) ls;}, where \ttt{ls} is defined as being of type \ttt{List<T>}.}

\example{Sometimes, a program can encounter a state in which it cannot continue. In this case, we can throw an \ttt{IllegalStateException}, designating that the program has reached a point that it should not normally. An example might be attempting to access a closed \ttt{Scanner} instance.}
\begin{verbnobox}[\small]
Scanner in = new Scanner(System.in);
in.close();
String s = in.nextLine(); // Throws IllegalStateException!
\end{verbnobox}
\subsection{Checked Exceptions}

A checked exception is one that the programmer must explicitly handle. The compiler will not allow the program to compile if the code that may throw the exception is not wrapped in either a \ttt{try/catch} block, or the method signature does not specify that it throws the exception. Many checked exceptions arise from I/O operations, such as reading from or to a data source. Considering this, we will discuss checked exceptions in the context of I/O operations in the following (non-sub)section.

\subsection{User-Defined Exceptions}
The programmer can define their own exceptions in terms of other exceptions. Exceptions are nothing more than class definitions, which may be extended and inherited. 

\example{Consider defining the \ttt{BadStringInputException} class, which inherits from \ttt{RuntimeException}. We might define this as an exception that is thrown when, after reading the user's input, we find that the input is not a ``alpha string,'' i.e., a string that contains only letters. We can define a constructor that takes a string as an argument, which serves as the message that is passed to the exception.\footnote{We note that, in general, creating new exceptions is rarely beneficial, since Java provides a plethora of exception definitions that cover most use cases.}}

\begin{cl}[]{Bad String Input Exception}
\begin{lstlisting}[language=MyJava]
class BadStringInputException extends RuntimeException {
  
  public BadStringInputException(String msg) {
    super(String.format("BadStringInputException: %s", msg));
  }
}
\end{lstlisting}
\end{cl}

Then, if we write code that reads a string from the user, we can throw a \ttt{BadStringInput\-Exception} if the input is a non-alphabetic string. The following code segment uses the \ttt{matches} method, which receives a regular expression, and returns whether or not the string satisfies the expression. In particular, \ttt{[a-zA-Z]+} states that there must be at least one lowercase or uppercase character.

\begin{cl}[]{Bad String Input Running Example}
\begin{lstlisting}[language=MyJava]
import java.util.Scanner;

class BadStringInputExceptionExample {
  
  public static void main(String[] args) {
    Scanner in = new Scanner(System.in);
    String s = in.nextLine();
    if (!s.matches("[a-zA-Z]+")) { throw new BadStringInputException(s); }
  }
}
\end{lstlisting}
\end{cl}

