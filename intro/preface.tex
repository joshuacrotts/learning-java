\chapter*{Preface}
\addcontentsline{toc}{chapter}{Preface}

A course in Java programming is multifaceted. That is, it covers several core concepts, including basic datatypes, strings, simple-to-intermediate data structures, object-oriented programming, and beyond. What is commonly omitted from these courses is the notion of proper unit testing methodologies. Real-world companies often employ testing frameworks to bullet-proof their codebases, and students who leave university without expose to such frameworks are behind their colleagues. Our textbook aims to rectify this delinquency by emphasizing testing from day one; we write tests before designing the implementation of our methods, a fundamental feature of test-driven development.

In our book we design methods rather than write them, an idea stemming from Felleisen's \textit{How to Design Programs}, wherein we should determine the definition of our data, the method signature (i.e., parameter and return types), appropriate examples and test cases, and only then follow with the method implementation. Immediately diving into a method implementation often results in endless hours of debugging that could have been saved by a few minutes of preparation. Extending this idea into subsequent computer science courses is no doubt excellent.

At Indiana University, students take either a course in Python or the Beginner/Intermediate Student Languages, the latter of which involves constant testing and remediation. Previous offerings of the successor course taught in Java lead students astray into a ``plug and chug'' mindset of re-running a program until it works. Our goal is to stop this once and for all (perhaps not truly ``once and for all,'' but rather we aim to make it a less frequent habit) by teaching Java correctly and efficiently. 

Object-oriented programming (and, more noteworthy, a second-semester computer science course) is tough for many students to grasp, but even more so if they lack the necessary prerequisite knowledge. Syntax is nothing short of a different way of spelling the same concept; we reinforce topics that students should already have exposure to: methods, variables, conditionals, recursion, loops, and simple data structures. We then follow this with the Java Collections API, generics, class design, advanced object-oriented programming, searching and sorting algorithms, algorithm analysis, and modern Java features such as pattern matching and concurrency. 

The ordering of topics presented in a Java course is hotly debated and has been ever since its creation and use in higher education. Such questions include the location of object-oriented programming principles: do we start off with objects or hold off until later? Depending on the style of a text, either option can work. Even though we, personally, are more of a fan of the ``early objects'' approach, and is how we learned Java many moons ago, we choose to place objects later in the curriculum. We do this to place greater emphasis on testing, method design, recursion, and data structures through the Collections API. Accordingly, after our midterm (roughly halfway through the semester), students should have a strong foundation of basic Java syntax sans objects and class design. The second half of the class is dedicated to just that: object-oriented programming and clearing up confusions that coincide and introduce themselves. 

We believe that this textbook can be used as any standard second-semester computer science course. Instructors are free to omit certain topics that may have been covered in a prerequisite (traditionally-styled) Java course. In those circumstances, it may be beneficial to dive further into the chapters on algorithm analysis and modern Java pragmatics. For students without a Java background (or instructors of said students), which we assume, we take the time to quickly yet effectively build confidence in Java's quirky syntax. Additionally, we understand that our approach to teaching loops (through recursion and a translation pipeline) may appear odd to some long-time programmers. So, an instructor may reorder these sections in whatever order they choose, but we strongly recommend retaining our chosen ordering for pedagogical purposes, particularly for those readers that are not taking a college class using this text.

Once again, by writing this book, we wish to ensure that students are better prepared for the more complex courses in a common computer science curriculum, e.g., data structures, operating systems, algorithms, programming languages, and whatever else lies ahead. A strong foundation keeps students motivated and pushes them to continue even when times are arduous, which we understand to be plenty thereof.

\begin{flushright}
Have a blast!\\\textit{Joshua Crotts}
\end{flushright}