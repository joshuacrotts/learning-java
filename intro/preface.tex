\chapter{Preface}

A course in Java programming is multifaceted. That is, it covers several core concepts, including basic datatypes, strings, simple-to-intermediate data structures, object-oriented programming, and beyond. What is commonly omitted from these courses is the notion of proper unit testing methodologies. Real-world companies often employ testing frameworks to bullet-proof their codebases, and students who leave university without expose to such frameworks are behind their colleagues. Our textbook aims to rectify this delinquency by emphasizing testing from day one; we write tests before designing the implementation of our methods, a fundamental feature of test-driven development.

In our book we design methods rather than write them, an idea stemming from Felleisen's \textit{How to Design Programs}, wherein we should determine the definition of our data, the method signature (i.e., parameter and return types), appropriate examples and test cases, and only then follow with the method implementation. Immediately diving into a method implementation often results in endless hours of debugging that could have been saved by a few minutes of preparation. Extending this idea into subsequent computer science courses is no doubt excellent.

At Indiana University, students take either a course in Python or the Beginner/Intermediate Student Languages, the latter of which involves constant testing and remediation. Previous offerings of the successor course taught in Java lead students astray into a ``plug and chug'' mindset of re-running a program until it works. Our goal is to stop this once and for all by teaching Java correctly and efficiently. 

Object-oriented programming (and, more noteworthy, a second-semester computer science course) is tough for many students to grasp, but even more so if they lack the necessary prerequisite knowledge. Syntax is nothing short of a different way of spelling the same concept; we reinforce topics that students should already have exposure to: methods, variables, conditionals, recursion, loops, and simple data structures. We then follow this with the Java Collections API, generics, class design, advanced object-oriented programming, searching and sorting algorithms, algorithm analysis, and modern Java features such as pattern matching and concurrency. 

Once again, by writing this book, we wish to ensure that students are better prepared for the more complex courses in a common computer science curriculum, e.g., data structures, operating systems, algorithms, programming languages, and whatever else lies ahead. A strong foundation keeps students motivated and pushes them to continue even when times are arduous, which we understand to be plenty thereof.