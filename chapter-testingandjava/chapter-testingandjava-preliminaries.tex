\section{A First Glimpse at Java}
It makes little sense to avoid the topic at hand, so let us jump right in and write a program! We have seen \textit{functions}\index{function} before, as well as some mathematics operations, perhaps in a different (language) context. 

\example{Our program will convert a given temperature in Fahrenheit to Celsius.}

\begin{cl}[TempConverter.java]{}
\begin{lstlisting}[language=MyJava]
class TempConverter {
  
  /**
   * Converts a temperature from Fahrenheit to Celsius.
   * @param temperature in F.
   * @return temperature in C.
   */
  static double fahrenheitToCelsius(double d) {
    return 0.0;
  }
}
\end{lstlisting}
\end{cl}

All code, in Java, belongs to a \textit{class}\index{class}. Classes have much more complex and concrete definitions that we will investigate in due time, but for now, we may think of them as the homes of our functions. By the way, functions in Java are called \textit{methods}\index{method}.\footnote{The reasoning is simple: a method belongs to a class. Other programming languages, e.g., C++ or Python, do not restrict the programmer to writing code only within a class. Thus, there is a distinction between functions, which do not reside within a class, and methods.} Again, this slight terminology differentiation is not without its reasons, but for all intents and purposes, functions are methods and vice versa. The class we have defined in the previous listing is called \ttt{TempConverter}, giving rise to believe that the class does something related to temperature conversion.

We write the \ttt{fahrenheitToCelsius} method, whose \textit{return type}\index{return type} is a \ttt{double}, and has one \textit{parameter}\index{parameter}, which is also a \ttt{double}. A \ttt{double} is a floating-point value, meaning it potentially has decimals. For our method, this choice makes sense, because if we were to instead receive an \ttt{int}, we would not be able to convert temperatures such as \ttt{35.5} degrees Fahrenheit to Celsius. 

The \ttt{static} keyword that we wrote has significance, but for now, consider it a series of six mandatory key presses (plus one for the space thereafter).

Above this method \textit{signature}\index{signature} is a Java documentation comment, providing a brief summary of the method's purpose, as well as the data it receives as parameters and its return value, should it be necessary. 

Inside its method body lies a single \ttt{return}, in which we return \ttt{0.0}. Returning a value is what a \textit{method call}\index{method call}, or \textit{method invocation}\index{method invocation}, resolves to. For example, if we were to call \ttt{fahrenheitToCelsius} with any arbitrary \ttt{double} value, the call would be substituted with \ttt{0.0}. This is otherwise called \textit{method application}\index{method application}.

\begin{verbnobox}[\small]
fahrenheitToCelsius(5)     -> 0.0
fahrenheitToCelsius(78)    -> 0.0
fahrenheitToCelsius(-3123) -> 0.0
\end{verbnobox}

Of course, this method is meaningless without an implementation. We want to design \textit{test cases}\index{test cases} to ensure the method works as expected. Test cases verify the correctness (or incorrectness) of a method. We, as the readers, know how to convert a temperature from Fahrenheit to Celsius, but telling a computer to do such a conversion is not as obvious at first glance. To test our methods, we will use the \textit{JUnit} testing framework\index{JUnit}. To create a test for \ttt{fahrenheitToCelsius}, we will make a second class called \ttt{TempConverterTester} to house a single method: \ttt{fahrenheitToCelsiusTest}.

\begin{cl}[TempConverterTester.java]{}
\begin{lstlisting}[language=MyJava]
class TempConverterTester {

  @Test
  void testFahrenheitToCelsius() {
    Assertions.assertAll(
      () -> Assertions.assertEquals(0, TempConverter.fahrenheitToCelsius(32)));
  }
}
\end{lstlisting}
\end{cl}

We want JUnit to recognize that \ttt{fahrenheitToCelsius} contains testing code, so we prepend the \ttt{@Test} annotation to the method signature. In its body, we call \ttt{Assertions\-.assertAll}, which receives a series of methods that are ran in succession. In our case, we want to assert that our \ttt{fahrenheitToCelsius} method should return \ttt{0} degrees Celsius when given a temperature of \ttt{32} degrees Fahrenheit. The first parameter to \ttt{assertEquals} is the expected value of the test, i.e., what we want it to produce. The second parameter is the actual value of the test, i.e., what our code produces. 

When writing tests, it is important to consider \textit{edge cases}\index{edge case} and all possible branches of a method implementation. Edge cases are inputs that are possibly missed by an implementation, e.g., \ttt{-40}, since it is the same in both Fahrenheit and Celsius, or \ttt{0}. So, let us add a few more test cases.\footnote{To condense our code, we use a \ttt{static} import of our \ttt{fahrenheitToCelsius} method, which allows us to call the method without specifying the full class name. We apply this same idea to those methods used from the \ttt{Assertions} class.}

\begin{cl}[TempConverterTester.java]{}
\begin{lstlisting}[language=MyJava]
import static Assertions.assertAll;
import static Assertions.assertEquals;
import static TempConverter.fahrenheitToCelsius;

class TempConverterTester {

  @Test
  void testFahrenheitToCelsius() {
    assertAll(
      () -> assertEquals(0, fahrenheitToCelsius(32)),
      () -> assertEquals(100, fahrenheitToCelsius(212)),
      () -> assertEquals(-40, fahrenheitToCelsius(40)),
      () -> assertEquals(-17.778, fahrenheitToCelsius(0), .01),
      () -> assertEquals(-273.15, fahrenheitToCelsius(-459), .01));
  }
}
\end{lstlisting}
\end{cl}

Floating-point arithmetic can cause precision/rounding errors. So, as an optional third argument to \ttt{assertEquals}, we might provide a \textit{delta}\index{delta argument}, which allows for precision up to a certain amount to be accepted as a valid answer. For example, our tolerance for the fourth and fifth test cases is \ttt{0.01}, meaning that if our actual value is less than $\pm 0.01$ away from the expected value, the test case succeeds.

Now that we have copious amounts of tests, we can write our method definition. Of course, it is trivial to write.

\begin{cl}[TempConverter.java]{}
\begin{lstlisting}[language=MyJava]
class TempConverter {

  /**
   * Converts a temperature from Fahrenheit to Celsius.
   * @param temperature in F.
   * @return temperature in C.
   */
  static double fahrenheitToCelsius(double d) {
    return (d - 32) * (5.0 / 9.0)
  }
}
\end{lstlisting}
\end{cl}

This definition brings up a few points about Java's type system. The primitive mathematics operations account for the types of its arguments. So, for instance, subtracting two integers will produce another integer. More noteworthy, perhaps, a division of two integers produces another integer, even if that result seems to be incorrect. Thus, \ttt{5 / 9} results in the integer \ttt{0}. If, however, we treat at least one of the operands as a floating-point value, we receive a correct result of approximately \ttt{0.555555}: \ttt{5.0 / 9}. Java by default uses the standard order of operations when evaluating mathematics expressions, so we force certain operations to occur first via parentheses.\footnote{By ``standard,'' we mean the widely-accepted paradigm of parentheses first, then exponents, followed by left-to-right multiplication/division, and finally left-to-right addition/subtraction.}

Unlike some programming languages that are \textit{dynamically-typed}, e.g., Scheme, Python, JavaScript, the Java programming language requires the programmer to specify the types of variables.\footnote{In Java 10, the \ttt{var} keyword was introduced, which automatically infers the type of a given expression.} Java has several default \textit{primitive datatypes}\index{primitive datatypes}, which are the simplest reducible form of a variable. Such types include \ttt{int}, \ttt{char}, \ttt{double}, \ttt{boolean}, and others. Integers, or \ttt{int}, are any positive or negative number without decimals. Doubles, or \ttt{double}, are values with decimals. Characters, or \ttt{char}, are a single character enclosed by single quotes, e.g., \ttt{\q{X}\q}. Finally, booleans, or \ttt{boolean}, are either \ttt{true} or \ttt{false}. There are other Java data types that specify varying levels of precision for given values. Integers are 32-bit signed values, meaning they have a range of $[-2^{31}, 2^{31})$. The \ttt{short} data type, on the contrary, is 16-bit signed. Beyond this is the \ttt{byte} data type that, as its name suggests, stores 8-bit signed integers. Floating-point values are more tricky, but while \ttt{double} uses 64 bits of precision, the \ttt{float} data type uses 32 bits of precision.

\example{Let us write a method that receives two three-dimensional vectors and returns the distance between the two. We can, effectively, think of this as the distance between two points in a three-dimensional plane. Therefore because each vector contains three components, we need six parameters, where each triplet represents the vectors $v_1$ and $v_2$.}

\begin{cl}[VectorDistance.java]{}
\begin{lstlisting}[language=MyJava]
class VectorDistance {

  /**
   * Computes the distance between two given vectors:
   * v_1=(x_1, y_1, z_1) and v_2=(x_2, y_2, z_2)
   */
  static double computeDistance(double x1, double y1, double z1, 
                                double x2, double y2, double z2) {
    return 0.0;
  }
}
\end{lstlisting}
\end{cl}

Again we start by writing the appropriate method signature with its respective parameters and a Java documentation comment explaining its purpose. For tests, we know that the distance between two Cartesian points in a three-dimensional plane is 

\[
D(v_1, v_2) = \sqrt{(x_1 - x_2)^2 + (y_1 - y_2)^2 + (z_1 - z_2)^2}
\]

So, let's now write a few test cases with a few arbitrarily-chosen points that we can verify with a calculator or manual computation.

\begin{cl}[VectorDistanceTester.java]{}
\begin{lstlisting}[language=MyJava]
import static Assertions.assertAll;
import static Assertions.assertEquals;
import static VectorDistance.computeDistance;

class VectorDistanceTester {

  @Test
  void testComputeDistance() {
    assertAll(
      assertEquals(8.66, computeDistance(3, 2, 1, 8, 7, 6), .01),
      assertEquals(12.20, computeDistance(0, 0, 0, 8, 7, 6), .01),
      assertEquals(8.30, computeDistance(-8, -2, 1, 0, 0, 0), .01));
  }
}
\end{lstlisting}
\end{cl}

Notice again our use of the optional delta parameter to allow us a bit of leeway with the rounding of our answer. Fortunately, the implementation of our method is just a retelling of the mathematical definition.

\begin{cl}[VectorDistance.java]{}
\begin{lstlisting}[language=MyJava]
class VectorDistance {

  /**
   * Computes the distance between two given vectors:
   * v_1=(x_1, y_1, z_1) and v_2=(x_2, y_2, z_2)
   */
  static double computeDistance(double x1, double y1, double z1, 
                                double x2, double y2, double z2) {
    return Math.sqrt(Math.pow(x1 - x2, 2) 
                   + Math.pow(y1 - y2, 2)
                   + Math.pow(z1 - z2, 2));
  }
}
\end{lstlisting}
\end{cl}

We make prolific use of Java's \ttt{Math} library in designing this method; we use the \ttt{sqrt} method for computing the square root of our result, as well as \ttt{pow} to square each intermediate difference.

\example{Slope-intercept is an incredibly common algebra and geometry problem, and even pokes its way into machine learning at times when computing best-fit lines. Let's write two methods, both of which receive two points $(x_1, y_1)$, $(x_2, y_2)$. The first method returns the slope $m$ of the points, and the second returns the y-intercept $b$ of the line. Their respective signatures are straightforward---each set of points is represented by two integer values and return doubles.}

\begin{align*}
m &= \dfrac{y_2-y_1}{x_2-x_1}\\
b &= y_1 - mx_1
\end{align*}

\begin{cl}[SlopeIntercept.java]{Computing the Slope Intercept Method Signatures}
\begin{lstlisting}[language=MyJava]
class SlopeIntercept {

  /**
   * Computes the slope of the line represented by the two Cartesian points.
   * @return slope of points.
   */
  static double slope(int x1, int y1, int x2, int y2) {
    return 0.0;
  }

  /**
   * Computes the y-intercept of the line represented by the two Cartesian points.
   * @return y-intercept of line represented by points.
   */
  static double yIntercept(x1, int y1, int x2, int y2) {
    return 0.0;
  }
}
\end{lstlisting}
\end{cl}

The tests that we write are verifiable by a calculator or mental math. Note that the \ttt{yIntercept} method depends on a successful implementation of \ttt{slope}, as designated by the formula of the former. In the next chapter, we will consider cases that invalidate the formula, e.g., when two points share an $x$ coordinate, in which the slope is undefined for those points.

\begin{cl}[SlopeInterceptTester.java]{Testing the Slope Intercept Methods}
\begin{lstlisting}[language=MyJava]
import static Assertions.assertAll;
import static Assertions.assertEquals;
import static SlopeIntercept.slope;
import static SlopeIntercept.yIntercept;

class SlopeInterceptTester {
  
  @Test
  void testSlope() {
    assertAll(
      () -> assertEquals(1, slope(0, 0, 1, 1)),
      () -> assertEquals(0, slope(0, 0, 1, 0)),
      () -> assertEquals(2, slope(8, 4, 2, 4)),
      () -> assertEquals(0.5, slope(-1, 5, 3, 7)));
  }

  @Test
  void testYIntercept() {
    assertAll(
      () -> assertEquals(0, yIntercept(0, 0, 1, 1)),
      () -> assertEquals(0, yIntercept(0, 0, 1, 0)),
      () -> assertEquals(4, yIntercept(8, 4, 2, 4)),
      () -> assertEquals(5.5, yIntercept(-1, 5, 3, 7)));
  }
}
\end{lstlisting}
\end{cl}

And the implementation of the two methods follows from the mathematical definitions. We replace our temporary \ttt{0.0} return values with the appropriate expressions, and all tests pass.

\begin{cl}[SlopeIntercept.java]{Implementing the Slope Intercept Methods}
\begin{lstlisting}[language=MyJava]
class SlopeIntercept {

  /**
   * Computes the slope of the line represented by the two Cartesian points.
   * @return slope of points.
   */
  static double slope(int x1, int y1, int x2, int y2) {
    return (y2 - y1) / (x2 - x1);
  }

  /**
   * Computes the y-intercept of the line represented by the two Cartesian points.
   * @return y-intercept of points.
   */
  static double yIntercept(x1, int y1, int x2, int y2) {
    return y1 - slope(x1, y1, x2, y2) * x1;
  }
}
\end{lstlisting}
\end{cl}

\example{We are starting to get used to some of Java's verbosity! Let us now write a method that, when given a value of $x$, evaluates the following quartic formula:}

\[
q(x) = 4x^4 + 7x^3 + 21x^2 - 65x + 3
\]

Its signature is straightforward: we receive a value of $x$, namely a \ttt{double}, and return a \ttt{double} since we are performing mathematical operations on \ttt{double} values. Again, we return zero as a temporary solution to ensure the program successfully compiles.
 
\begin{cl}[QuarticFormulaSolver.java]{}
\begin{lstlisting}[language=MyJava]
class QuarticFormulaSolver {

  /**
   * Evaluates the following quartic equation:
   * 4x^4 + 7x^3 + 21x^2 - 65x + 3.
   * @param the input variable x.
   * @return the result of the expression after substituting x.
   */
  static double solveQuartic(double x) {
    return 0.0;
  }
}
\end{lstlisting}
\end{cl}

Test cases are certainly warranted, but may be a bit tedious to compute by hand, so we recommend using a verified calculator to compute expected solutions!\footnote{Remember that the coefficient is applied \textit{after} applying the exponent to the variable. That is, if $x=3$, then $4x^3$ is equal to $4 \cdot (3)^3$, which resolves to $4 \cdot 27$, which resolves to $108$.}

\begin{cl}[QuarticFormulaSolverTester.java]{}
\begin{lstlisting}[language=MyJava]
import static Assertions.assertAll;
import static Assertions.assertEquals;
import static QuarticFormulaSolver.solveQuartic;

class QuarticFormulaSolverTester {

  @Test
  void testSolveQuartic() {
    assertAll(
      () -> assertEquals(3, solveQuartic(0)),
      () -> assertEquals(510, solveQuartic(3)),
      () -> assertEquals(229878, solveQuartic(15)),
      () -> assertEquals(313445.1875, solveQuartic(16.25)));
  }
}
\end{lstlisting}
\end{cl}

Again, we write tests \textit{before} the method implementation because we know, intuitively, how to solve an equation for a variable, but a computer has to be told how to solve this task. Fortunately for us, a quartic equation solver is nothing more than returning the result of the expression. We, of course, have to use the exponential \ttt{Math.pow} method again (or conjoin several multiplicatives of $x$), but otherwise, it is straightforward.

\begin{cl}[QuarticFormulaSolver.java]{}
\begin{lstlisting}[language=MyJava]
class QuarticFormulaSolver {

  /**
   * Evaluates the following quartic equation:
   * 4x^4 + 7x^3 + 21x^2 - 65x + 3.
   * @param the input variable x.
   * @return the result of the expression after substituting x.
   */
  static double solveQuartic(double x) {
    return 4 * Math.pow(x, 4) + 7 * Math.pow(x, 3) + 21 * Math.pow(x, 2) - 65 * x + 3;
  }
}
\end{lstlisting}
\end{cl}

And, as expected, all tests pass. With only methods and math operations at our disposal, the capabilities of said methods is quite limited. Let us start revamping our tool set by reintroducing strings.

\subsection*{The Main Method}
Java programming tutorials are quick to throw a lot of information at the reader/viewer, and our textbook is no exception to this practice. Unfortunately, Java is one of the more verbose programming languages, and to get up and running with a method, we must wrap its definition inside a class. From there we can, of course, begin to write our \ttt{static} method. The question, of course, is what does the static keyword mean? For the first few chapters, we will intentionally omit the definition, as it would almost certainly confuse the reader coming from another language. Therefore, for the time being, simply view it as six characters, plus a space, that must be typed in order to write a method to then test with JUnit. Though, imagine a case in which we want to output the result of some value without using a test, as we will do in many examples. Java requires a \ttt{main} method in any executable Java class that does not use tests. For the sake of completeness, let's now write the traditional ``Hello, world!'' program, but in Java using the \ttt{main} method and not tests.

\begin{cl}[]{Example of the Main Method}
\begin{lstlisting}[language=MyJava]
class MainMethod {

  public static void main(String[] args) {
    System.out.println("Hello, world!");
  }
}
\end{lstlisting}
\end{cl}

Yikes, that is a lot of required code to output some text to the console; what does \ttt{public} mean, and what are those \ttt{[]} brackets after the \ttt{String} word? Once again, we will not detail of their importance but view them as more mandatory characters to type when writing a \ttt{main} method. The only word we will explain is \ttt{void}, which means that the method does not return a value. If the readers are coming from a functional programming language, e.g., Scheme/Racket or OCaml, then it is almost certainly the case that they never worked with methods that did not return a value nor received no arguments. The \ttt{println} method in particular has no return value; its significance is the fact that it outputs some text to the terminal/console, which is a side-effect of the method. We'll come back to what all of this means in subsequent chapters, but we could not avoid at least briefly discussing it and its existence in the Java language.