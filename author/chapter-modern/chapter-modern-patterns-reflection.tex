\section{Reflection}

Reflection\index{reflection}, while not a necessarily new computer science concept, is a powerful way for programming languages to interpret and potentially modify its own structure. We made an example of reflection in our chapter on classes and objects to pass around a class type as a parameter. 

\myexample{Suppose that we want to be able to search, then invoke, a method based on its name. Reconsider the primitive calculator example from many chapters ago, where we utilize a case dispatch on the operator received as a terminal argument. As we add functions to the system, we must proportionally add code in the \ttt{main} method to account for the new case, which is tiresome at best. Java allows us to lookup a method, at runtime, using its reflection API, and pass parameters accordingly. As a substitute for terminal arguments (and the \ttt{main} method in general), we will pass an ``argument array'' to a static \ttt{calculate} method, so we can easily run unit tests.}

\begin{lstlisting}[language=MyJava]
import static Assertions.assertAll;
import static Assertions.assertEquals;

class ReflectiveCalculatorTester {

  private static final double DELTA = 0.0001;

  @Test
  void testCalculator() {
    assertAll(
      () -> assertEquals(5, 
                         calculate(new String[] {"add", "2", "3"}), 
                         DELTA),
      () -> assertEquals(5, 
                         calculate(new String[] {"subtract", "10", "5"}), 
                         DELTA),
      () -> assertEquals(10, 
                         calculate(new String[] {"multiply", "2", "5"}), 
                         DELTA),
      () -> assertEquals(2, 
                         calculate(new String[] {"divide", "10", "5"}), 
                         DELTA));
  }
}
\end{lstlisting}

To retrieve a method at runtime, we first need to retrieve the class in which it lives through the \ttt{Class.forName} method. In this case we pass \ttt{ReflectiveCalculator}, as that is the name of our class. Note that the return value is of type \ttt{Class<?>}, denoting that it is a reflective class type.

With the class type object in hand, we must now get the method object to call. To do so we need two values: its identifier, and its parameter types. We will assume that the arguments to the calculator functions are strings, which can be converted inside the respective methods. This approach makes it significantly easier to inform the reflection API of our parameter type(s). Therefore we use the \ttt{getDeclaredMethod} method, which takes the name of the method as a string, and a variadic number of \ttt{Class<?>} objects that represent the parameter types. In our case, the parameter type is a \ttt{String[]}, so we pass \ttt{String[].class}. Because we might attempt to reference a non-existent method, we must wrap this in a \ttt{try-catch} block.

Finally, we invoke the method encapsulated by the \ttt{Method} using the conveniently named \ttt{invoke} method, which receives the object on which to invoke the method (in this case, \ttt{null} because the method is static), and the necessary arguments. The return value is an \ttt{Object}, so we must cast it to the appropriate type. There are several issues that may arise when calling a method, such as the method being inaccessible due to its access modifier, the method throwing a checked exception, or passing the wrong number of arguments. We must handle these issues accordingly via \ttt{try-catch} blocks.\footnote{Our code uses the Java 7 feature of the vertical pipe `\ttt{|}' to catch/handle multiple exceptions at once, removing duplicate code.} The four calculator methods are trivial and have been omitted; all they do is convert the string arguments to numbers and perform, then return, the corresponding operation.

There is one small intricacy with how Java handles variadic arguments and passing arrays to reflective methods. Passing an array to a variadic argument method unwraps the arguments and passes them individually. For example, if we have a method \ttt{foo(String... args)}, and we pass \ttt{foo(new String[] \string{"Hello", "World"\string})}, the method receives two arguments: \ttt{"Hello"} and \ttt{"World"}. In our case this is problematic, since we want our computation methods to receive the entire array of arguments. To accomplish this, we can cast the array to an \ttt{Object} and pass it to \ttt{invoke} as a single argument.

\begin{lstlisting}[language=MyJava]
import java.lang.reflect.Method;
import java.lang.reflect.InvocationTargetException;
import java.lang.IllegalAccessException;
import java.lang.NoSuchMethodException;

class ReflectiveCalculator {

  /**
   * Calculates the result of a given operation on two numbers.
   * @param args - the operation to perform, and the two numbers.
   * @return the result of the operation.
   */
  static double calculate(String[] args) {
    Class<?> cls = ReflectiveCalculator.class;
    try {
      Method mtd = cls.getMethod(args[0], String[].class);
      return (double) mtd.invoke(null, args);
    } catch (NoSuchMethodException
           | InvocationTargetException
           | IllegalAccessException ex) {
      throw new RuntimeException(ex);
    }
  }
}
\end{lstlisting}

\myexample{The Python programming language, including many others, offer a REPL, or a read-evaluate-print-loop, that users can run at the terminal to evaluate expressions and programs. This is done to alleviate the need to open a source file in a text editor, type the code, then run the file with the respective command. Java is not one of these languages out-of-the-box, unfortunately. There is an application, namely JShell, which introduces this functionality. In this example, we will write our own version of JShell, where the programmer can type and evaluate expressions and statements at the terminal.\footnote{Credit goes to Terence Parr for this example and ``assignment'' from his graduate course on programming languages at the University of San Francisco. That's right--this is a graduate-level programming example!}}

First, we need to establish a few details. Our program will read, from standard input, a subset of Java code. After reading, we spin up an instance of the Java compiler and compile the source code into a \ttt{.class} file. Finally, using reflection, we search for and execute that code. This is the goal at a high level, but we certainly need to break this down into sub-components. Let's see some examples of what we will be able to do in due time:

\begin{verbnobox}[\small]
> int i = 5;
> System.out.println(i)
5
> class Animal { String speak() { return ""; }}
> class Dog extends Animal { String speak() { return "Wolf!"; }}
> Animal a = new Dog();
> System.out.println(a.speak());
"Wolf!"
> System.out.println(List.of(i, i + 2));
[5, 7]
> System.out.println(IntStream.iterate(0, i -> i + 5).limit(10));
[0, 5, 10, 15, 20, 25, 30, 35, 40, 45]
\end{verbnobox}

As we can see, powerful programs are a possibility only by using the REPL. The question now is, how do we get to that point?

The idea with this project is to output our lines of source code into a temporary Java class. In essence, we create a temporary directory on the system that that stores a separate class (file), enumerated from zero, containing whatever we enter at the REPL. Each time we enter code, we wrap it inside a class called \ttt{"InterpX"} where \ttt{X} is the current class identifier, so to speak. These classes are extended with each new segment of code received to allow access to prior declarations/definitions. As an example, consider the following class definitions generated after inputting the first two lines that we showed earlier:

\begin{verbnobox}[\small]
class Interp0 {
  static int i = 5;
  static void execute() {}
}

class Interp1 extends Interp0 {
  static void execute() {
    System.out.println(i);  
  }
}
\end{verbnobox}

The \ttt{execute} method is generated all the time, but we only care about its contents (i.e., we only populate its body) when we write something that is not a declaration. For example, if we write a class definition or a variable, it is nonsensical to place it inside the \ttt{execute} method body, since in the former case it would not compile, and in the latter case it would be local to only that version of \ttt{execute}. So, we need a way of determining whether or not a piece of code \emph{is} a declaration/definition. All declarations in our subset of Java will be restricted to the following grammar:

\setlength{\grammarparsep}{20pt plus 1pt minus 1pt} % increase separation between rules
\setlength{\grammarindent}{12em} % increase separation between LHS/RHS 
\begin{figure}[H]
  \begin{grammar}
      <expr> ::= <varDecl> 
              \alt <methodDecl> 
              \alt <classDecl>

      <varDecl> ::= <type> ` ' <id> `;'
                \alt <type> `[]' ` ' <id> `;'

      <methodDecl> ::= <type> ` ' <id> `(' <params> `)' `{' <body> `}'

      <classDecl> ::= `class ' <id> `{' <body> `}'
                  \alt `class ' <id> ` extends ' <id> `{' <body> `}'

  \end{grammar}
  \caption{ Extended BNF Grammar for Declarations }
\end{figure}

Let's assume that the \ttt{type} rule is any arbitrary sequence of characters representing a type in Java (e.g., \ttt{void}, \ttt{int}, \ttt{Dog}), as is an identifier. Under these assumptions, it is simple to parse a string to determine if it's a declaration. We will further assume that the user is acting in good faith and typing only valid Java code that compiles under said assumptions; responses to inputs such as \ttt{intttt x = 5;} will not be considered in this example.\footnote{We make prolific use of regular expressions in this next section. You are free to gloss over these details or use raw string parsing with substring and equals, but regex makes the parsing process significantly faster.}

\begin{lstlisting}[language=MyJava]
class JavaRepl {

  private boolean isDeclaration(String s) {
    return this.isVariableDeclaration(s)
        || this.isMethodDeclaration(s)
        || this.isClassDeclaration(s);
  }

  /**
   * Determines if a given string is a type of Java variable declaration.
   * Types of var declarations:
   * - int x;
   * - int x = 5;
   * - int[] a = new int[5];
   * - String x = "Hi!";
   * - ArrayList<Integer> foo;
   */
  private boolean isVariableDeclaration(String s) {
    return s.matches("[a-zA-Z_\\[\\]<>]+\\s*[a-zA-Z0-9_]+(\\s*=\\s*.+?)?;");
  }

  /**
   * Determines if a given string is a type of Java method declaration.
   * Types of method declarations:
   * - int foo() {}
   * - int bar(int x, int y) {}
   * - String baz(ArrayList<Integer> foo) {}
   */
  private boolean isMethodDeclaration(String s) {
    return s.matches("[a-zA-Z_\\[\\]<>]+\\s+[a-zA-Z0-9_]+\\(.*?\\)\\s*\\{.*?\\}");
  }

  /**
   * Determines if a given string is a type of Java class declaration.
   * Types of class declarations:
   * - class Animal {}
   * - class Dog extends Animal {}
   */
  private boolean isClassDeclaration(String s) {
    return s.matches("class\\s+[a-zA-Z_<>]+\\s+\\{.*?\\}") 
        || s.matches("class\\s+[a-zA-Z_<>]+\\s+extends\\s+[a-zA-Z_<>]+\\s
                     *\\{.*?\\}");
  }
}
\end{lstlisting}

Anything that is not a declaration is something else that belongs inside the body of \ttt{execute}, and is executed accordingly. For instance, if we declare $i$ to be zero, then use a post-increment operator on $i$, the \ttt{i++} statement is stored inside \ttt{Interp1}'s \ttt{execute} body, and is invoked prior to the next standard input reading. 

Now, onto the main event. We know how to read lines/data from standard input, so that's nothing more than a rehashing of what we've seen before. We create an infinite loop, read in a single string (line), and create a \ttt{File} object out of that string by storing it in a temporary class file. Where do we create that temporary class file? In a temporary directory, which is created by a static method in the\ttt{Files} class. Let's set this up in the constructor of \ttt{JavaRepl}, since the temporary directory (path) won't change for the lifetime of the program. Let's also instantiate the standard input reading mechanism, i.e., a \ttt{BufferedReader} that operates over an \ttt{InputStreamReader}.

\begin{lstlisting}[language=MyJava]
import java.io.*;
import java.nio.file.Files;
import java.nio.file.Path;

class JavaRepl {

  private final Path TMP_DIR;
  private final BufferedReader IN; 
  private int classNo;

  JavaRepl() {
    try {
      this.TMP_DIR = Files.createTempDirectory();
    } catch (IOException ex) {
      throw new RuntimeException(ex);
    } 
    
    this.IN = new BufferedReader(new InputStreamReader(System.in));
    this.classNo = 0;
  }
}
\end{lstlisting}

With this, we know that we want to store the class and what its name should be, thanks to a counter that starts at zero. In the \ttt{createJavaFile} method, we create a \ttt{File} at the correct location, with its contents populated by a writer of some kind. Here is where we make use of the \ttt{isDeclaration} method---if the source code that we pass is, in fact, a declaration, we precede it with the \ttt{static} keyword. Otherwise, it resides inside the \ttt{static void execute()} method. Creating and returning this file is straightforward. We need to account for the fact that if the class number is zero, we do not extend any class. As a corollary point, to be able to use certain classes, e.g., \ttt{List}, at the REPL, we include two wildcard imports from the I/O and utilities Java packages.

\begin{lstlisting}[language=MyJava]
class JavaRepl {
  // ... other methods not shown.
  
  private File createJavaFile(String src) {
    File f = new File(String.format("%s/Interp%d.java",
                                    this.TMP_DIR, 
                                    this.classNo));
    try (PrintWriter pw = new PrintWriter(new FileWriter(f))) {
      pw.println("import java.io.*;");
      pw.println("import java.util.*;");
      // If it is the starting class we do not have it extend anything.
      pw.println(String.format("class Interp%d %s", this.classNo,
                               this.classNo == 0
                               ? "{" 
                               : "extends Interp" + (this.classNo - 1) + " {"));

      // If it's a declaration, it cannot be inside a method.
      if (this.isDeclaration(src)) {
        pw.printf("static %s\n", src);
        pw.println("static void execute() {}");
      } else {
        pw.printf("static void execute() { %s }\n", src);
      }
      pw.println("}");
    } catch (IOException ex) {
      ex.printStackTrace();
    }
    return f;
  }
}
\end{lstlisting}

Here we run into our first of two roadblocks: how to compile the file at runtime. Fortunately, there exists the Java compiler API, which contains methods and classes to compile a file definition. Most of this code is boilerplate ``setup,'' so understanding it completely is unnecessary. At the core, we instantiate a compiler object, as well as a list of diagnostics that may result from compiling the program (e.g., error messages). These get passed into a compilation unit that executes the compiler. Upon success, it produces a \ttt{.class} file in the temporary directory, as we would if we were compiling our Java code by hand. We will instantiate the \ttt{JavaCompiler}, \ttt{StandardJavaFileManager}, and \ttt{DiagnosticCollector} objects in the constructor. More importantly we have the \ttt{executeRepl} method, which starts the REPL and listens for data on standard input. This is where we use the API to compile and produce the \ttt{.class} file.   

\begin{lstlisting}[language=MyJava]
import javax.tools.*;
import java.lang.StringBuilder;

class JavaRepl {
  // ... other instance variables not shown.

  private final JavaCompiler JAVAC;
  private final StandardJavaFileManager FILE_MANAGER;
  private final DiagnosticCollector<? super JavaFileObject> DS;

  JavaRepl() {
    // ... other instantiations not shown.
    this.JAVAC = ToolProvider.getSystemJavaCompiler();
    this.DS = new DiagnosticCollector<>();
    this.FILE_MANAGER = this.JAVAC.getStandardFileManager(this.DS, null, null);
  }

  /**
   * Continuously loops, reading in lines of input representing valid Java
   * programs. These are converted into statements/expressions that are fed into
   * a skeleton Java class file. We use Java's runtime compiler to execute
   * these, and the reflection API to dynamically load the class at runtime.
   */
  private void executeRepl() throws Exception {
    List<File> programs = new ArrayList<>();

    while (true) {
      // For now assume that one line is the entire program.
      StringBuilder sb = new StringBuilder();
      System.out.print("> ");
      sb.append(this.IN.readLine());
      programs.add(this.createJavaFile(sb.toString()));

      // Create the compiler from these files.
      var units = this.FILE_MANAGER.getJavaFileObjectsFromFiles(programs);
      this.task = (JavacTask) this.JAVAC.getTask(null, this.FILE_MANAGER, this.DS,
                                                 null, null, units);

      // Compile the list of programs.
      boolean ok = this.task.call();
      if (!ok) {
        for (var diag : this.DS.getDiagnostics()) {
          System.err.println(diag);
        }
      } else {
        // TODO.
      }
    }
  }
}
\end{lstlisting}

If the compilation was unsuccessful, thereby meaning \ttt{ok} is false, we iterate over the diagnostics received from the compile and print them to the standard error stream. Otherwise, we know it successfully compiled and we now have a class file.

Now comes the second roadblock: we want to load this file into memory via the reflection API and preserve changes to static fields. For example, if we increment/reassign a variable, its state should be updated across the REPL. To do this, we need to use a common class loader\index{common class loader}, which persists changes to static fields and means we can load a new class at runtime. Indeed, we want to load the new class that we just compiled, namely \ttt{Interp$X$}, and invoke its \ttt{execute} method. This is nothing new, but what we have not seen is the common class loader approach; we store an instance of \ttt{URLClassLoader} that gets instantiated to refer to the temporary directory path, in our constructor. Therefore, when loading a class via \ttt{loadClass}, any changes we made to previous variables remain loaded into memory.

\begin{lstlisting}[language=MyJava]
class JavaRepl {

  // ... other instance variables not shown.

  private final URLClassLoader CLASS_LOADER;

  JavaRepl() {
    // ... other instantiations not shown.
    try {
      this.CLASS_LOADER = URLClassLoader.newInstance(
                            new URL[]{this.TMP_DIR.toUri().toURL()});
    } catch (MalformedURLException e) {
      throw new RuntimeException(e);
    }
  }

  /**
   * Continuously loops, reading in lines of input representing valid Java
   * programs. These are converted into statements/expressions that are
   * fed into a skeleton Java class file. We use Java's runtime compiler to
   * execute these, and the reflection API to dynamically load the class at runtime.
   */
  private void executeRepl() throws Exception {
    List<File> programs = new ArrayList<>();

    while (true) {
      // ... compilation omitted.
      if (!ok) {
        // ... omitted.
      } else {
        Class<?> cls = CLASS_LOADER.loadClass("Interpret" + this.classNo);
        Method method = cls.getMethod("execute");
        method.invoke(null);
        this.classNo++;
      }
    }
  }
}
\end{lstlisting}

In summary, we load the current class, reflectively grab its \ttt{execute} method, then execute it with \ttt{null} as an argument, designating that it is a static method in the class. Finally we increment the class number for the next line.