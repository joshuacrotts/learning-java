\section{Verbosity}

\subsection{Main Method and Anonymous Classes}
All the way back in~\Cref{chapter-testingandjava}, we introduced the \ttt{main} method, and stated that all methods must belong inside a class. 
Prior to Java version 21, this was indeed true. 
Java 21, however, has a much cleaner and succinct syntax for writing the \ttt{main} method that does not need \ttt{public}, nor \ttt{static}, nor the input array of strings. 
These changes make Java much more beginner-friendly. 
Let's see what a ``Hello, world!'' program looks like with the improvements.

\begin{lstlisting}[language=MyJava]
class MainMethod {
  
  void main() {
    System.out.println("Hello, world!");
  }
}
\end{lstlisting}

Additionally, these changes mean that any methods that we want to reference/invoke within \ttt{main} (that are defined inside the respective class) do not need to be marked as static. 
For instance, let's once again write the \ttt{double fToC(double f)} method, only this time, we will call it from inside the \ttt{main} method.

\begin{lstlisting}[language=MyJava]
class MainMethod {

  double fToC(double f) {
    return (f - 32) * (5.0 / 9.0);
  }
  
  void main() {
    System.out.printf("%d deg F = %d deg C.\n", 32, fToC(32));
  }
}
\end{lstlisting}

Whether we label \ttt{fToC} as static is irrelevant in this circumstance. 
Unfortunately, if we want to create unit tests for this method, it needs to be \ttt{static}, as we cannot reference it outside the class definition without the static qualifier. 
Though, what if we could write unit tests within the \ttt{MainMethod} class, rather than writing an entirely separate file? 
Making such an alteration couples the logic of the method with the tests, which, in a large project is not an advisable choice. 

To write the unit test, all we need is the \ttt{void testFToC()} method and prepend the \ttt{@Test} annotation.\footnote{The respective import statements are also necessary.} 
Then, our IDE will automatically detect that we have a test method in the file and is executable.\footnote{We will note that we could have done this back in Chapter~\ref{chapter-testingandjava}, but we favor separating the tests and the method definition, even if this means we must use the \ttt{static} keyword.} 
Since we wish to emphasize writing tests \emph{before} the (method) implementation, we place the testing method above the method that it tests. 

\begin{lstlisting}[language=MyJava]
import static Assertions.assertAll;
import static Assertions.assertEquals;

class MainMethod {

  @Test
  void testFToC() {
    assertAll(
      () -> assertEquals(0, fToC(32));
      () -> assertEquals(-40, fToC(-40));
      () -> assertEquals(100, fToC(212)));
  }
  
  /**
   * Converts a temperature in Fahrenheit to Celsius.
   * @param f - degrees Fahrenheit.
   * @return degrees Celsius.
   */
  double fToC(double f) {
    return (f - 32) * (5.0 / 9.0); 
  }
}
\end{lstlisting}

In addition to a less-verbose \ttt{main} method, Java 21 also introduces \emph{nameless/anonymous classes}, which reduces the required keystrokes even further by removing the need for a class definition. 
So, if all we want to do is write the \ttt{main} method (and perhaps other methods callable from \ttt{main}), we might write the following:

\begin{lstlisting}[language=MyJava]
/**
 * Converts a temperature in Fahrenheit to Celsius.
 * @param f - degrees Fahrenheit.
 * @return degrees Celsius.
 */
double fToC(double f) {
  return (f - 32) * (5.0 / 9.0); 
}

void main() {
  System.out.printf("%d deg F = %d deg C\n", 32, fToC(32));
}
\end{lstlisting}

Note that nameless classes (or the methods contained within) cannot be referenced from outside the definition of the class, nor can we write and execute JUnit tests. 
So, the utility of nameless classes is little, in our opinion.

% \subsection{String Templates and Interpolation}

% Consider the following code, which instantiates an array of $512$ elements to contain random floating-point values between $[0, 512)$:

% \begin{verbnobox}[\small]
% final int COUNT = 512;
% double[] arr = new double[COUNT];
% for (int i = 0; i < arr.length; i++) {
%   arr[i] = Math.random() * arr.length;
% }
% double total = Arrays.stream(arr).sum();
% double avg = total / arr.length;
% \end{verbnobox}

% Suppose that we want to output a message containing the total and average of the data in the \ttt{arr} array. There are a few ways that we could achieve this goal: (1) string concatenation, (2) string formatting.

% String concatenation is, of course, straightforward but means we cannot easily format the resulting output string. Plus, string concatenation is slow, requiring a new string for each concatenation operation.

% \begin{verbnobox}[\small]
% String res = "The sum of " + COUNT + " rand nums is " + total + ". "
%              + "The avg is " + avg + ".";
% \end{verbnobox}

% On the other hand, let's see what we get through \ttt{String.format}, which composes one string with the provided data. The problem with this approach is that it requires knowing the format specifiers for arbitrary data (see the table from Chapter~\ref{chapter-testingandjava}).

% \begin{verbnobox}[\small]
% String res = String.format("The sum of %d rand nums is %lf. The avg is %lf.",
%                            COUNT, total, avg);
% \end{verbnobox}

% Other programming languages, e.g., Python, support \emph{string interpolation}, which is a technique to embed content directly in a string without the need for format specifiers and repetitive concatenation. We embed the variables in a special ``format string,'' which resolve at runtime to their values.

% Prior to JDK 22, Java provided no mechanism for string interpolation. Nowadays, however, Java has several avenues for supporting string interpolation via \emph{string templates}. To declare a string that uses string interpolation, we must use the \ttt{STR} constant. We follow this by a string literal that may or may not contain expressions that resolve to values. These expressions much be enclosed via \ttt{\textbackslash$\{\}$}. Let's see such an example.

% \begin{verbnobox}[\small]
% String res = 
%   STR."The sum of \{COUNT} rand nums is \{total}. The avg is \{avg}.";
% \end{verbnobox}

% The resulting string stored in \ttt{res}, at runtime, contains the values in the variables \ttt{COUNT}, \ttt{total}, and \ttt{512}, where the latter two are indeterminate. Though, what if we want to round the floating-point variables to, say, two decimals? The \ttt{FMT} constant is here to save the day. It uses the same format specifiers as those for traditional string formatting, e.g., \ttt{\%d}, but can be integrated with string templates. Let's see an example; we will format the \ttt{total} and \ttt{average} variables to use two and four decimal places respectively. Moreover, this example causes the resulting string to be longer than our textbox allows. In other such circumstances where we might break the string literal manually and conjoin them with a concatenation operator, Java provides a solution to this as well. Three side-by-side double-quotes will parse all following lines as a string without the need to manually insert concatenation operations. Be aware, though, that doing so adds a newline character at the point of the break, which may or may not be desired.

% \begin{verbnobox}[\small]
% String res = FMT."""
%   The sum of \{COUNT\} rand nums is %.2f\{total}.
%   The avg is %.4f\{avg}.
%   """
% \end{verbnobox}

