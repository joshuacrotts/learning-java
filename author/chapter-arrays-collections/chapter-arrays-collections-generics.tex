\section{Type Parameters}

\emph{Generics}\index{generics} as a concept go far back in programming history, generally reknown as type parameterization\index{type parameterization} or \emph{parameteric polymorphism}\emph{parameteric polymorphism}, which we briefly touched on during our discussion of how to instantiate instances of \ttt{ArrayList} from the Collections API. 
Imagine, for a moment, if Java programmers had to, by hand, design different implementation of \ttt{ArrayList} for \ttt{Integer}, \ttt{String}, \ttt{Double}, and so on for every object type. 
Not only is this impossible (because there are a countably-infinite number of types), it is also extremely cumbersome and redundant, since the only altering parameter is the underlying element type in the data structure. 
Before Java 5, we could only use ``generics'' via collections of type \ttt{Object}, since it is the root class object type, meaning any element could be stored in any type of collection.

\begin{lstlisting}[language=MyJava]
import java.util.ArrayList;

class WeakGenerics {

  public static void main(String[] args) {
    ArrayList al1 = new ArrayList();
    al1.add(new Integer(42));
    al1.add(new Integer(43));
    Integer x = (Integer) al1.get(0);
  }
}
\end{lstlisting}

Performing runtime casts in the Java 5 sense is prone to errors, not to mention the possibility of adding disjoint types into a collection. 
For example, nothing prevents us from adding objects of type \ttt{String} or \ttt{Integer} into an \ttt{ArrayList}. 
Generics\index{generics} were introduced to convert the problem from one encountered at runtime to one encountered moreso at compile-time. 

Since we have yet to discuss objects in detail, we will hold off on a significant discussion of generic class implementations. 
To keep everything to the point, we can write any class to be generic and store objects of an arbitrary type. 
Fortunately, we can also design generic static methods. 

To declare a static method as generic, we must specify the type variable(s) necessary to use the method. 
These \emph{quantifier} come after the \ttt{static} keyword but before the return type. 
For instance, if we want to say that an object of type \ttt{T} is used in the static method \ttt{foo}, we declare it as \ttt{static <T> void foo(...)\string{...\string}}. 
Then, if we want to say that the method returns or receives an object of type \ttt{T}, we substitute the return/parameter type with \ttt{T}, e.g., \ttt{static <T> T foo(T arg)\string{...\string}}. 
At compile-time, Java looks for method invocations of \ttt{foo} and substitutes the \ttt{T} for the type \ttt{foo} is invoked with. 

\myexample{Let's design the \ttt{search} method, which receives a list~$t$ and an object~$k$, where the elements of~$t$ are of type \ttt{T} and~$k$ is also of type \ttt{T}.} 
Its purpose is to return the index of the first occurrence of~$k$ in~$t$, and~$-1$ if it does not exist. 
Because all objects have a \ttt{equals} method for object equality, we can take advantage of that when comparing objects in the list against the search parameter. 
When testing, we will instantiate the type parameter to several different object types to demonstrate.

\begin{lstlisting}[language=MyJava]
import static Assertions.assertAll;
import static Assertions.assertEquals;

import java.util.List;
import java.util.ArrayList;

class GenericSearchTester {

  @Test 
  void testGenericSearch() {
    List<Integer> l1 
      = new ArrayList<>(List.of(1, 2, 3, 4, 5));
    List<String> l2 
      = new ArrayList<>(List.of("a", "b", "c", "d", "e"));
    List<Double> l3 
      = new ArrayList<>(List.of(1.0, 2.0, 3.0, 4.0, 5.0));
    List<Character> l4 
      = new ArrayList<>(List.of('a', 'b', 'c', 'd', 'e'));
    List<List<Integer>> l5 
      = new ArrayList<>();

    assertAll (
      () -> assertEquals(1, genericSearch(l1, 2)),
      () -> assertEquals(-1, genericSearch(l1, 6)),
      () -> assertEquals(1, genericSearch(l2, "b")),
      () -> assertEquals(-1, genericSearch(l2, "f")),
      () -> assertEquals(1, genericSearch(l3, 2.0)),
      () -> assertEquals(-1, genericSearch(l3, 6.0)),
      () -> assertEquals(1, genericSearch(l4, 'b')),
      () -> assertEquals(-1, genericSearch(l4, 'f')),
      () -> assertEquals(-1, genericSearch(l5, List.of(1, 2, 3))));
  }
}
\end{lstlisting}

%\enlargethispage{-6\baselineskip}
\begin{lstlisting}[language=MyJava]
import java.util.List;

class GenericSearch {

  /**
   * Returns the index of the first occurrence of k in t, 
   * or -1 if it does not exist.
   * @param t - the list of type T.
   * @param k - the object of type T to search for.
   * @return the index of k or -1.
   */
  static <T> int genericSearch(List<T> t, T k) {
    for (int i = 0; i < t.size(); i++) {
      if (t.get(i).equals(k)) { return i; }
    }
    return -1;
  }
}
\end{lstlisting}

\subsection{Bounded Type Parameters}
To restrict the type parameter to a certain subset of types, we can use \emph{bounded type parameters}\index{bounded type parameters}\index{type parameters}. 
As an example, we might wish to restrict a type parameter for a method to only types that implement the \ttt{Comparable} interface.\footnote{For now, ``implementing'' an ``interface'' means that a type ``behaves like'' another type. So, a type that implements \ttt{Comparable} means that it behaves like a ``comparable'' object.} 
Doing so means that the type parameter has access to any methods defined by the interface, in this case, \ttt{compareTo} being the only available method. 
To specify a bounded type parameter, we use the \ttt{extends} keyword, e.g., \ttt{<T extends Comparable>}, which serves as an upper-bound on the type parameter, since we are restricting the type parameter to a subset of types that are ``above'' the specified type. 
We might also wish to use a lower-bound, which restricts the type parameter to a subset of types that are ``below'' the specified type. 
For example, if we want to restrict the type parameter to only types that are superclasses of \ttt{Integer}, we can use \ttt{<T super Integer>}.

%\enlargethispage{1\baselineskip}
\myexample{Let's design the \ttt{max} method, which receives a list~$t$ whose elements are of type \ttt{T}, and returns the maximum element in the list.}
Because determining the max element of a list involves comparison-based checking, we restrict the type parameter to only types that implement \ttt{Comparable}. 
In the previous section, we discussed that \ttt{Optional} is a container class that either holds a value or is empty.
An exercise at the end of this section asks you to use \ttt{Optional} in designing a similar method, rather than returning \ttt{null}, as we do here.

\begin{lstlisting}[language=MyJava]
import static Assertions.assertAll;
import static Assertions.assertEquals;

import java.util.List;
import java.util.ArrayList;

class GenericMaxTester {
  
  @Test 
  void testGenericMax() {
    List<Integer> l1 = new ArrayList<>(List.of(5, 10, 20, 7, 2));
    List<String> l2 = new ArrayList<>(List.of("A", "e", "x", "3", "N"));
    List<Double> l3 = new ArrayList<>(List.of(500.0, 3.0, Math.PI));
    List<Character> l4 = new ArrayList<>(List.of('?','@','A','a','Z'));
    List<List<Integer>> l5 = new ArrayList<>();
    assertAll (
      () -> assertEquals(20, genericMax(l1)),
      () -> assertEquals("x", genericMax(l2)),
      () -> assertEquals(500.0, genericMax(l3)),
      () -> assertEquals('a', genericMax(l4)),
      () -> assertEquals(null, genericMax(l5)));
  }
}
\end{lstlisting}

%\enlargethispage{-2\baselineskip}
\begin{lstlisting}[language=MyJava]
import java.util.Comparable;
import java.util.List;

class GenericMax {
  
  /**
   * Returns the maximum element in the list according to 
   * the compareTo implementation of the type parameter.
   * @param t - the list of type T, where T is a type 
   *            that implements Comparable.
   * @return the maximum element in the list.
   */
  static <T extends Comparable<T>> T genericMax(List<T> t) {
    if (t.isEmpty()) { return null; }
    else {
      T max = t.get(0);
      for (int i = 1; i < t.size(); i++) {
        if (t.get(i).compareTo(max) > 0) { 
          max = t.get(i); 
        }
      }
      return max;
    }
  }
}
\end{lstlisting}

\subsubsection*{Wildcards and Unspecific Bounds}
Sometimes we want to specify that a collection contains different, but related, types. 
If we declare a \ttt{List<Integer>}, then Java throws a compile-time error if we attempt to pass, say, an unboxed \ttt{double}, since a \ttt{Double} is not of type \ttt{Integer}. 
As a solution, Java allows the programmer to enforce \emph{wildcard} bounds on the type of a generic. 

\myexample{Reusing the example that we just talked about, if we want to store both \ttt{Integer} and \ttt{Double} objects in the same collection, we need to examine how they are related.} 
Both of these classes extend the \ttt{Number} superclass, so we can place an upper-bound on the type parameter to say that anything that extends \ttt{Number}, whatever it may be, can be stored in the collection. 
Wildcards are denoted via the question mark: `\ttt{?}', to represent that it can be substituted with any other type that satisfies the bound criteria. 
Our example method, which computes the sum of a list of numbers, requires the substitutable type to be a subclass of \ttt{Number}.

\begin{lstlisting}[language=MyJava]
import static Assertions.assertAll;
import static Assertions.assertEquals;

import java.util.List;

class NumberListTester {

  private static final double DELTA = 0.0000001;

  @Test
  void testSum() {
    assertAll(
      () -> assertEquals(0, sum(List.of())),
      () -> assertEquals(42, sum(List.of(-1, (short) 138.2, 
                                         2.6d, -95L, -2.8f)), DELTA));
  }
}
\end{lstlisting}

\begin{lstlisting}[language=MyJava]
import java.util.List;

class NumberList {

  /**
   * Computes the sum of a list of Number subclasses.
   * @param ls - list of Number instances.
   * @return sum as a double.
   */
  static double sum(List<? extends Number> ls) {
    return ls.stream()
             .mapToDouble(Number::doubleValue)
             .sum();
  }
}
\end{lstlisting}

\myexample{Lower-bounded type parameters are the dual to upper-bounded type parameters.} 
If we want to restrict our possible types in a generic implementation to be only superclasses of a type, we can easily do so. 
For instance, the following code specifies that the input list must only contain objects that are superclasses of \ttt{Integer}, or are \ttt{Integer} itself. 
Unfortunately, the only classes that are ancestors/superclasses of \ttt{Integer} are \ttt{Number} and \ttt{Object}, which severely limits the capabilities of our list. 
For the sake of an example, we might return a string containing the ``stringified'' elements, separated by commas, enclosed by brackets.

\begin{lstlisting}[language=MyJava]
import static Assertions.assertAll;
import static Assertions.assertEquals;

import java.util.List;

class NumberListTester {

  @Test
  void testStringify() {
    assertAll(
      () -> assertEquals("[]", stringify(List.of())),
      () -> assertEquals("[42, 32, Hi]", stringify(List.of(42, 32, (Object) "Hi"))));
  }
}
\end{lstlisting}

\begin{lstlisting}[language=MyJava]
import java.util.List;
import java.util.stream.Collectors;

class NumberList {

  /**
   * Receives a list of objects that are superclasses of Integer
   * and converts them to their string counterparts and puts them
   * in a list representation.
   * @param ls - list of instances that are superclasses of Integer.
   * @return stringified list.
   */
  static String stringify(List<? super Integer> ls) {
    return ls.stream()
             .map(Object::toString)
             .collect(Collectors.joining(", ", "[", "]"));
  }
}
\end{lstlisting}

Without our own classes to work with, the potential benefits for wildcards and type parameter bounds are not easy to see.
We present them in this chapter, though, to minimally demo their existence.