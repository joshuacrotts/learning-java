\section{Exercises}

\myexercise{1}{chapter-classes}{This exercise has 3 parts.}

The \emph{Kotlin} programming language supports customized \emph{ranges}. That is, we can define an interval using dot notation, e.g., \ttt{1..10}, then query a value over that interval. For instance, \ttt{x in 1..10} returns whether or not \ttt{x} is between \ttt{1} and \ttt{10}, inclusive. This comparison, however, extends beyond primitive datatypes; ranges may operate over classes. For example, we can create a range \ttt{"hi".."howdy"}, which defines the range of strings in between \ttt{"hi"} and \ttt{"howdy"}.

\begin{enumerate}[label=(\alph*)]
    \item Design the generic \ttt{Range} class. It should store, as instance variables, a minimum and a maximum value, both of which are of type \ttt{<T extends Comparable<T>>}, meaning \ttt{T} must be a comparable type.
    \item The \ttt{Range} constructor should receive these two values as parameters and assign them to the instance variables accordingly. 
    \item Design the \ttt{boolean contains(T v)} method that returns whether or not $v$ is between the interval that this range operates over. 
\end{enumerate}

\myexercise{2}{chapter-advanced-oop}{Design the generic static method \ttt{T validateInput(String prompt, String errResp, U extends Predicate<T> p)} that receives a prompt, an error response, and an object that implements the \ttt{Predicate} interface to test whether or not the received value, received through standard input, is valid. If the value is invalid according to the predicate, print the error response and re-prompt the user. Otherwise, return the entered value.}

\myexercise{2}{This exercise has three parts.} In this exercise, you'll be developing a \ttt{Document} interface along with its implementing classes:
\begin{itemize}
\item \ttt{TextDocument}
\item \ttt{SpreadsheetDocument}
\item \ttt{PresentationDocument}
\end{itemize}

\noindent The \ttt{Document} interface is defined as follows:
\begin{verbnobox}[\small]
interface Document {

  /**
   * Returns the number of pages in this document.
   */
  int numberOfPages();

  /**
   * Returns a string representing that the Document
   * is being printed.
   */
  default String print() {
    return "Printing the document!";
  }
}
\end{verbnobox}

Notice that we have a \ttt{default} method, which is one that an implementing class does \textbf{not} have to implement. It provides ``default'' functionality, should the ``implementee'' not want to implement the method (hence the name!).

\begin{enumerate}[label=(\alph*)]
\item Implement the other three classes with the following specifications:
\begin{itemize}
\item A \ttt{TextDocument} consists of 100 pages. When it is printed, it should return a message \ttt{"Printing text document!"}.
\item A \ttt{SpreadsheetDocument} has 50 pages. When it is printed, it should return a message \ttt{"Printing spreadsheet document!"}.
\item A \ttt{PresentationDocument} contains 20 pages. It utilizes the default implementation of the \ttt{print} method.
\end{itemize}
\item Design the \ttt{PrintingOffice} class, which includes the following static method: \ttt{static OptionalDouble averagePages(List<Document> documents)}. This method calculates and returns the average number of pages across the provided list of \ttt{Document} objects. Remember why we use \ttt{Optional}: if there are no \ttt{Document} objects in the list, we would be dividing by zero if we took the average! 
\item Inside the \ttt{PrintingOffice} class, design the \ttt{static void printDocuments(List<Document> documents)} method, responsible for invoking the \ttt{print} method on each object in the list of \ttt{Document} instances.
\end{enumerate}

\myexercise{3}{chapter-advanced-oop}{A \emph{lazy list} is one that, in theory, produces infinite results! Consider the \ttt{ILazyList} interface below:}
\begin{lstlisting}[language=MyJava]
interface ILazyList<T> {
  T next();
}
\end{lstlisting}

When calling \ttt{next} on a lazy list, we update the contents of the lazy list and return the next result. We mark this as a generic interface to allow for any desired return type. For instance, below is a lazy list that produces factorial values:\footnote{We will ignore the intricacies that come with Java's implementation of the \ttt{int} datatype. To make this truly infinite, we could use \ttt{BigInteger}.}
\begin{lstlisting}[language=MyJava]
class FactorialLazyList implements ILazyList<Integer> {

  private int n;
  private int fact;
 
  FactorialLazyList() {
    this.n = 1;
    this.fact = 1;
  }

  @Override
  public int next() {
    this.fact *= this.n;
    this.n++;
    return this.fact;
  }
}
\end{lstlisting}

Testing it with ten calls to \ttt{next} yields predictable results.

\begin{lstlisting}[language=MyJava]
import static Assertions.assertAll;
import static Assertions.assertEquals;

class FactorialLazyListTester {

  @Test
  void testFactorialLazyList() {
    ILazyList<Integer> FS = new FactorialLazyList();
    assertAll(
      () -> assertEquals(1, FS.next()),
      () -> assertEquals(2, FS.next()),
      () -> assertEquals(6, FS.next()),
      () -> assertEquals(24, FS.next()),
      () -> assertEquals(120, FS.next()),
      () -> assertEquals(720, FS.next()),
      () -> assertEquals(5040, FS.next()),
      () -> assertEquals(40320, FS.next()),
      () -> assertEquals(362880, FS.next()),
      () -> assertEquals(3628800, FS.next()));
  }
}
\end{lstlisting}

Design the \ttt{FibonacciLazyList} class, which implements \ttt{ILazyList<Integer>} and correctly overrides \ttt{next} to produce Fibonacci sequence values. You code should \emph{not} use any loops or recursion. Recall that the Fibonacci sequence is defined as $f(n) = f(n - 1) + f(n - 2)$ for all $n\geq{2}$. The base cases are $f(0) = 0$ and $f(1) = 1$.

\begin{lstlisting}[language=MyJava]
import static Assertions.assertAll;
import static Assertions.assertEquals;

class FibonacciLazyListTester {

  @Test
  void testFibonacciLazyList() {
    ILazyList<Integer> FS = new FibonacciLazyList();
    assertAll(
      () -> assertEquals(0, FS.next()),
      () -> assertEquals(1, FS.next()),
      () -> assertEquals(1, FS.next()),
      () -> assertEquals(2, FS.next()),
      () -> assertEquals(3, FS.next()),
      () -> assertEquals(5, FS.next()),
      () -> assertEquals(8, FS.next()),
      () -> assertEquals(13, FS.next()),
      () -> assertEquals(21, FS.next()),
      () -> assertEquals(34, FS.next()));
  }
}
\end{lstlisting}

\myexercise{2}{chapter-advanced-oop}{Design the \ttt{LazyListTake} class. It should receive an \ttt{ILazyList} and an integer $n$ denoting how many elements to take, as parameters. Then, write a \ttt{List<T> getList()} method, which returns a \ttt{List<T>} of $n$ elements from the given lazy list.}


\begin{lstlisting}[language=MyJava]
import static Assertions.assertAll;
import static Assertions.assertEquals;

class LazyListTakeTester {

 @Test
 void testLazyListTake() {
  LazyListTake llt1 = new LazyListTake(new FactorialLazyList(), 8);
  LazyListTake llt2 = new LazyListTake(new FibonacciLazyList(), 10);

  assertAll(
    () -> assertEquals("[1, 2, 6, 24, 120, 720, 5040, 40320]",
                       llt1.getList().toString()),
    () -> assertEquals("[0, 1, 1, 2, 3, 5, 8, 13, 21, 34]",
                       llt2.getList().toString()));
 }
}
\end{lstlisting}

\myexercise{2}{chapter-advanced-oop}{Java's functional API allows us to pass lambda expressions as arguments to other methods, as well as method references (as we saw in Chapter~\ref{chapter-arrays-collections}). Design the generic \ttt{FunctionalLazyList} class to implement \ttt{ILazyList}, whose constructor receives a unary function \ttt{Function<T, T> f} and an initial value \ttt{T t}. Then, override the \ttt{next} method to invoke $f$ on the current element of the lazy list and return the previous. For example, the following test case shows the expected results when creating a lazy list of infinite positive multiples of three.}

\begin{lstlisting}[language=MyJava]
import static Assertions.assertEquals;
import static Assertions.assertAll;

class FunctionalLazyListTester {

  @Test
  void testMultiplesOfThreeLazyList() {
    ILazyList<Integer> mtll = new FunctionalLazyList<>(x -> x + 3, 0);
    assertAll(
      () -> assertEquals(0, mtll.next()),
      () -> assertEquals(3, mtll.next()),
      () -> assertEquals(6, mtll.next()),
      () -> assertEquals(9, mtll.next()),
      () -> assertEquals(12, mtll.next()));
  }
}
\end{lstlisting}

What's awesome about this exercise is that it allows us to define the elements of the lazy list as any arbitrary lambda expression, meaning that we could redefine \ttt{FactorialLazyList} and \ttt{FibonacciLazyList} in terms of \ttt{FunctionLazyList}. We can generate infinitely many ones, squares, triples, or whatever else we desire.

\myexercise{2}{chapter-advanced-oop}{Design the generic \ttt{CyclicLazyList} class, which implements \ttt{ILazyList}, whose constructor is variadic and receives any number of values. Upon calling \ttt{next}, the cyclic lazy list should return the first item received from the constructor, then the second, and so forth until reaching the end. After returning all the values, cycle back to the front and continue. For instance, if we invoke \ttt{new CyclicLazyList<Integer>(1, 2, 3)}, invoking \ttt{.next} five times will produce \ttt{1}, \ttt{2}, \ttt{3}, \ttt{1}, \ttt{2}.}

\myexercise{1}{chapter-advanced-oop}{Design the \ttt{static <T> Predicate<T> orEquals(Predicate<T> pred, T x)} method that, when given a predicate $p$ and an object $x$, returns a \emph{new} predicate that returns true if its argument $x'$ is equal (using \ttt{equals}) to $x$ or satisfies $p(x)$.}

\myexercise{2}{chapter-advanced-oop}{Design the \ttt{static <T> List<T> predicateOrEquals(List<T> ls, Predicate<T> pred, BiFunction<T, T, Boolean>, T x)} method that, when given a list of values $ls$, a predicate $p$, a function $f$, and a value $x$ that returns the list of values in $ls$ that either satisfy $p$ or are equal according to $f$. For the purposes of this question, $f$ is a method of two arguments of type $T$ that determines whether or not they are ``equal'' according to some criteria.}

\myexercise{1}{chapter-advanced-oop}{Design the \ttt{static <T> boolean andMap(List<T> ls, Predicate<T> pred)} method that returns whether or not all elements of the input list satisfy the given predicate. Use the stream API to solve this problem, but do \emph{not} use the \ttt{allMatch} method, as that method solves the problem we want \emph{you} to solve!}

\myexercise{2}{chapter-advanced-oop}{Design the \ttt{static <T, U> U foldr(List<T> ls, BiFunction<T, U, U> f, U u)} method that receives a list of values $ls$, a function $f$, and an initial value $u$. The method should return the result of folding the list from the right with the given function and initial value. By ``folding,'' we mean that we apply $f$ to the last element of the list and the initial value, then apply $f$ to the second-to-last element and the result of the previous application, and so forth. To think of this in terms of infix notation over some list, consider the list $[a, b, c, d]$. Folding it over the function $\circ$ and initial value $u$ is $a \circ (b \circ (c \circ (d \circ u)))$. Do \emph{not} use the \ttt{reduce} method, as that method solves the problem we want \emph{you} to solve!}

\myexercise{1}{chapter-advanced-oop}{Design the \ttt{static <T, U> List<U> buildList(int n, Function<T, U> f)} method that receives an integer $n$ and a function $f$ and returns a list of $n$ elements, where the $i^\text{th}$ element is $f(i)$. For example, if we invoke \ttt{buildList(5, x -> x * x)}, we should receive the list $[1, 4, 9, 16, 25]$.}

\myexercise{1}{chapter-advanced-oop}{This exercise involves the doubly-linked list we wrote in the chapter.}
Design the \ttt{int size()} method, which returns the number of elements in the list. You can do this either recursively or with a loop. For better practice, try (and thoroughly test) both implementations.

\myexercise{2}{chapter-advanced-oop}{This exercise involves the doubly-linked list we wrote in the chapter.}
Design the \ttt{void set(int i, T v)} method, which overwrites/assigns, at index $i$, the value $v$. If the provided index is out-of-bounds, do nothing.

\myexercise{2}{chapter-advanced-oop}{This exercise involves the doubly-linked list we wrote in the chapter.}
Design the \ttt{void insert(int i, T v)} method, which inserts the value $v$ at index $i$. As an example, if we insert $4$ into the list $[20, 5, 100, 25]$ at index $2$, the list then becomes $[20, 5, 4, 100, 25]$. If the provided index is out-of-bounds, do nothing.

\myexercise{2}{chapter-advanced-oop}{This exercise involves the interpreter we wrote in the chapter.}
Our interpreter, so far, is very memory inefficient. The reason is not apparent at first glance, but consider every time that we create a \ttt{NumberNode} or, especially, a \ttt{BooleanNode}. There are only two possibilities for a \ttt{BooleanNode}: true and false, which are immutable by design. So, the interpreter should never waste time allocating an instance thereof when it can simply reference an existing one. Such an optimization follows the ``factory'' design pattern, which we will explore in Chapter~\ref{chapter-modern}.

Privatize the \ttt{BooleanNode} constructor, then design the \ttt{static BooleanNode of(boolean b)} method. It receives a boolean value $v$ and returns a reference to a pre-allocated static true or false node. Similarly, also privatize the \ttt{NumberNode} constructor and design the \ttt{static NumberNode of(double v)} method. This method returns a new \ttt{NumberNode} if the given value $v$ is not an integer between $[0, 1000)$. Otherwise, precache those integer values and store them in a private lookup table.

\myexercise{1}{chapter-advanced-oop}{This exercise involves the interpreter we wrote in the chapter.}
Add the \ttt{"read-number"} and \ttt{"print"} primitive operations to the language. The latter is polymorphic, meaning it can print both numbers and booleans.

\myexercise{2}{chapter-advanced-oop}{This exercise involves the interpreter we wrote in the chapter.}
Functional programming languages, in general, are a composition of expressions, wherein statements are more of an afterthought. To this end, design the \ttt{BeginNode} abstract syntax tree node, which receives a list of abstract syntax trees. At the interpreter level, the \ttt{BeginNode} should evaluate each of the abstract syntax trees in the list, and return the result of the last one.

\myexercise{2}{chapter-advanced-oop}{This exercise involves the interpreter we wrote in the chapter.}
Variables, in our language, are defined and bound exactly once, namely when they are defined within a let node. Though, in imperative programming, it is often crucial to allow variable reassignments. Design the \ttt{SetNode} class, which receives a variable and an abstract syntax tree, and reassigns the variable to the result of the abstract syntax tree. At the interpreter level, the \ttt{SetNode} should evaluate the abstract syntax tree, and reassign the variable to the result in the current environment (and only the current environment). This means that you'll need to modify the \ttt{Environment} class to allow for variable reassignments. Hint: create a \ttt{set} method in the \ttt{Environment} class.

\myexercise{2}{chapter-advanced-oop}{This exercise involves the interpreter we wrote in the chapter.} Recursion is nice and intuitive, for the most part. Unfortunately, it is not always the most efficient way to solve a problem. For example, the Fibonacci sequence, as we saw in Chapter~\ref{chapter-crl}, is often defined recursively, but it is much more efficient to define it iteratively (or even with tail recursion). Design the \ttt{WhileNode} class, which receives a condition and an abstract syntax tree, and evaluates the abstract syntax tree until the condition is false. At the interpreter level, the \ttt{WhileNode} should evaluate the condition, and if it is true, evaluate the abstract syntax tree, and repeat until the condition is false. To test your implementation, you will need to combine the \ttt{WhileNode} with both the \ttt{SetNode} and \ttt{BeginNode} classes.

\myexercise{3}{chapter-advanced-oop}{This exercise involves the interpreter we wrote in the chapter.} Having to manually update our case analysis on the primitive operator type is cumbersome and prone to mistakes. A better solution would be to store the operator and its corresponding ``handler'' method, i.e., the method that receives the operands and does the logic of the operator. We can do this via a map where the keys are the string operators and the values are functional references to the handlers. Unfortunately, Java does not directly support passing methods as parameters, meaning they are not first-class. Conversely, we can make use of Java's functional interfaces to achieve our goal. Namely, the interface will contain one method: \ttt{AstNode apply(List<AstNode> args, Environment env)}, where \ttt{args} is the list of evaluated arguments. We will call the interface \ttt{IFunction} and make it generic, with the first type quantified to a list of \ttt{AstNode} instances, and the second type quantified to \ttt{AstNode}. Hopefully, the connection between these quantified types and the signature of \ttt{apply} is apparent. Using the below definition of \ttt{IFunction}, update \ttt{PrimNode} to no longer perform a case analysis in favor of the map. We provide an example of populating the map with the initial operators in a \ttt{static} block.

\begin{lstlisting}[language=MyJava]
@FunctionalInterface
interface IFunction<T, R> {
  
  R apply(T t);
}
\end{lstlisting}

\begin{lstlisting}[language=MyJava]
import java.util.Map;
import java.util.HashMap;

class PrimNode extends AstNode {
  
  private static final Map<String, IFunction<List<AstNode>, AstNode>> OPS;
  
  static {
    OPS = new HashMap<>();
    OPS.put("+", this.primPlus);
  }

  @Override
  AstNode eval(Environment env) { /* TODO. */  }

  private AstNode primPlus(List<AstNode> args, Environment env) { 
    /* Details omitted. */ 
  }
}
\end{lstlisting}

\myexercise{3}{chapter-advanced-oop}{This exercise is multi-part and involves the interpreter we wrote in the chapter.}
\begin{enumerate}[label=(\alph*)]
  \item First, design the \ttt{ProgramNode} class, which allows the user to define a program as a sequence of statements rather than a single expression.
  \item Design the \ttt{DefNode} class, which allows the user to create a global definition. Because we're now working with definitions that do not extend the environment, we should use the set method in environment. When creating a global definition via \ttt{DefNode}, we're expressing the idea that, from that point forward, the (root) environment should contain a binding from the identifier to whatever value it binds.
  \item Design the \ttt{FuncNode} node. We will consider a function definition as an abstract tree node that begins with \ttt{FuncNode}. This node has two parameters to its constructor: a list of parameter (string) identifiers, and a single abstract syntax tree node representing the body of the function. We will only consider functions that return values; void functions do not exist in this language.
  \item Design the \ttt{ApplyNode} class, which applies a function to its arguments. You do not need to consider applications in which the first argument is a non-function. 

  Calling/Invoking a function is perhaps the hardest part of this exercise. Here's the idea, which is synonymous and shared with almost all programming languages: 

  \begin{enumerate}[label=(\roman*)] 
  \item First, evaluate each argument of the function call. This will result in several evaluated abstract syntax trees, which should be stored in a list. 
  \item We then want to create an environment in which the formal parameters are bound to their arguments. Overload the extend method in \ttt{Environment} to now receive a list of string identifiers and a list of (evaluated) AST arguments. Bind each formal to its corresponding AST, and return the extended environment. 
  \item Evaluate the function identifier to get its function definition as an abstract syntax tree.
  \item Call \ttt{eval} on the function body and pass the new (extended) environment.
  \end{enumerate}
  This seems like a lot of work (because it is), but it means you can write really cool programs, including those that use recursion!
  %\enlargethispage{2\baselineskip}
  \begin{verbnobox}[\small]
new ProgramNode(
  new DefNode("!", 
    new FuncNode(
      List.of("n"),
      new IfNode(
        new PrimNode("eq?", 
          new VarNode("n"), 
          new NumNode(0)),
        new NumNode(1),
        new PrimNode("*", 
          new VarNode("n"), 
          new ApplyNode("!", 
            new PrimNode("-", 
            new VarNode("n"), 
            new NumNode(1))))))),
  new ApplyNode("!", new NumNode(5)))
\end{verbnobox}
\end{enumerate}

\myexercise{2}{chapter-advanced-oop}{This exercise involves the interpreter we wrote in the chapter, and relies on the addition of \ttt{FuncNode} and \ttt{ApplyNode}.}
Our current version of the interpreter uses \textit{dynamic scoping}. A dynamically-scoped interpreter is one that uses the value of the closest declaration of a variable. This seems like nonsense without an example, so consider the following code listing.
\begin{lstlisting}[language=MyScheme, frame=single]
(define f
  (let ([x 10])
    ((*;$\lambda$;*) ()
      x)))

(let ([x 3])
  (f))
\end{lstlisting}

Under dynamic scoping, this program outputs \ttt{3}, because the binding of \ttt{x} takes on the value \ttt{3}. On the other hand, if we were using a \textit{lexically-scoped} interpreter, the program would output \ttt{10}, which seems to make more sense due to the binding that exists immediately above the function declaration. The question is: how do we implement lexical scoping into our interpreter? The answer is via \textit{closures}, which are data structures that couple a function definition with an environment. Then, when we apply a closure to an argument (if it exists), we restore the environment that was captured by the closure. 

To this end, design the \ttt{ClosureNode} class, whose constructor receives a \ttt{FuncNode} and an \ttt{Environment}. The respective \ttt{eval} method returns \ttt{this}, but \ttt{FuncNode} changes slightly. Rather than returning \ttt{this}, we return a \ttt{ClosureNode}, which wraps the current \ttt{FuncNode} and the environment passed to \ttt{eval}. Finally, inside \ttt{ApplyNode}, evaluating the function definition should resolve to a closure. Evaluate the arguments to the closure inside the passed environment, but \textit{extend} the captured environment, and bind the formals to the arguments in this extended environment. The body of the closure's function definition is then evaluated inside this new environment. 

Making this alteration not only causes our programs to output the ``common sense'' result, but also means we can implement recursive functions using \textit{only} a \ttt{LetNode}. See if you can figure out how to do this!

\myexercise{3}{chapter-advanced-oop}{This exercise involves the interpreter we wrote in the chapter.}
Data structures are a core and fundamental feature of programming languages. A language without them, or at least one to build others on top of, suffers severely in terms of usability. We will implement a \emph{cons}-like data structure for our interpreter. In functional programming, we often use three operations to act on data structures akin to linked lists: \emph{cons}, \emph{first}, and \emph{rest}, to construct a new list, retrieve the first element, and retrieve the rest of the list respectively. We can inductively define a cons list as follows:

\begin{verbnobox}[\small]
A ConsList is one of:
 - new ConsList()
 - new ConsList(x, ConsList)
\end{verbnobox}

Implement the cons data structure into your interpreter. This should involve designing the \ttt{ConsNode} class that conforms to the aforementioned data definition. Moreover, you will need to update \ttt{PrimNode} to account for the \ttt{first} and \ttt{rest} primitive operations, as well as an \ttt{empty?} predicate, which returns whether or not the cons list is empty. Finally, update the \ttt{AstNode} class to print a stringified representation of a \ttt{ConsNode}, which amounts to printing each element, separated by spaces, inside of brackets, e.g., \ttt{[$l_0, l_1, ..., l_{n-1}]$}.

\myexercise{2}{chapter-advanced-oop}{This exercise involves the interpreter we wrote in the chapter.}
Having to manually type out the abstract syntax tree constructors when writing tests is extremely tedious. Design a \emph{lexer} for the language described by the interpreter. That is, the text is broken up into tokens that are then categorized. For example, \ttt{'('} might become \ttt{OPEN\_PAREN}, \ttt{"lambda"} might become \ttt{SYMBOL}, \ttt{"variable-name"} might become \ttt{SYMBOL}, and \ttt{123.45} might become \ttt{NUMBER}. The output of the lexer is a list of tokens. Part of the trick is to ensure that after reading an open parenthesis, the next token is not grabbed as part of the open parenthesis.

\myexercise{3}{chapter-advanced-oop}{This exercise involves the interpreter we wrote in the chapter.}
Design a parser for the language described by the interpreter. The idea is to tokenize a raw string, then parse the tokens to create an abstract syntax tree that represents the program. A good starting point would be to parse \emph{all} parenthesized expressions into what we will call \ttt{SExprNode}, then traverse over the tree to ``correct'' them into their true nodes, e.g., whether they are \ttt{IfNode}, \ttt{LetNode}, and so forth. Realistically, all programs in our language are, at their core, either primitive values or s-expressions.

\myexercise{2}{chapter-advanced-oop}{This exercise involves the interpreter we wrote in the chapter.}
The Scheme programming language and its derivatives support \emph{code quotation}, i.e., the ability to convert an evaluable expression into data. As an example, if we evaluate \ttt{new QuoteNode(new VarNode("x"))}, we receive a symbol as the output, rather than the evaluated symbol via environment lookup. Add the \ttt{QuoteNode} class to your interpreter.

\myexercise{3}{chapter-advanced-oop}{In Chapter~\ref{chapter-crl}, we discussed tail recursion and an action performed by some programming languages known as tail-call optimization. We know that we can convert any (tail) recursive algorithm into one that uses a loop, and we described said process in the chapter. There is yet another approach that we can mimic in Java with a bit of trickery and interfaces.}

The problem with tail recursion (and recursion in general) in Java is the fact that it does not convert tail calls into iteration, which means the stack quickly overflows with activation records. We can make use of a \emph{trampoline} to force the recursion into iteration through \emph{thunks}. In essence, we have a tail recursive method that returns either a value or makes a tail recursive call, such as the factorial example below.\footnote{We omit the driver method to shorten the code, as the important part lies inside the recursive implementation.} Inside our base case, we invoke the \ttt{done} method with the accumulator value. Otherwise, we invoke the \ttt{call} method containing a lambda expression of no arguments, whose right-hand side is a recursive call to \ttt{factTR}. Functions, or lambda expressions, that receive no arguments are called thunks.


\begin{lstlisting}[language=MyJava]
import static Assertions.assertAll;
import static Assertions.assertEquals;

class FactorialTailRecursiveTester {
  
  @Test 
  void testFactTailRecursiveTrampoline() {
    assertAll(
      () -> assertEquals(BigInteger.valueOf(1), 
                         factTailCall(BigInteger.valueOf(0), BigInteger.ONE)),
      () -> assertEquals(BigInteger.valueOf(120), 
                         factTailCall(BigInteger.valueOf(5), BigInteger.ONE)),
      () -> assertDoesNotThrow(StackOverflowException.class,
                               () -> factTailCall(BigInteger.valueOf(50000), 
                                                  BigInteger.ONE)),
    )
  }
}
\end{lstlisting}

\begin{lstlisting}[language=MyJava]
import java.lang.BigInteger;

class FactorialTailRecursive {
 
  /**
   * Tail-recursive factorial function. Uses BigInteger to 
   * avoid number overflow and thunks to avoid stack overflow.
   * @param n - the number to compute the factorial of.
   * @param ac - the accumulator.
   * @return a tail call that is either done or not done.
   */
  static ITailCall<BigInteger> factTailCall(BigInteger n, BigInteger ac) {
    if (n.equals(BigInteger.ZERO)) {
      return TailCallUtils.done(ac);
    } else {
      return TailCallUtils.call(() -> 
               factTailCall(n.subtract(BigInteger.ONE), ac.multiply(n)));
    }
  }
}
\end{lstlisting}

The idea is that we have a helper class and method, namely \ttt{invoke}, that continuously applies the thunks, \textbf{inside a while loop}, until the computation is done. The trampoline analogy is used because we bounce on the trampoline while invoking thunks and jump off when we are ``done.''

First, design the generic \ttt{ITailCall<T>} interface. It should contain only one (non-default) method: \ttt{ITailCall<T> apply()}, which is necessary for the \ttt{invoke} method. The remaining methods are all default, meaning they must have a body. Design the \ttt{boolean isDone()} method to always return false. Design the \ttt{T getValue()} method to simply return \ttt{null}. Finally, design the \ttt{T invoke()} method that stores a local variable and constantly calls \ttt{apply} on itself until it is ``done.'' 

Second, design the \ttt{TailCallUtils} final class to contain a private constructor (this class will only utilize and define two static methods). The two methods are as follows:

\begin{itemize}
  \item \ttt{static <T> ITailCall<T> call(ITailCall next)}, which receives and returns the next tail call to apply. This definition should be exactly one line long and as simple as it seems.
  \item \ttt{static <T> ITailCall<T> done(T val)}, which receives the value to return from the trampoline. We need to create an instance of an interface, which sounds bizarre, but is possible only when we provide an implementation of its methods. So, return a \ttt{new ITailCall<>()}, and inside its body, override the \ttt{isDone} and \ttt{getValue} methods with the correct bodies.
\end{itemize}

Finally, run the factorial test that we provided earlier in its JUnit suite. It should pass and not stack overflow, hence the inclusion of an \ttt{assertDoesNotThrow} call.

\myexercise{3}{chapter-arrays-collections}{Recall the unification exercise from Chapter~\ref{chapter-arrays-collections}. We can take the idea of unification a step further, which is the basis for almost all logic programming languages such as Prolog. For instance, take the expression \ttt{p(X, f(Y))}; attempting to unify this with \ttt{p(q(r(x)), f(b(x)))} returns a unification assignment of \ttt{X : q(r(x)), Y : b(x)}. It is possible for a unification to not return any possible assignment. As an example, unifying \ttt{p(a, b)} with \ttt{p(Y, Y)} returns an empty assignment because it is not possible to unify \ttt{a} with \ttt{Y}, then unify \ttt{b} with \ttt{Y}.}

Design three classes: \ttt{Variable}, \ttt{Constant}, and \ttt{Predicate}. Each of these should implement the \ttt{IUnifiable} interface, which supplies one method: \ttt{Assignment unify(IUnifiable u, Assignment as)}. An \ttt{Assignment} is simply a mapping of \ttt{IUnifiable} objects to \ttt{IUnifiable} objects, resembling a map data structure. Variables in this small language will be represented as uppercased letters, whereas constants are lowercase. If two \ttt{IUnifiable} objects cannot be unified, then \ttt{unify} should return \ttt{null}.

Constants are straightforward: constants can only be unified with other constants if they are equivalent. Constants can only be unified with variables if that variable does not have an existing assignment and, if it does, it must be equal to \ttt{this} constant. Constants cannot be unified with predicates.

Variables can only be unified with constants if the variable does not have an existing assignment and, if it does, it must be equal to the constant passed as an argument. Variables can only be unified with other variables if at least one is bound to a constant; if they are both bound, then they must be equivalent constants. 

Predicates can only be unified with variables if the variable does not have an existing assignment and, if it does, it must be equal to \ttt{this} predicate. Predicates can only be unified with predicates if it is possible to successfully unify all of its arguments. E.g., \ttt{p(a, z(b), c)} unifies with \ttt{p(X, z(Y), Z)} because we return the assignment \ttt{X : a, Y : b, Z : c}. 

\myexercise{3}{chapter-advanced-oop}{In this series of problems, you will design several methods that act on very large \emph{natural numbers} resembling the \ttt{BigInteger} class. You \emph{\textbf{cannot}} use any methods from \ttt{BigInteger}, or the \ttt{BigInteger} class itself.} 

\begin{enumerate}[label=(\alph*)]
    \item Design the \ttt{BigNat} class, which has a constructor that receives a string. The \ttt{BigNat} class stores a \ttt{List<Integer>} as an instance variable. You will need to convert the given string into said list. Store the digits in reverse order, i.e., the least-significant digit (the ones digit) of the number is the first element of the list.
    \item Override the \ttt{public String toString()} method to return a string representation of the \ttt{BigNat} object. 
    % In particular, place commas between the digits where necessary, e.g., \ttt{"1,000"} and \ttt{"12,345,678"}.
    \item Design the \ttt{BigNat clone()} method that returns a new \ttt{BigNat} instance that contains the same number.
    \item Override the \ttt{public boolean equals(Object obj)} method to compare two \ttt{BigNat} values for equality. Remember that you have to cast the given parameter to an instance of the \ttt{BigNat} class. 
    \item Implement the \ttt{Comparable<BigNat>} interface, and override the \ttt{public int compareTo(BigNat b)} method to return the sign of the result of comparing the given \ttt{BigNat} (which we will call $b$) to \ttt{this} \ttt{BigNat} (which we will call $a$). Namely, if $a < b$, return $-1$, if $a > b$, return $1$, otherwise return $0$.
    \item Design the \ttt{void add(BigNat bn)} method, which adds a \ttt{BigNat} to \ttt{this} \ttt{BigNat}. The method should not return anything. Note: this problem is harder than it may look at first glance!
    \item Design the \ttt{void sub(BigNat bn)} method, which subtracts a \ttt{BigNat} from \ttt{this} \ttt{BigNat}. If the subtrahend (the right-hand side of the subtraction) is greater than the minuend, the result is zero. Over the natural numbers, this is called the \emph{monus} operator.
    \item Design the \ttt{void mul(BigNat bn)} method, which multiplies a \ttt{BigNat} with \ttt{this} \ttt{BigNat}. Note: remember how we implement multiplication recursively? You shouldn't use recursion for this problem, but what \emph{is} multiplication? Think about the performance implications of this approach.
    \item Design the \ttt{void div(BigNat bn)} method, which divides a \ttt{BigNat} with \ttt{this} \ttt{BigNat}. If the divisor is greater than the dividend, assign the dividend to be zero. If the divisor is zero, do nothing at all. Otherwise, perform integer division. Note: we can implement division recursively. You shouldn't use recursion for this problem, but what \emph{is} division? Think about the performance implications of this approach.
\end{enumerate}

\myexercise{3}{chapter-advanced-oop}{Quine's method of truth resolution~\cite{methodsoflogic} is a method of automatically reasoning about the truth of a propositional logic statement (recall the exercise from Chapter~\ref{chapter-crl}). The method is as follows:}

\begin{enumerate}
    \item Choose an atom $P$ from the statement. Consider two cases: when $P$ is true and when $P$ is false. Derive the consequences of each case. The rules follow those of the propositional logic connectives.
    \item Repeat this process for each sub-statement until there are no more sub-statements, and you have only true or false results. If you have \emph{both} true and false results, the statement is a contingency. If all branches lead to true, the statement is a tautology. If all branches lead to false, the statement is a contradiction. 
\end{enumerate}

Design several classes to represent a series of well-formed schemata in propositional logic, namely \ttt{CondNode}, \ttt{BicondNode}, \ttt{NegNode}, \ttt{AndNode}, \ttt{OrNode}, and \ttt{AtomNode}, all of which extend a root \ttt{Node} class, similar to our representation of the abstract syntax trees within the ASPL interpreter. Then, design three methods: \ttt{boolean isTautology(Node t)}, \ttt{boolean isContingency(Node t)}, and \ttt{boolean isContradiction(Node t)}, which return whether or not the given statement is a tautology, contingency, or contradiction, respectively. You may assume that the input is a well-formed schema. Note that only one of these methods needs a full-fledged recursive traversal over the data; the other two can be implemented in terms of the first.