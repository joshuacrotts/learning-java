\section{A First Glimpse at Java}

It makes little sense to avoid the topic at hand, so let's jump right in and write a program! 
We have seen \emph{functions}\index{function} before, as well as some mathematics operations, perhaps in a different (language) context. 

\myexample{Our program will consist of a function that converts a given temperature in Fahrenheit to Celsius.}

\begin{lstlisting}[language=MyJava]
class TempConverter {
  
  /**
   * Converts a temperature from Fahrenheit to Celsius.
   * @param f - temperature in degrees Fahrenheit.
   * @return temperature in degrees Celsius.
   */
  static double fToC(double f) {
    return 0.0;
  }
}
\end{lstlisting}

All code, in Java, belongs to a \emph{class}\index{class}. 
Classes have much more complex and concrete definitions that we will investigate in due time, but for now, we may think of them as the homes of our functions. 
By the way, functions in Java are called \emph{methods}\index{method}.\footnote{The reasoning is simple: a method belongs to a class. Other programming languages, e.g., C++ or Python, do not restrict the programmer to writing code only within a class. Thus, there is a distinction between functions, which do not reside within a class, and methods.} 
Again, this slight terminology differentiation is not without its reasons, but for all intents and purposes, functions are methods and vice versa. 
The class we have defined in the previous listing is called \ttt{TempConverter}, giving rise to believe that the class does something related to temperature conversion.

We wrote the \ttt{fToC} method \emph{signature}\index{method signature}\index{signature}, whose \emph{return type}\index{return type} is a \ttt{double}, and has one \emph{parameter}\index{parameter}, which is also a \ttt{double}. 
A \ttt{double} is a floating-point value, meaning it potentially has decimals. 
For our method, this choice makes sense, because if we were to instead receive an integer \ttt{int}, we would not be able to convert temperatures such as \ttt{35.5} degrees Fahrenheit to Celsius. 

The \ttt{static} keyword that we wrote also has significance, but for now, consider it a series of six mandatory key presses (plus one for the space thereafter).

Above this method \emph{signature}\index{signature} is a Java documentation comment, providing a brief summary of the method's purpose, as well as the data it receives as parameters and its return value, should it be necessary. 

Inside its \emph{body}\index{method body}\index{body} lies a single \ttt{return}, in which we return \ttt{0.0}. 
The \emph{return value}\index{return value} is what a \emph{method call}\index{method call}, or \emph{method invocation}\index{method invocation}, resolves to. 
For example, if we were to call \ttt{fToC} with any arbitrary \ttt{double} value, the call would be substituted with \ttt{0.0}. 
Passing a value, or an \emph{argument}\index{argument}, to a method is sometimes called \emph{method application}\index{method application}.

\begin{verbnobox}[\small]
fToC(5)     -> 0.0
fToC(78)    -> 0.0
fToC(-3123) -> 0.0
\end{verbnobox}

The \ttt{fToC} method is meaningless without a complete implementation. 
Before we write the rest of the method body, however, we should design \emph{test cases}\index{test cases} to ensure it works as expected. 
Test cases verify the correctness (or incorrectness) of a method. 
We, as the readers, know how to convert a temperature from Fahrenheit to Celsius, but telling a computer to do such a conversion is not as obvious at first glance. 
To test our methods, we will use the \emph{JUnit} testing framework\index{JUnit}. Creating tests for \ttt{fToC} is straightforward; we will make a second class called \ttt{TempConverterTester} to house a single method: \ttt{testFToC}.

\begin{lstlisting}[language=MyJava]
class TempConverterTester {

  @Test
  void testfToC() {
    Assertions.assertAll(
      () -> Assertions.assertEquals(0, TempConverter.fToC(32)));
  }
}
\end{lstlisting}

We want JUnit to recognize that \ttt{fToC} contains testing code, so we prepend the \ttt{@Test} annotation to the method signature. 
In its body, we call \ttt{Assertions\-.assertAll}, which receives a series of methods that are ran in succession. 
In our case, we want to assert that our \ttt{fToC} method should return \ttt{0} degrees Celsius when given a temperature of \ttt{32} degrees Fahrenheit. 
The first parameter to \ttt{assertEquals} is the expected value of the test, i.e., what we want (and know) it to produce. 
The second parameter is the actuaspringerl value of the test, i.e., what our code produces. 

When writing tests, it is important to consider \emph{edge cases}\index{edge case} and all possible branches of a method implementation. 
Edge cases are inputs that are possibly missed by an implementation, e.g., \ttt{-40}, since it is the same in both Fahrenheit and Celsius, or \ttt{0}. 
So, let us add a few more test cases.\footnote{To condense our code, we exclude the \ttt{TempConverter} class name out of conciseness. On the other hand, we can omit the \ttt{Assertions} class name by importing the two methods as shown in the listing.}

\enlargethispage{-1\baselineskip}
\begin{lstlisting}[language=MyJava]
import static Assertions.assertAll;
import static Assertions.assertEquals;

class TempConverterTester {

  @Test
  void testfToC() {
    assertAll(
      () -> assertEquals(0, fToC(32)),
      () -> assertEquals(100, fToC(212)),
      () -> assertEquals(-40, fToC(40)),
      () -> assertEquals(-17.778, fToC(0), .01),
      () -> assertEquals(-273.15, fToC(-459), .01));
  }
}
\end{lstlisting}

Computers are, unfortunately, not perfect; floating-point operations may result in precision/rounding errors. 
So, as an optional third argument to \ttt{assertEquals}, we can provide a \emph{delta}\index{delta argument}, which allows for precision up to a certain amount to be accepted as a valid answer. 
For example, our tolerance for the fourth and fifth test cases is \ttt{0.01}, meaning that if our actual value is less than~$\pm 0.01$ away from the expected value, the test case succeeds.

Now that we have copious amounts of tests, we can write our method definition. Of course, it is trivial and follows the well-known mathematical definition.

\begin{lstlisting}[language=MyJava]
class TempConverter {

  /**
   * Converts a temperature from Fahrenheit to Celsius.
   * @param temperature in F.
   * @return temperature in C.
   */
  static double fToC(double d) {
    return (d - 32) * (5.0 / 9.0)
  }
}
\end{lstlisting}

The definition of \ttt{fToC} brings up a few points about Java's type system. 
The primitive mathematics operations account for the types of its arguments. 
So, for instance, subtracting two integers will produce another integer. 
More noteworthy, perhaps, a division of two integers produces another integer, even if that result seems to be incorrect. 
Thus, \ttt{5 / 9} results in the integer \ttt{0}. 
If, however, we treat at least one of the operands as a floating-point value, we receive a correct result of approximately \ttt{0.555555}: \ttt{5.0 / 9}. 
Java by default uses the standard order of operations when evaluating mathematics expressions, so we force certain operations to occur first via parentheses.\footnote{By ``standard,'' we mean the widely-accepted paradigm of parentheses first, then exponents, followed by left-to-right multiplication/division, and finally left-to-right addition/subtraction.}

Unlike some programming languages that are \emph{dynamically-typed}, e.g., Scheme, Python, JavaScript, the Java programming language requires the programmer to specify the types of variables~\citep[Chapter~5]{copl}.\footnote{In Java 10, the \ttt{var} keyword was introduced, which automatically infers the type of a given expression~\citep[Chapter~14]{java}.} 
Java has several default \emph{primitive datatypes}\index{primitive datatypes}, which are the simplest reducible form of a variable. 
Such types include \ttt{int}, \ttt{char}, \ttt{double}, \ttt{boolean}, and others.
Integers, or \ttt{int}, are any positive or negative number without decimals. 
Doubles, or \ttt{double}, are values with decimals.\footnote{The \ttt{double} data type cannot represent all real numbers. For example, the value $\pi$ is an irrational (real) number, and thus cannot be represented exactly in a finite number of bits.} 
Characters, or \ttt{char}, are a single character enclosed by single quotes, e.g., \ttt{\q{X}\q}. 
Finally, booleans, or \ttt{boolean}, are either \ttt{true} or \ttt{false}. 
There are other Java data types that specify varying levels of precision for given values. 
Integers are 32-bit signed values, meaning they have a range of $[-2^{31}, 2^{31})$.
The \ttt{short} data type, on the contrary, is 16-bit signed. 
Beyond this is the \ttt{byte} data type that, as its name suggests, stores 8-bit signed integers. 
Floating-point values are more tricky, but while \ttt{double} uses 64 bits of precision, the \ttt{float} data type uses 32 bits of precision.

\myexample{Let us design a method that receives two three-dimensional vectors and returns the distance between the two.} 
We can, effectively, think of this as the distance between two points in a three-dimensional plane. 
Therefore, because each vector contains three components, we need six parameters, where each triplet represents the vectors~$v_1$ and~$v_2$.

\begin{lstlisting}[language=MyJava]
class VectorDistance {

  /**
   * Computes the distance between two given vectors:
   * v_1=(x_1, y_1, z_1) and v_2=(x_2, y_2, z_2)
   */
  static double computeDistance(double x1, double y1, double z1, 
                                double x2, double y2, double z2) {
    return 0.0;
  }
}
\end{lstlisting}

Again, we start by writing the appropriate method signature with its respective parameters and a Java documentation comment explaining its purpose. 
For tests, we know that the distance between two Cartesian points in a three-dimensional plane is 

\[
D(v_1, v_2) = \sqrt{(x_1 - x_2)^2 + (y_1 - y_2)^2 + (z_1 - z_2)^2}
\]

So, let's now write a few test cases with some arbitrarily-chosen points that we can verify with a calculator or manual computation.

\begin{lstlisting}[language=MyJava]
import static Assertions.assertAll;
import static Assertions.assertEquals;

class VectorDistanceTester {

  @Test
  void testComputeDistance() {
    assertAll(
      assertEquals(8.66, computeDistance(3, 2, 1, 8, 7, 6), .01),
      assertEquals(12.20, computeDistance(0, 0, 0, 8, 7, 6), .01),
      assertEquals(8.30, computeDistance(-8, -2, 1, 0, 0, 0), .01));
  }
}
\end{lstlisting}

Notice again our use of the optional delta parameter to allow us a bit of leeway with the rounding of our answer. 
Fortunately, the implementation of our method is just a retelling of the mathematical definition.

\enlargethispage{-2\baselineskip}
\begin{lstlisting}[language=MyJava]
class VectorDistance {

  /**
   * Computes the distance between two given vectors:
   * v_1=(x_1, y_1, z_1) and v_2=(x_2, y_2, z_2).
   * @param x1 - x component of first vector.
   * @param y1 - y component of first vector.
   * @param z1 - z component of first vector.
   * @param x2 - x component of second vector.
   * @param y2 - y component of second vector.
   * @param z2 - z component of second vector.
   * @return distance between v1 and v2.
   */
  static double computeDistance(double x1, double y1, double z1, 
                                double x2, double y2, double z2) {
    return Math.sqrt(Math.pow(x1 - x2, 2) 
                   + Math.pow(y1 - y2, 2)
                   + Math.pow(z1 - z2, 2));
  }
}
\end{lstlisting}

We make prolific use of Java's \ttt{Math} library in designing this method; we use the \ttt{sqrt} method for computing the square root of our result, as well as \ttt{pow} to square each intermediate difference.

\myexample{Slope-intercept is an incredibly common algebra and geometry problem, and even pokes its way into machine learning at times when computing best-fit lines.} 
Let's design two methods, both of which receive two points $(x_1, y_1)$, $(x_2, y_2)$. 
The first method returns the slope~$m$ of the points, and the second returns the y-intercept~$b$ of the line. 
Their respective signatures are straightforward---each set of points is represented by two integer values and return doubles.
\begin{align*}
m &= \dfrac{y_2-y_1}{x_2-x_1}\\
b &= y_1 - mx_1
\end{align*}
\enlargethispage{3\baselineskip}
\begin{lstlisting}[language=MyJava]
class SlopeIntercept {

  /**
   * Computes the slope of the line represented by the two points.
   * @param x1 - x-coordinate of point 1.
   * @param y1 - y-coordinate of point 1.
   * @param x2 - x-coordinate of point 2.
   * @param y2 - y-coordinate of point 2.
   * @return slope of points.
   */
  static double slope(int x1, int y1, int x2, int y2) { return 0.0; }

  /**
   * Computes the y-intercept of the line represented by the two points.
   * @param x1 - x-coordinate of point 1.
   * @param y1 - y-coordinate of point 1.
   * @param x2 - x-coordinate of point 2.
   * @param y2 - y-coordinate of point 2.
   * @return y-intercept of line represented by points.
   */
  static double yIntercept(x1, int y1, int x2, int y2) { return 0.0; }
}
\end{lstlisting}

The tests that we write are verifiable by a calculator or mental math. 
Our \ttt{yIntercept} method depends on a successful implementation of \ttt{slope}, as designated by the formula of the former. 
In the next chapter, we will consider cases that invalidate the formula, e.g., when two points share an~$x$ coordinate, in which the slope is undefined for those points.

\begin{lstlisting}[language=MyJava]
import static Assertions.assertAll;
import static Assertions.assertEquals;

class SlopeInterceptTester {
  
  @Test
  void testSlope() {
    assertAll(
      () -> assertEquals(1, slope(0, 0, 1, 1)),
      () -> assertEquals(0, slope(0, 0, 1, 0)),
      () -> assertEquals(2, slope(8, 4, 2, 4)),
      () -> assertEquals(0.5, slope(-1, 5, 3, 7)));
  }

  @Test
  void testYIntercept() {
    assertAll(
      () -> assertEquals(0, yIntercept(0, 0, 1, 1)),
      () -> assertEquals(0, yIntercept(0, 0, 1, 0)),
      () -> assertEquals(4, yIntercept(8, 4, 2, 4)),
      () -> assertEquals(5.5, yIntercept(-1, 5, 3, 7)));
  }
}
\end{lstlisting}

And the implementation of those two methods follows directly from the mathematical definitions. 
We replace our temporary \ttt{0.0} return values with the appropriate expressions, and all tests pass.

\begin{lstlisting}[language=MyJava]
class SlopeIntercept {

  /**
   * Computes the slope of the line represented by the two points.
   * @param x1 - x-coordinate of point 1.
   * @param y1 - y-coordinate of point 1.
   * @param x2 - x-coordinate of point 2.
   * @param y2 - y-coordinate of point 2.
   * @return slope of points.
   */
  static double slope(int x1, int y1, int x2, int y2) {
    return (y2 - y1) / (x2 - x1);
  }

  /**
   * Computes the y-intercept of the line represented by the two points.
   * @param x1 - x-coordinate of point 1.
   * @param y1 - y-coordinate of point 1.
   * @param x2 - x-coordinate of point 2.
   * @param y2 - y-coordinate of point 2.
   * @return y-intercept of points.
   */
  static double yIntercept(x1, int y1, int x2, int y2) {
    return y1 - slope(x1, y1, x2, y2) * x1;
  }
}
\end{lstlisting}

\myexample{We are starting to get used to some of Java's verbosity!} 
Let's now design a method that, when given a (numeric) value of~$x$, evaluates the following quartic formula:

\[
q(x) = 4x^4 + 7x^3 + 21x^2 - 65x + 3
\]

Its signature is straightforward: we receive a value of~$x$, namely a \ttt{double}, and return a \ttt{double}, since our operations work over \ttt{double} values. 
Again, we return zero as a temporary solution to ensure the program successfully compiles.

%\enlargethispage{-1\baselineskip}
\begin{lstlisting}[language=MyJava]
class QuarticFormulaSolver {

  /**
   * Evaluates the equation 4x^4 + 7x^3 + 21x^2 - 65x + 3.
   * @param x - the input variable.
   * @return the result of the expression after substituting x.
   */
  static double solveQuartic(double x) { return 0.0; }
}
\end{lstlisting}

Test cases are certainly warranted, but may be a bit tedious to compute by hand, so we recommend using a verified calculator to compute expected solutions!\footnote{Remember that the coefficient is applied \emph{after} applying the exponent to the variable. That is, if $x=3$, then $4x^3$ is equal to $4 \cdot (3)^3$, which resolves to $4 \cdot 27$, which resolves to $108$.}

\begin{lstlisting}[language=MyJava]
import static Assertions.assertAll;
import static Assertions.assertEquals;

class QuarticFormulaSolverTester {

  @Test
  void testSolveQuartic() {
    assertAll(
      () -> assertEquals(3, solveQuartic(0)),
      () -> assertEquals(510, solveQuartic(3)),
      () -> assertEquals(229878, solveQuartic(15)),
      () -> assertEquals(313445.1875, solveQuartic(16.25)));
  }
}
\end{lstlisting}

We write tests \emph{before} the method implementation because we know, intuitively, how to solve an equation for a variable, whereas a computer has to be told how to solve this task. 
Fortunately for us, a quartic equation solver is nothing more than returning the result of the expression. 
We have to use the exponential \ttt{Math.pow} method again (or conjoin several multiplicatives of~$x$), but otherwise, nothing new is utilized.

\enlargethispage{\baselineskip}
\begin{lstlisting}[language=MyJava]
class QuarticFormulaSolver {

  /**
   * Evaluates the equation 4x^4 + 7x^3 + 21x^2 - 65x + 3.
   * @param x - the input variable.
   * @return the result of the expression after substituting x.
   */
  static double solveQuartic(double x) {
    return 4 * Math.pow(x, 4) + 7 * Math.pow(x, 3) + 
               21 * Math.pow(x, 2) - 65 * x + 3;
  }
}
\end{lstlisting}

And, as expected, all tests pass. 
With only methods and math operations at our disposal, the capabilities of said methods are quite limited. 
% Let's start revamping our tool set by reintroducing strings.

\subsection{The Main Method}
Java programming tutorials are quick to throw a lot of information at the reader/viewer, and our textbook is no exception to this practice. 
Unfortunately, Java is a verbose programming languages, and to begin designing a method, we must wrap its definition inside a class. 
After this step, we can design the \ttt{static} method implementation. 
Readers no doubt question the significance of \ttt{static}. 
For the first few chapters, we will intentionally omit its definition, as an explanation would almost certainly confuse the reader coming from another language. 
Therefore, for the time being, simply view it as six characters, plus a space, that must be typed in order to write a method that we can then test with JUnit. 

Imagine, however, a situation where we want to output the result of an expression without using a test, as we will do in many examples. 
Java requires a \ttt{main} method in any executable Java class that does not use tests. 
For the sake of completeness, let's now write the traditional ``Hello, world!'' program in Java using the \ttt{main} method instead of tests.

\begin{lstlisting}[language=MyJava]
class MainMethod {

  public static void main(String[] args) {
    System.out.println("Hello, world!");
  }
}
\end{lstlisting}

Yikes, that is a lot of required code to output some text to the console; what does \ttt{public} mean, and what are those \ttt{[]} brackets after the \ttt{String} word?
Once again, we will not detail their importance, but view them as more mandatory characters to type when writing a \ttt{main} method. 
The only word we \emph{will} explain is \ttt{void}, which means that the method does not return a value. 
If the readers are coming from a functional programming language, e.g., Scheme/Racket or OCaml, then it is almost certainly the case that they never worked with methods that did not return a value nor received no arguments.\footnote{A method of no arguments is called a \emph{thunk}, or a \emph{nullary} method.} 
The \ttt{println} method, for example, has no return value; its significance comes from the fact that it outputs text to the terminal/console, which is a \emph{method side-effect}\index{side-effect}. 
We'll come back to what all of this means in subsequent chapters, but we could not avoid at least briefly mentioning it and its existence in the Java language.

% \subsection{Quick Check Questions}
% \begin{enumerate}[label=(\arabic*)]
%   \item \mcq{In Java, what is the term used to describe functions that belong to a class?}{Procedures,Subroutines,Methods,Functions}

%   \item \mcq{Which keyword is used in Java to define a class member that can be accessed without needing an instance of the class?}{\texttt{public},\texttt{private},\texttt{static},\texttt{final}}

%   \item \mcq{What is the return type of the \texttt{fToC} method in the provided example?}{\texttt{int},\texttt{float},\texttt{double},\texttt{String}}

%   \item \mcq{What type of comment is used to document a method in Java?}{Single-line comment (\texttt{//}),Multi-line comment (\texttt{/* /}),JavaDoc comment (\texttt{/* */}),Inline comment}
  
%   \item \mcq{What does the \texttt{@Test} annotation indicate in a Java class?}{The method is the main entry point of the program.,The method is to be executed as a test case.,The method is deprecated.,The method is thread-safe.}

%   \item \mcq{What is the purpose of the \texttt{assertEquals} method in JUnit testing?}{To check if a variable is initialized,To compare two values for equality,To verify if an exception is thrown,To ensure a loop executes a specific number of times}

%   \item \mcq{What is the significance of using \texttt{Math.pow} in the method implementation for computing distance between vectors?}{It converts integers to double.,It performs square root calculation.,It raises a number to the power of another number.,It rounds off decimal values to a set precision.}

%   \item \mcq{What is the primitive datatype for a variable that can hold decimal values in Java?}{\texttt{char},\texttt{int},\texttt{double},\texttt{boolean}}

%   \item \mcq{How are floating-point arithmetic operations handled in Java when involving both integers and floating-point numbers?}{The operation is not allowed and causes a compilation error.,The result is always rounded down to the nearest integer.,The operation must be explicitly cast to a floating-point operation.,The operation automatically promotes to a floating-point operation.}

%   \item \mcq{In the context of the \texttt{main} method in Java, what does \texttt{void} indicate?}{The method returns an integer.,The method returns nothing.,The method must return a string.,The method returns an array of strings.}
% \end{enumerate}