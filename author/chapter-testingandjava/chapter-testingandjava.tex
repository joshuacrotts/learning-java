\setcounter{excounter}{1}
\setcounter{examplecounter}{1}
\chapter{Testing and Java Basics}
\label{chapter-testingandjava} % Always give a unique label
% use \chaptermark{}
% to alter or adjust the chapter heading in the running head

\abstract*{This chapter introduces readers to the basics of Java programming and testing. We will cover the fundamentals of Java syntax, including variables and data types. At the same time, we discuss the importance of test-driven development and how to write test cases in Java for methods. Our emphasis on testing makes work of the fact that testing is a critical part of writing correct and robust software. Test-driven development, in particular, is utilized as a ``design recipe,'' of sorts, to guide programmers to solve a task. }

\abstract{This chapter introduces readers to the basics of Java programming and testing. We will cover the fundamentals of Java syntax, including variables and data types. At the same time, we discuss the importance of test-driven development and how to write test cases in Java for methods. Our emphasis on testing makes work of the fact that testing is a critical part of writing correct and robust software. Test-driven development, in particular, is utilized as a ``design recipe,'' of sorts, to guide programmers to solve a task. }

\section{A First Glimpse at Java}

It makes little sense to avoid the topic at hand, so let's jump right in and write a program! 
We have seen \emph{functions}\index{function} before, as well as some mathematics operations, perhaps in a different (language) context. 

\myexample{Our program will consist of a function that converts a given temperature in Fahrenheit to Celsius.}

\begin{lstlisting}[language=MyJava]
class TempConverter {
  
  /**
   * Converts a temperature from Fahrenheit to Celsius.
   * @param f temperature in degrees Fahrenheit.
   * @return temperature in degrees Celsius.
   */
  static double fToC(double f) {
    return 0.0;
  }
}
\end{lstlisting}

All code, in Java, belongs to a \emph{class}\index{class}. 
Classes have much more complex and concrete definitions that we will investigate in due time, but for now, we may think of them as the homes of our functions. 
By the way, functions in Java are called \emph{methods}\index{method}.\footnote{The reasoning is simple: a method belongs to a class. Other programming languages, e.g., C++ or Python, do not restrict the programmer to writing code only within a class. Thus, there is a distinction between functions, which do not reside within a class, and methods.} 
Again, this slight terminology differentiation is not without its reasons, but for all intents and purposes, functions are methods and vice versa. 
The class we have defined in the previous listing is called \ttt{TempConverter}, giving rise to believe that the class does something related to temperature conversion.

We wrote the \ttt{fToC} method \emph{signature}\index{method signature}\index{signature}, whose \emph{return type}\index{return type} is a \ttt{double}, and has one \emph{parameter}\index{parameter}, which is also a \ttt{double}. 
A \ttt{double} is a floating-point value, meaning it potentially has decimals. 
For our method, this choice makes sense, because if we were to instead receive an integer \ttt{int}, we would not be able to convert temperatures such as \ttt{35.5} degrees Fahrenheit to Celsius. 

The \ttt{static} keyword that we wrote also has significance, but for now, consider it a series of six mandatory key presses (plus one for the space thereafter).

Above this method \emph{signature}\index{signature} is a Java documentation comment, providing a brief summary of the method's purpose, as well as the data it receives as parameters and its return value, should it be necessary. 

Inside its \emph{body}\index{method body}\index{body} lies a single \ttt{return}, in which we return \ttt{0.0}. 
The \emph{return value}\index{return value} is what a \emph{method call}\index{method call}, or \emph{method invocation}\index{method invocation}, resolves to. 
For example, if we were to call \ttt{fToC} with any arbitrary \ttt{double} value, the call would be substituted with \ttt{0.0}. 
Passing a value, or an \emph{argument}\index{argument}, to a method is sometimes called \emph{method application}\index{method application}.

\begin{verbnobox}[\small]
fToC(5)     => 0.0
fToC(78)    => 0.0
fToC(-3123) => 0.0
\end{verbnobox}

The \ttt{fToC} method is meaningless without a complete implementation. 
Before we write the rest of the method body, however, we should design \emph{test cases}\index{test cases} to ensure it works as expected. 
Test cases verify the correctness (or incorrectness) of a method. 
We, as the readers, know how to convert a temperature from Fahrenheit to Celsius, but telling a computer to do such a conversion is not as obvious at first glance. 
To test our methods, we will use the \emph{JUnit} testing framework\index{JUnit}. Creating tests for \ttt{fToC} is straightforward; we will make a second class called \ttt{TempConverterTester} to house a single method: \ttt{testFToC}.

\begin{lstlisting}[language=MyJava]
class TempConverterTester {

  @Test
  void testfToC() {
    Assertions.assertAll(
      () -> Assertions.assertEquals(0, TempConverter.fToC(32)));
  }
}
\end{lstlisting}

We want JUnit to recognize that \ttt{fToC} contains testing code, so we prepend the \ttt{@Test} annotation to the method signature. 
In its body, we call \ttt{Assertions\-.assertAll}, which receives a series of methods that are ran in succession. 
In our case, we want to assert that our \ttt{fToC} method should return \ttt{0} degrees Celsius when given a temperature of \ttt{32} degrees Fahrenheit. 
The first parameter to \ttt{assertEquals} is the expected value of the test, i.e., what we want (and know) it to produce. 
The second parameter is the actual value of the test, i.e., what our code produces. 

When writing tests, it is important to consider \emph{edge cases}\index{edge case} and all possible branches of a method implementation. 
Edge cases are inputs that are possibly missed by an implementation, e.g., \ttt{-40}, since it is the same in both Fahrenheit and Celsius, or \ttt{0}. 
So, let us add a few more test cases.\footnote{To condense our code, we exclude the \ttt{TempConverter} class name out of conciseness. On the other hand, we can omit the \ttt{Assertions} class name by importing the two methods as shown in the listing.}

\enlargethispage{-5\baselineskip}
\begin{lstlisting}[language=MyJava]
import static Assertions.assertAll;
import static Assertions.assertEquals;

class TempConverterTester {

  @Test
  void testfToC() {
    assertAll(
      () -> assertEquals(0, fToC(32)),
      () -> assertEquals(100, fToC(212)),
      () -> assertEquals(-40, fToC(40)),
      () -> assertEquals(-17.778, fToC(0), .01),
      () -> assertEquals(-273.15, fToC(-459), .01));
  }
}
\end{lstlisting}

Computers are, unfortunately, not perfect; floating-point operations may result in precision/rounding errors. 
So, as an optional third argument to \ttt{assertEquals}, we can provide a \emph{delta}\index{delta argument}, which allows for precision up to a certain amount to be accepted as a valid answer. 
For example, our tolerance for the fourth and fifth test cases is \ttt{0.01}, meaning that if our actual value is less than~$\pm 0.01$ away from the expected value, the test case succeeds.

Now that we have copious amounts of tests, we can write our method definition. Of course, it is trivial and follows the well-known mathematical definition.

\begin{lstlisting}[language=MyJava]
class TempConverter {

  /**
   * Converts a temperature from Fahrenheit to Celsius.
   * @param temperature in F.
   * @return temperature in C.
   */
  static double fToC(double d) {
    return (d - 32) * (5.0 / 9.0)
  }
}
\end{lstlisting}

The definition of \ttt{fToC} brings up a few points about Java's type system. 
The primitive mathematics operations account for the types of its arguments. 
So, for instance, subtracting two integers will produce another integer. 
More noteworthy, perhaps, a division of two integers produces another integer, even if that result seems to be incorrect. 
Thus, \ttt{5 / 9} results in the integer \ttt{0}. 
If, however, we treat at least one of the operands as a floating-point value, we receive a correct result of approximately \ttt{0.555555}: \ttt{5.0 / 9}. 
Java by default uses the standard order of operations when evaluating mathematics expressions, so we force certain operations to occur first via parentheses.\footnote{By ``standard,'' we mean the widely-accepted paradigm of parentheses first, then exponents, followed by left-to-right multiplication/division, and finally left-to-right addition/subtraction.}

Unlike some programming languages that are \emph{dynamically-typed}, e.g., Scheme, Python, JavaScript, the Java programming language requires the programmer to specify the types of variables~\citep[Chapter~5]{copl}.\footnote{In Java 10, the \ttt{var} keyword was introduced, which automatically infers the type of a given expression~\citep[Chapter~14]{java}.} 
Java has several default \emph{primitive datatypes}\index{primitive datatypes}, which are the simplest reducible form of a variable. 
Such types include \ttt{int}, \ttt{char}, \ttt{double}, \ttt{boolean}, and others.
Integers, or \ttt{int}, are any positive or negative number without decimals. 
Doubles, or \ttt{double}, are values with decimals.\footnote{The \ttt{double} data type cannot represent all real numbers. For example, the value $\pi$ is an irrational (real) number, and thus cannot be represented exactly in a finite number of bits.} 
Characters, or \ttt{char}, are a single character enclosed by single quotes, e.g., \ttt{\q{X}\q}. 
Finally, booleans, or \ttt{boolean}, are either \ttt{true} or \ttt{false}. 
There are other Java data types that specify varying levels of precision for given values. 
Integers are 32-bit signed values, meaning they have a range of $[-2^{31}, 2^{31})$.
The \ttt{short} data type, on the contrary, is 16-bit signed. 
Beyond this is the \ttt{byte} data type that, as its name suggests, stores 8-bit signed integers. 
Floating-point values are more tricky, but while \ttt{double} uses 64 bits of precision, the \ttt{float} data type uses 32 bits of precision.

\myexample{Let us design a method that receives two three-dimensional vectors and returns the distance between the two.} 
We can, effectively, think of this as the distance between two points in a three-dimensional plane. 
Therefore, because each vector contains three components, we need six parameters, where each triplet represents the vectors~$v_1$ and~$v_2$.

\begin{lstlisting}[language=MyJava]
class VectorDistance {

  /**
   * Computes the distance between two given vectors:
   * v_1=(x_1, y_1, z_1) and v_2=(x_2, y_2, z_2)
   */
  static double computeDistance(double x1, double y1, double z1, 
                                double x2, double y2, double z2) {
    return 0.0;
  }
}
\end{lstlisting}

Again, we start by writing the appropriate method signature with its respective parameters and a Java documentation comment explaining its purpose. 
For tests, we know that the distance between two Cartesian points in a three-dimensional plane is 

\[
D(v_1, v_2) = \sqrt{(x_1 - x_2)^2 + (y_1 - y_2)^2 + (z_1 - z_2)^2}
\]

So, let's now write a few test cases with some arbitrarily-chosen points that we can verify with a calculator or manual computation.

\begin{lstlisting}[language=MyJava]
import static Assertions.assertAll;
import static Assertions.assertEquals;

class VectorDistanceTester {

  @Test
  void testComputeDistance() {
    assertAll(
      assertEquals(8.66, computeDistance(3, 2, 1, 8, 7, 6), .01),
      assertEquals(12.20, computeDistance(0, 0, 0, 8, 7, 6), .01),
      assertEquals(8.30, computeDistance(-8, -2, 1, 0, 0, 0), .01));
  }
}
\end{lstlisting}

Notice again our use of the optional delta parameter to allow us a bit of leeway with the rounding of our answer. 
Fortunately, the implementation of our method is just a retelling of the mathematical definition.

\enlargethispage{-2\baselineskip}
\begin{lstlisting}[language=MyJava]
class VectorDistance {

  /**
   * Computes the distance between two given vectors:
   * v_1=(x_1, y_1, z_1) and v_2=(x_2, y_2, z_2).
   * @param x1 x component of first vector.
   * @param y1 y component of first vector.
   * @param z1 z component of first vector.
   * @param x2 x component of second vector.
   * @param y2 y component of second vector.
   * @param z2 z component of second vector.
   * @return distance between v1 and v2.
   */
  static double computeDistance(double x1, double y1, double z1, 
                                double x2, double y2, double z2) {
    return Math.sqrt(Math.pow(x1 - x2, 2) 
                   + Math.pow(y1 - y2, 2)
                   + Math.pow(z1 - z2, 2));
  }
}
\end{lstlisting}

We make prolific use of Java's \ttt{Math} library in designing this method; we use the \ttt{sqrt} method for computing the square root of our result, as well as \ttt{pow} to square each intermediate difference.

\myexample{Slope-intercept is an incredibly common algebra and geometry problem, and even pokes its way into machine learning at times when computing best-fit lines.} 
Let's design two methods, both of which receive two points $(x_1, y_1)$, $(x_2, y_2)$. 
The first method returns the slope~$m$ of the points, and the second returns the y-intercept~$b$ of the line. 
Their respective signatures are straightforward---each set of points is represented by two integer values and return doubles.
\begin{align*}
m &= \dfrac{y_2-y_1}{x_2-x_1}\\
b &= y_1 - mx_1
\end{align*}
\enlargethispage{3\baselineskip}
\begin{lstlisting}[language=MyJava]
class SlopeIntercept {

  /**
   * Computes the slope of the line represented by the two points.
   * @param x1 x-coordinate of point 1.
   * @param y1 y-coordinate of point 1.
   * @param x2 x-coordinate of point 2.
   * @param y2 y-coordinate of point 2.
   * @return slope of points.
   */
  static double slope(int x1, int y1, int x2, int y2) { return 0.0; }

  /**
   * Computes the y-intercept of the line represented by the two points.
   * @param x1 x-coordinate of point 1.
   * @param y1 y-coordinate of point 1.
   * @param x2 x-coordinate of point 2.
   * @param y2 y-coordinate of point 2.
   * @return y-intercept of line represented by points.
   */
  static double yIntercept(x1, int y1, int x2, int y2) { return 0.0; }
}
\end{lstlisting}

The tests that we write are verifiable by a calculator or mental math. 
Our \ttt{yIntercept} method depends on a successful implementation of \ttt{slope}, as designated by the formula of the former. 
In the next chapter, we will consider cases that invalidate the formula, e.g., when two points share an~$x$ coordinate, in which the slope is undefined for those points.

\begin{lstlisting}[language=MyJava]
import static Assertions.assertAll;
import static Assertions.assertEquals;

class SlopeInterceptTester {
  
  @Test
  void testSlope() {
    assertAll(
      () -> assertEquals(1, slope(0, 0, 1, 1)),
      () -> assertEquals(0, slope(0, 0, 1, 0)),
      () -> assertEquals(2, slope(8, 4, 2, 4)),
      () -> assertEquals(0.5, slope(-1, 5, 3, 7)));
  }

  @Test
  void testYIntercept() {
    assertAll(
      () -> assertEquals(0, yIntercept(0, 0, 1, 1)),
      () -> assertEquals(0, yIntercept(0, 0, 1, 0)),
      () -> assertEquals(4, yIntercept(8, 4, 2, 4)),
      () -> assertEquals(5.5, yIntercept(-1, 5, 3, 7)));
  }
}
\end{lstlisting}

And the implementation of those two methods follows directly from the mathematical definitions. 
We replace our temporary \ttt{0.0} return values with the appropriate expressions, and all tests pass.

\begin{lstlisting}[language=MyJava]
class SlopeIntercept {

  /**
   * Computes the slope of the line represented by the two points.
   * @param x1 x-coordinate of point 1.
   * @param y1 y-coordinate of point 1.
   * @param x2 x-coordinate of point 2.
   * @param y2 y-coordinate of point 2.
   * @return slope of points.
   */
  static double slope(int x1, int y1, int x2, int y2) {
    return (y2 - y1) / (x2 - x1);
  }

  /**
   * Computes the y-intercept of the line represented by the two points.
   * @param x1 x-coordinate of point 1.
   * @param y1 y-coordinate of point 1.
   * @param x2 x-coordinate of point 2.
   * @param y2 y-coordinate of point 2.
   * @return y-intercept of points.
   */
  static double yIntercept(x1, int y1, int x2, int y2) {
    return y1 - slope(x1, y1, x2, y2) * x1;
  }
}
\end{lstlisting}

\myexample{We are starting to get used to some of Java's verbosity!} 
Let's now design a method that, when given a (numeric) value of~$x$, evaluates the following quartic formula:

\[
q(x) = 4x^4 + 7x^3 + 21x^2 - 65x + 3
\]

Its signature is straightforward: we receive a value of~$x$, namely a \ttt{double}, and return a \ttt{double}, since our operations work over \ttt{double} values. 
Again, we return zero as a temporary solution to ensure the program successfully compiles.

%\enlargethispage{-1\baselineskip}
\begin{lstlisting}[language=MyJava]
class QuarticFormulaSolver {

  /**
   * Evaluates the equation 4x^4 + 7x^3 + 21x^2 - 65x + 3.
   * @param x the input variable.
   * @return the result of the expression after substituting x.
   */
  static double solveQuartic(double x) { return 0.0; }
}
\end{lstlisting}

Test cases are certainly warranted, but may be a bit tedious to compute by hand, so we recommend using a verified calculator to compute expected solutions!\footnote{Remember that the coefficient is applied \emph{after} applying the exponent to the variable. That is, if $x=3$, then $4x^3$ is equal to $4 \cdot (3)^3$, which resolves to $4 \cdot 27$, which resolves to $108$.}

\begin{lstlisting}[language=MyJava]
import static Assertions.assertAll;
import static Assertions.assertEquals;

class QuarticFormulaSolverTester {

  @Test
  void testSolveQuartic() {
    assertAll(
      () -> assertEquals(3, solveQuartic(0)),
      () -> assertEquals(510, solveQuartic(3)),
      () -> assertEquals(229878, solveQuartic(15)),
      () -> assertEquals(313445.1875, solveQuartic(16.25)));
  }
}
\end{lstlisting}

We write tests \emph{before} the method implementation because we know, intuitively, how to solve an equation for a variable, whereas a computer has to be told how to solve this task. 
Fortunately for us, a quartic equation solver is nothing more than returning the result of the expression. 
We have to use the exponential \ttt{Math.pow} method again (or conjoin several multiplicatives of~$x$), but otherwise, nothing new is utilized.

\enlargethispage{\baselineskip}
\begin{lstlisting}[language=MyJava]
class QuarticFormulaSolver {

  /**
   * Evaluates the equation 4x^4 + 7x^3 + 21x^2 - 65x + 3.
   * @param x the input variable.
   * @return the result of the expression after substituting x.
   */
  static double solveQuartic(double x) {
    return 4 * Math.pow(x, 4) + 7 * Math.pow(x, 3) + 
               21 * Math.pow(x, 2) - 65 * x + 3;
  }
}
\end{lstlisting}

And, as expected, all tests pass. 
With only methods and math operations at our disposal, the capabilities of said methods are quite limited. 
% Let's start revamping our tool set by reintroducing strings.

\subsection{The Main Method}
Java programming tutorials are quick to throw a lot of information at the reader/viewer, and our textbook is no exception to this practice. 
Unfortunately, Java is a verbose programming languages, and to begin designing a method, we must wrap its definition inside a class. 
After this step, we can design the \ttt{static} method implementation. 
Readers no doubt question the significance of \ttt{static}. 
For the first few chapters, we will intentionally omit its definition, as an explanation would almost certainly confuse the reader coming from another language. 
Therefore, for the time being, simply view it as six characters, plus a space, that must be typed in order to write a method that we can then test with JUnit. 

Imagine, however, a situation where we want to output the result of an expression without using a test, as we will do in many examples. 
Java requires a \ttt{main} method in any executable Java class that does not use tests. 
For the sake of completeness, let's now write the traditional ``Hello, world!'' program in Java using the \ttt{main} method instead of tests.

\begin{lstlisting}[language=MyJava]
class MainMethod {

  public static void main(String[] args) {
    System.out.println("Hello, world!");
  }
}
\end{lstlisting}

Yikes, that is a lot of required code to output some text to the console; what does \ttt{public} mean, and what are those \ttt{[]} brackets after the \ttt{String} word?
Once again, we will not detail their importance, but view them as more mandatory characters to type when writing a \ttt{main} method. 
The only word we \emph{will} explain is \ttt{void}, which means that the method does not return a value. 
If the readers are coming from a functional programming language, e.g., Scheme/Racket or OCaml, then it is almost certainly the case that they never worked with methods that did not return a value nor received no arguments.\footnote{A method of no arguments is called a \emph{thunk}, or a \emph{nullary} method.} 
The \ttt{println} method, for example, has no return value; its significance comes from the fact that it outputs text to the terminal/console, which is a \emph{method side-effect}\index{side-effect}. 
We'll come back to what all of this means in subsequent chapters, but we could not avoid at least briefly mentioning it and its existence in the Java language.

% \subsection{Quick Check Questions}
% \begin{enumerate}[label=(\arabic*)]
%   \item \mcq{In Java, what is the term used to describe functions that belong to a class?}{Procedures,Subroutines,Methods,Functions}

%   \item \mcq{Which keyword is used in Java to define a class member that can be accessed without needing an instance of the class?}{\texttt{public},\texttt{private},\texttt{static},\texttt{final}}

%   \item \mcq{What is the return type of the \texttt{fToC} method in the provided example?}{\texttt{int},\texttt{float},\texttt{double},\texttt{String}}

%   \item \mcq{What type of comment is used to document a method in Java?}{Single-line comment (\texttt{//}),Multi-line comment (\texttt{/* /}),JavaDoc comment (\texttt{/* */}),Inline comment}
  
%   \item \mcq{What does the \texttt{@Test} annotation indicate in a Java class?}{The method is the main entry point of the program.,The method is to be executed as a test case.,The method is deprecated.,The method is thread-safe.}

%   \item \mcq{What is the purpose of the \texttt{assertEquals} method in JUnit testing?}{To check if a variable is initialized,To compare two values for equality,To verify if an exception is thrown,To ensure a loop executes a specific number of times}

%   \item \mcq{What is the significance of using \texttt{Math.pow} in the method implementation for computing distance between vectors?}{It converts integers to double.,It performs square root calculation.,It raises a number to the power of another number.,It rounds off decimal values to a set precision.}

%   \item \mcq{What is the primitive datatype for a variable that can hold decimal values in Java?}{\texttt{char},\texttt{int},\texttt{double},\texttt{boolean}}

%   \item \mcq{How are floating-point arithmetic operations handled in Java when involving both integers and floating-point numbers?}{The operation is not allowed and causes a compilation error.,The result is always rounded down to the nearest integer.,The operation must be explicitly cast to a floating-point operation.,The operation automatically promotes to a floating-point operation.}

%   \item \mcq{In the context of the \texttt{main} method in Java, what does \texttt{void} indicate?}{The method returns an integer.,The method returns nothing.,The method must return a string.,The method returns an array of strings.}
% \end{enumerate}
\section{Strings}

\emph{Strings}, as you might recall from another programming language, are sequences of characters. 
Characters are enclosed by single quotes, e.g., \ttt{\q{}x\q{}}. 
A Java \ttt{String} is enclosed by double quotes. e.g., \ttt{"Hello!"}. 
Strings may contain any number of characters and any kind of character, including no characters at all or only a single character. 
There is an apparent distinction between \ttt{\q{}x\q{}} and \ttt{"x"}: the former is a \ttt{char} and the latter is a \ttt{String}. 
Note the capitalization on the word \ttt{String}; the case sensitivity is important, because \ttt{String} is not one of the primitive datatypes that we described in the previous section. 
Rather, it is a sequence of \ttt{char} values coalesced into one value ``under-the-hood,'' so to speak. 
We can declare a \ttt{String} as a variable using the keyword combined with a variable name, just as we do for primitives. 

\begin{lstlisting}[language=MyJava]
class NewStringTests {

  public static void main(String[] args) {
    String s1 = "Hello, world!";
    String s2 = "How are you doing?";
    String s3 = "This is another string!";
  }
}
\end{lstlisting}

\begin{figure}[tp]
  %\begin{wrapfigure}[25]{r}[0.75in]{0.55\textwidth}
    \small
    \begin{tcolorbox}[title=String Class]
      A \emph{string}\index{string} is an immutable sequence of characters. Strings are indexed from~$0$ to~$|S|-1$, where~$|S|$ is the number of characters of~$S$.
      \vspace{2ex}
    \begin{description}
      \item[\ttt{$S_1$ + $S_2$}] adds the characters from~$S_2$ onto the end of~$S_1$, producing a new string.
      \item[\ttt{int $S$.length()}] returns the number of characters in~$S$.
      \item[\ttt{char $S$.charAt($i$)}] retrieves the $(i + 1)^\text{th}$ character in~$S$. We can also say that this retrieves the character at index~$i$ of~$S$.
      \item[\ttt{String $S$.substring($i$, $j$)}] returns a new string containing the characters from index~$i$, inclusive, to index~$j$, exclusive. The number of extracted characters is $j - i$. We will use the notation $S' \sqsubseteq S$ to denote that~$S'$ is a substring of~$S$.
      \item[\ttt{String $S$.substring($i$)}] returns a new string from index~$i$ to the end of~$S$.
      \item[\ttt{int $S$.indexOf($S'$)}] returns the index of the first instance of~$S'$ in~$S$, or~$-1$ if $S' \not\sqsubseteq S$.
      \item[\ttt{boolean $S$.contains($S'$)}] returns \ttt{true} if~$S' \sqsubseteq S$; \ttt{false} otherwise.
      \item[\ttt{String $S$.repeat($n$)}] returns a new string containing~$n$ copies of~$S$.
      \item[\ttt{String String.valueOf($v$)}] returns a stringified version of~$v$, where~$v$ is some primitive value.
      \item[\ttt{int Integer.parseInt($S$)}] returns the integer representation of a string~$S$ if it can be parsed as such.
    \end{description}
  \end{tcolorbox}
    \caption{Useful String Methods.}
    \label{fig:strings}
  \end{figure}

We can conjoin, or \emph{concatenate}\index{concatenation}\index{concatenate}, strings together with the \ttt{+} operator. 
Concatenating one string~$s_2$ onto the end of another string~$s_1$ creates a new string~$s_3$ by copying the characters from~$s_1$ and~$s_2$, in that order. 

Figure~\ref{fig:strings} states that strings are immutable, so how can we possibly combine them without altering one or the other? 
When concatenating two strings~$s_1$ and~$s_2$, Java creates a new string that is the result of the concatenation, leaving~$s_1$ and~$s_2$ unaltered.\footnote{Hence, a series of repeated string concatenations is inefficient. The performance implications of this are addressed in a subsequent chapter.}

To retrieve the number of characters in a string, invoke \ttt{.length} on the string. 
The empty string has length zero, and spaces/whitespace characters count towards the length of a string, since spaces \emph{are} characters. 
For instance, the length of \ttt{"  a  "} is five because there are two spaces, followed by a lowercase \ttt{\q{}a\q{}}, followed by two more spaces.

Comparing strings for equality seems straightforward: we can use \ttt{==} to compare one string versus another. 
Using \ttt{==} for determining string equality is a common Java beginner pitfall! 
Strings are \emph{objects} and cannot be compared for value-equality using the \ttt{==} operator. 
Introducing this term ``value-equality'' insinuates that strings can, in fact, be compared using \ttt{==}, which is theoretically correct. 
The problem is that the result of a comparison using \ttt{==} compares the memory addresses of the strings. 
In other words, \ttt{$s_1$ == $s_2$} returns whether the two strings reference, or point to, the same string in memory. 
 
\begin{lstlisting}[language=MyJava]
class NewStringTests {

  public static void main(String[] args) {
    String s1 = "Hello";
    String s2 = s1;
    System.out.println(s1 == s2); // true
  }
}
\end{lstlisting}

In the above code snippet we declare~$s_1$ as the \emph{string literal}\index{string literal} \ttt{"Hello"}, then initialize~$s_2$ to point to~$s_1$. 
So there are, in effect, two references to the string literal~\ttt{"Hello"}. 
Let's try something a little more tricky: suppose we declare three strings, where~$s_1$ is the same as before,~$s_2$ is the string literal \ttt{"Hello"}, and~$s_3$ is the string literal \ttt{"World"}. 
Comparing~$s_1$ to~$s_2$, strangely enough, also outputs true, but why? 
It seems that~$s_1$ and~$s_2$ reference different string literals, even though they contain the same characters. 
Indeed, the latter is obviously the case, but Java performs an optimization called \emph{string pool caching}\index{string pool cache}. 
That is, if two strings \emph{are} the same string literal, it makes little sense for them to point to two distinct references, entirely because strings are immutable. 
Therefore, Java optimizes these references to point to a single allocated string literal. 
Comparing~$s_1$ or~$s_2$ with~$s_3$ outputs false, which is the anticipated result.

\begin{lstlisting}[language=MyJava]
class NewStringTests {

  public static void main(String[] args) {
    String s1 = "Hello";
    String s2 = "Hello";
    String s3 = "World";
    System.out.println(s1 == s2); // true (?)
    System.out.println(s2 == s3); // false
  }
}
\end{lstlisting}

If we want, for some reason, to circumvent Java's string caching capabilities, we need a way of \emph{instantiating}\index{instantiation} a new string for our variables to reference. 
We use the power of \ttt{new String} to create a brand new, non-cached string reference. 
We treat this as a method, of sorts, called the \emph{object constructor}\index{constructor}. 
In Chapter~\ref{chapter-classes}, we will reintroduce constructors in our discussion on objects and classes, but for now, consider it (namely \ttt{new String}) as a method for creating distinct \ttt{String} instances. 
This method is \emph{overloaded}\index{method overloading} to receive either zero or one parameter. 
The latter implementation receives a \ttt{String}, whose characters are copied into the new \ttt{String} instance. 
If we pass a string literal to the constructor, it copies the characters \emph{from} the literal into the new string. 
At this point, the only possible object to be equal to~$s_1$ is~$s_1$ itself or another string that points to its value.

\enlargethispage{2\baselineskip}
\begin{lstlisting}[language=MyJava]
class NewStringTests {

  public static void main(String[] args) {
    String s1 = new String("hello");
    String s2 = new String("hello");
    String s3 = new String("world");
    String s4 = s1;
    System.out.println(s1 == s1); // true
    System.out.println(s1 == s2); // false!
    System.out.println(s2 == s3); // false
    System.out.println(s1 == s4); // true
  }
}
\end{lstlisting}

\myexample{One final point to make about string equality is that string concatenation is not \emph{always} subject to the same optimization as string literals.} 
Consider the following code snippet, which creates two string literals $s_1$ and $s_2$. Then, we concatenate~$s_1$ and~$s_2$ into a string $s_3$, and separately concatenate the string literals themselves into the string~$s_4$. 
What we observe is that~$s_3$ and~$s_4$ do not point to the same reference, even though they contain the same characters.

\begin{lstlisting}[language=MyJava]
class NewStringTests {

  public static void main(String[] args) {
    String s1 = "Hello";
    String s2 = "World";
    String s3 = s1 + s2;
    String s4 = "Hello" + "World";
    System.out.println(s3 == s4); // false
  }
}
\end{lstlisting}

On the other hand, consider the following code, where we create two strings~$s_1$ and~$s_2$ to both reference the concatenation of the string literals \ttt{"x"} and \ttt{"y"}. 
Comparing~$s_1$ and~$s_2$ for object equality confusingly produces \ttt{true}.
At compile-time, Java optimizes (and evaluates) the string literal concatenations, which leads to it being pooled to reference the same (string) object in memory.

\begin{lstlisting}[language=MyJava]
class NewStringTests {

  public static void main(String[] args) {
    String s1 = "x" + "y";
    String s2 = "x" + "y";
    System.out.println(s1 == s2); // true
  }
}
\end{lstlisting}

\myexample{What if we want to compare strings based on their character content, rather than by memory reference?} 
The \ttt{String} class provides a handy \ttt{equals} method that we invoke on instances of strings. 
Two strings~$s_1$ and~$s_2$ are equal if they are \emph{lexicographically equal}. 
In essence, this is a long and scary word to represent the concept of ``containing the same characters.'' 
Lexicographical comparisons are case-sensitive, meaning that uppercase and lowercase letters are report an unequal comparison.

\begin{lstlisting}[language=MyJava]
class StringLexicographicallyEqual {

  public static void main(String[] args) {
    String s1 = new String("hello");
    String s2 = new String("hello");
    String s3 = new String("world");
    System.out.println(s1.equals(s2)); // true
    System.out.println(s2.equals(s3)); // false
  }
}
\end{lstlisting}

Let's veer into the discussion on the lexicographical ordering of strings. 
Like numbers, strings are comparable and can be, e.g., ``less than'' another. 
According to Java (as well as most other programming languages), one string is less than another if it is lexicographically less than another, and this idea extends to all (string) comparison operations; not just equality. 
Memorizing the string ordering rules is cumbersome, so we propose the S.N.U.L. acronym. 
In general, special characters (S) are less than numbers (N), which are less than uppercase characters (U), which are less than lowercase characters (L).\footnote{We say ``in general'' because certain characters that we might view as special, e.g., \ttt{\q{}@\q{}}, have a larger numeric value than numbers, e.g., \ttt{\q{}2\q{}}.} 
For a full description with the modifications to our generalization, view an ASCII table, which provides the numerical equivalents of all standard American keyboard characters. 

\myexample{Java returns the distance between the first non-equal characters in a string using \ttt{compareTo}.} 
For instance, \ttt{"hello".compareTo("hi")} returns \ttt{-4} because the distance from the first non-equal characters, those being \ttt{\q{}e\q} and \ttt{\q{i}\q}, is minus four characters, since \ttt{\q{}i\q{}} is greater than \ttt{\q{}e\q{}}. 
If we compare \ttt{"hello"} against \ttt{"Hello"}, we get \ttt{32}, because according to the ASCII table, \ttt{\q{}h\q{}} corresponds to the integer~$104$, and \ttt{\q{}H\q{}} corresponds to the integer~$72$; their difference is $104 - 72$. 
Comparing \ttt{"hello"} against \ttt{"hello"} produces zero, indicating that they contain the same characters.

Strings are \emph{indexed from zero}, which means that the characters in the string are located at \emph{indices} from zero to the length of the string minus one. 
So, for example, in the string \ttt{"hello"}, the character \ttt{\q{}h\q{}} is at index zero, \ttt{\q{}e\q{}} is at index one, \ttt{\q{}l\q{}} is at index two, \ttt{\q{}l\q{}} is at index three, and \ttt{\q{}o\q{}} is at index four. 
Knowing this fact is crucial to working with helper methods such as \ttt{charAt}, \ttt{indexOf}, and \ttt{substring}, which we will now discuss.

\myexample{To retrieve the character at a given index, we invoke \ttt{charAt} on the string: \ttt{"hello".charAt(1)} returns \ttt{\q{}e\q{}}.} 
Attempting to index out of bounds with either a negative number or a number that is equal to or exceeds the length of the string results in a \ttt{StringIndexOutOfBoundsException} exception.\footnote{An \emph{exception}\index{exception} in Java is a type of ``error'' that occurs while a program is running. We put ``error'' in quotes because ``errors,'' at least in Java, are unpredictable and are usually a sign of a severe problem in the program such as running out of system memory.}

\myexample{We can use \ttt{indexOf} to find the index of the first occurrence of some value in a string.} 
To demonstrate, \ttt{"hello, how are you?".indexOf("are")} returns \ttt{11} because the substring \ttt{"are"} occurs starting at index~$11$ of the provided string. 
If the supplied argument is not present in the string, \ttt{indexOf} returns~$-1$.

\myexample{A \emph{substring} of a string~$s$ is a sequence of subsequent characters inside~$s$.} 
We can extract a substring~$s'$ from some string~$s$ using the \ttt{substring} method: \ttt{"abcde".substring(2, 4)} returns \ttt{"cd"}. 
Note that the last index is exclusive, meaning that, as a rule of thumb, the number of characters returned in the substring is equal to the second argument minus the first argument. 
There is also a handy second version of this method that receives only one argument: it returns the substring from the given index up to the end of the string. 
For example, \ttt{"abcdefg".substring(1)} returns everything but the first character, namely \ttt{"bcdefg"}.

\myexample{We can convert between datatypes, such as an integer to a string and vice versa, using the \ttt{String.valueOf} and \ttt{Integer.parseInt} methods respectively.} 
Passing any non-string primitive value as an argument to \ttt{valueOf} converts it into its string counterpart.
Oppositely, if we pass a string that represents an integer to \ttt{parseInt}, we obtain the corresponding integer. 
The \ttt{Double.parseDouble} and \ttt{Boolean.parseBoolean} methods behave similarly. 
Passing an invalid string, i.e., one that does not represent the respective datatype, results in Java throwing a \ttt{NumberFormatException} exception.

\begin{verbnobox}[\small]
String s1 = String.valueOf(1234);    // "1234"
String s2 = String.valueOf(Math.PI); // "3.141592653589793"
int n1 = Integer.parseInt(s1);       // 1234
double n2 = Double.parseDouble(s2);  // 3.141593
\end{verbnobox}

% \subsection{Quick Check Questions}
% \begin{enumerate}[label=(\arabic*)]
%   \item \mcq{What is the correct way to declare a string in Java?}{\texttt{char s = 'Hello'},\texttt{String s = 'Hello'},\texttt{String s = "Hello"},\texttt{char[] s = "Hello"}}

%   \item \mcq{How do you concatenate two strings in Java?}{Using the \texttt{conjoin()} method,Using the \texttt{+} operator,Using the \texttt{append()} method,Using the \texttt{++} operator}
%   \item \mcq{How can you compare two strings for content character equality in Java?}{Using the \texttt{==} operator,Using the \texttt{equals()} method,Using the \texttt{compare()} method,Using the \texttt{compareTo()} method}

%   \item \mcq{What does the \texttt{length()} method of a String return?}{The size in bytes of the string,The number of characters in the string,The number of bytes in the string,The capacity of the string buffer}

%   \item \mcq{What is the result of using \texttt{new String("hello") == new String("hello")}?}{\texttt{true}, because the contents of the strings are the same,\texttt{false}, because they are two different objects in memory,\texttt{true}, because of string interning,\texttt{false}, because the \texttt{==} operator checks for value equality}

%   \item \mcq{What does the \texttt{charAt(int index)} method of a String do?}{Returns the character at the specified index, starting from 1,Returns the ASCII value of the character at the specified index,Returns the character at the specified index, starting from 0,Changes the character at the specified index}

%   \item \mcq{How does the \texttt{substring(int beginIndex, int endIndex)} method work in Java?}{It returns a new string that starts from \texttt{beginIndex} and ends at \texttt{endIndex}, inclusive,It returns a new string that starts from \texttt{beginIndex} and ends at \texttt{endIndex}, exclusive,It modifies the original string to contain only the specified range of characters,It throws an exception if \texttt{endIndex} is less than \texttt{beginIndex}}

%   \item \mcq{What is the return type of the \texttt{indexOf(String str)} method when the specified string is not found?}{\texttt{null},\texttt{0},An empty string,\texttt{-1}}

%   \item \mcq{What is the purpose of the \texttt{String.valueOf(int value)} method?}{To convert an integer value to its corresponding ASCII character string,To find the string representation of the integer value in a string array,To convert an integer value to its string representation,To parse an integer value from a string}

%   \item \mcq{What exception is thrown when an invalid string is passed to the \texttt{Integer.parseInt(String s)} method?}{\texttt{IllegalArgumentException},\texttt{InputMismatchException},\texttt{NumberFormatException},\texttt{ParseException}}
% \end{enumerate}
\section{Standard I/O}
Early on in a Java programmer's career, they encounter the issue of reading from the ``console,'' or standard input, as well as the dubiously useful act of debugging by printing data to standard output. Many programmers are aptly familiar with these when coming from other programming languages.

First, we need to discuss the nature of the \emph{standard data streams}. Java (and the operating system in general), utilizes three standard data streams: standard input, standard output, and standard error. We can think of these as sources for reading data from and writing data to. The \emph{standard output stream} is often accessed using the \ttt{System.out} class, then through its various methods, e.g., \ttt{println}, \ttt{print}, and \ttt{printf}. To output a line to standard output, we invoke \ttt{System.out.println} with a string (or some other datatype that is coerced into a string). For relaying messages to the user in a terminal-based application or even when debugging a program, outputting information to standard output is a good idea. On the other hand, sometimes a program fails or the programmer wants to output an error message. It is possible to output error messages to standard output, since they are otherwise indistinguishable. Java has a dedicated \emph{standard error stream} for outputting error messages and logs via \ttt{System.err}. We glossed over this method, but let's discuss \ttt{printf} in more detail due to its inherent power.

The \ttt{printf} method originates from C, but is handy for printing multiple values at once without resorting to unnecessary string concatenation. In addition, it preserves the formatting of the data, which is handy when wanting to treat a \ttt{double} as a floating-point number in a string representation. It receives at least one argument: a format string, and is one of several \emph{variadic methods} that we will discuss. A format string contains special format characters and possibly other text. 

\myexample{To output an \ttt{int} or \ttt{long} using \ttt{printf}, we use the \ttt{\%d} format specifier.}
\begin{verbnobox}[\small]
int x = 42;
System.out.printf("The value of x is %d\n", x);
System.out.printf("We can inline ints 42 or as literals %d\n", 42);
\end{verbnobox}

\myexample{To output a \ttt{double} using \ttt{printf}, we use the \ttt{\%f} format specifier. We can also specify the number of digits \ttt{n} to print after the decimal point by using the format specifier \ttt{.\%nf}. Note that floating a decimal to $n$ digits does not change the value of that variable; rather, it only changes its string/output representation.}

\begin{verbnobox}[\small]
double x = 42.0;
System.out.printf("The value of x is %f\n", x);
System.out.printf("PI to 2 decimals is %.2f\n", Math.PI);
System.out.printf("PI with all decimals is %f\n", Math.PI);
\end{verbnobox}

There are many ways to get creative with \ttt{printf},\index{\ttt{printf}} including space padding, number formatting, left/right-alignment, and more. We will not discuss these in detail, but instead we provide Table~\ref{fig:format-specifiers} of the most common format specifiers\index{format specifiers}.

\begin{figure}[htbp!]
\centering
\begin{tabular}{|c|c|}
\hline
Format Specifier & Description \\
\hline
\hline
\ttt{\%d} & Integer (\ttt{int}/\ttt{long})\\
\hline
\ttt{\%nf} & Floating-point number to $n$ decimals (\ttt{float}/\ttt{double})\\
\hline
\ttt{\%s} & String \\
\hline
\ttt{\%c} & Character (\ttt{char})\\
\hline
\ttt{\%b} & Boolean (\ttt{boolean})\\
\hline
\end{tabular}
\caption{Common Format Specifiers}
\label{fig:format-specifiers}
\end{figure}

The \emph{standard input stream}\index{standard input stream}\index{standard input} allows us to ``read data from the console.'' We place this phrase in quotes because the standard input stream is not necessarily the console/terminal; it simply refers to reading characters from the keyboard that are then stored inside this data stream.

\myexample{Suppose we want to read an integer from the standard input stream. To do so, we first need to instantiate a \ttt{Scanner}, which declares a ``pipe,'' so to speak, from which information is read. It is important to state that, while a \ttt{Scanner} may read from the standard input stream, it can read from other input streams, e.g., files or network connections. We will explore this further in subsequent chapters, but for now, let's declare a \ttt{Scanner} object to read from standard input.}

\begin{verbnobox}[\small]
Scanner in = new Scanner(System.in);
\end{verbnobox}

The \ttt{Scanner} class has handy methods for retrieving data from the stream it is scanning (which we will dub the \emph{scannee}). As we said in the example prompt, to read an integer from the scannee, we use \ttt{nextInt}, which retrieves and removes the next-available integer from the scannee data stream. Note that the \ttt{Scanner} class is line-buffered, meaning that the data will not be processed by the ``retriever methods,'' e.g., \ttt{nextInt}, until there is a new-line character in the input stream. To force a new-line, we press the ``Enter''/``Return'' key.

\begin{verbnobox}[\small]
Scanner in = new Scanner(System.in);
int x = in.nextInt();
System.out.println(x);
\end{verbnobox}

Running the program and typing in any $32$-bit integer feeds it into standard input, then echos it to standard output. Entering any other non-integer value crashes the program with an \ttt{InputMismatchException} exception. So, what if we want to read in a \ttt{String} from the scannee; would we use \ttt{nextString}? Unfortunately, this is not correct. We need to instead use \ttt{nextLine}. The \ttt{nextLine}\index{\ttt{nextLine}} method reads a ``line'' of text, as a string, from the scannee. We define a ``line'' as all characters until the first occurrence of a new-line. Invoking \ttt{nextLine} consumes these characters, including the newline, from the input stream, and stores them into a variable, if requested. It does not, however, store the newline in the variable. 

\begin{verbnobox}[\small]
Scanner in = new Scanner(System.in);
String line = in.nextLine();
System.out.println(line);
\end{verbnobox}

Typing in some characters, which may or may not be numbers, followed by a new-line, stores them in the \ttt{line} variable, excluding said new-line. Though, what happens if we prompt for an integer \emph{then} a string? The program does something quite strange. We type the integer, hit ``Return,'' and the program terminates as if it did not prompt for a string. This is because of how both \ttt{nextInt} and \ttt{nextLine} behave: \ttt{nextInt} consumes all data up to but excluding an integer from the input stream; ignoring leading whitespaces. So, after consuming the integer, a new-line character remains in the input stream buffer. Then, \ttt{nextLine} intends to wait until a newline is in the buffer. Because the input stream buffer presently contains a new-line, it takes everything before the new-line, which comprises the empty string, and consumes both said empty string and the new-line from the buffer. To circumvent this issue, we can insert a call to \ttt{nextLine} in between the calls to \ttt{nextInt} and \ttt{nextLine}, thereby consuming the lone new-line character, clearing the buffer. Notice that we do not put a return value on the left-hand side of this intermediate \ttt{nextLine} invocation; this is because such a variable would hold the empty string, which for the purposes of this program is a worthless (variable) assignment.

\begin{verbnobox}[\small]
Scanner in = new Scanner(System.in);
int x = in.nextInt();
in.nextLine();
String line = in.nextLine();
\end{verbnobox}

\myexample{Let's reimplement Python's \ttt{input} function, which receives a \ttt{String} serving as a prompt for the user to enter data. To make it a bit more user-friendly and elegant, we will add a colon and a space after the given prompt. Because we open a \ttt{Scanner} that reads from the standard input stream, there is no need to worry about, say, calling \ttt{nextInt} prior to invoking \ttt{input}. If, on the other hand, we declared a static global \ttt{Scanner} that reads standard input, and we use that to read an integer \emph{and} inside \ttt{input}, we would be in trouble. In our case, the possible scanners connected to the standard input stream differ, so this (the integer-input problem) never occurs.}

\begin{verbnobox}[\small]
static String input(String prompt) {
  Scanner in = new Scanner(System.in);
  System.out.printf("%s: ", prompt);
  return in.nextLine();
}
\end{verbnobox}

\myexample{Suppose we want to write a method that reads three Cartesian points, as integers, from standard input, and computes the area of the triangle that comprises these points. We can type all integers on the same line, as separated by spaces, because \ttt{nextInt} only parses the \emph{next} integer delimited by spaces. And, as we said before, \ttt{nextInt} skips over existing trailing spaces in the input stream buffer, so those spaces are omitted. From there, we use the formula for computing the area of the triangle from those points.} 
\[
\dfrac{x_1(y_2 - y_3) + x_2(y_3 - y_1) + x_3 (y_1 - y_2)}{2}
\]
%\enlargethispage{-6\baselineskip}
\begin{lstlisting}[language=MyJava]
import java.util.Scanner;

class ThreePointArea {

  /**
   * Computes the area of a triangle given three Cartesian points 
   * via standard input.
   * @return the area of the triangle.
   */
  static double computeThreePointArea() {
    Scanner in = new Scanner(System.in);
    int x1 = in.nextInt();
    int y1 = in.nextInt();
    int x2 = in.nextInt();
    int y2 = in.nextInt();
    int x3 = in.nextInt();
    int y3 = in.nextInt();
    return (x1 * (y2 - y3) + x2 * (y3 - y1) + x3 * (y1 - y2)) / 2.0;
  }
}
\end{lstlisting}
We make a note that reading data from a scanner inside a static method that computes some value is not a very good idea; a better solution would be to read the data inside the \ttt{main} method, then call \ttt{computeThreePointArea} with the six arguments representing each point.

%\enlargethispage{2\baselineskip}
\begin{lstlisting}[language=MyJava]
import java.util.Scanner;

class ThreePointArea {
  
  /**
   * Computes the area of a triangle given three 
   * Cartesian points as parameters.
   * @param x1 - x coordinate of first point.
   * @param y1 - y coordinate of first point.
   * @param x2 - x coordinate of second point.
   * @param y2 - y coordinate of second point.
   * @param x3 - x coordinate of third point.
   * @param y3 - y coordinate of third point.
   * @return the area of the triangle.
   */
  static double computeThreePointArea(double x1, double y1,
                                      double x2, double y2,
                                      double x3, double y3) {
    return (x1 * (y2 - y3) + x2 * (y3 - y1) + x3 * (y1 - y2)) / 2.0;
  }

  public static void main(String[] args) {
    Scanner in = new Scanner(System.in);
    int x1 = in.nextInt();
    int y1 = in.nextInt();
    ...
    System.out.println(computeThreePointArea(x1, y1, x2, y2, x3, y3));
  }
}
\end{lstlisting}

\section{Randomness}
So-called ``true'' randomness is difficult to implement from a computing standpoint. Thus, for most intents and purposes (i.e., all of those described in this textbook), it is sufficient to use \emph{pseudorandomess}\index{pseudorandom}\index{pseudorandomness} to generate random values. A pseudorandom number generator computes seemingly random values using a deterministic algorithm, which means that the output values from the generator are predictable. Although it might be incredibly difficult to predict values from a pseudorandom number generator, it is theoretically possible, making them insufficient and insecure for cryptographic schemata and algorithms. For writing, say, a word-guessing game that picks a word from a list at random, it is perfectly reasonable to use a pseudorandom number generator.

Well, how do we generate pseudorandom numbers in Java? There are a few methods, and many textbooks opt to use \ttt{Math.random},\index{\ttt{Math.random()}} which we will explain, but our examples will largely constitute use of the \ttt{Random}\index{\ttt{Random}} class. Testing methods that rely on randomness is difficult, so our following code snippets do not come with testing suites. 

\myexample{Using \ttt{Random}, let's generate an integer between $0$ and $9$, inclusive on both bounds. To do so, we first need to instantiate a \ttt{Random} object, which we will call \ttt{random}. Then, we should invoke \ttt{nextInt} on the \ttt{random} object with an argument of $10$. Passing the argument $n$ to \ttt{nextInt} returns an integer $x \in [0, n-1]$.\footnote{Readers should note that this version of \ttt{nextInt} is different from te one provided by the \ttt{Scanner} class and cannot be used interchangably.}}

\begin{verbnobox}[\small]
Random random = new Random();
int x = random.nextInt(10); // x in [0, 9]
\end{verbnobox}

\myexample{Imagine we want to generate an integer between $-50$ and $50$, inclusive on both bounds. The idea is to generate an integer between $0$ and $100$, inclusive, then subtract $50$ from the result.}
\begin{verbnobox}[\small]
Random random = new Random();
int x = random.nextInt(101) - 50; // x in [-50, 50]
\end{verbnobox}
\myexample{When creating a \ttt{Random} object, we can pass a \emph{seed} to the constructor, which is an integer that determines the sequence of pseudorandom numbers generated by our \ttt{Random} instance. Therefore if we pass the same seed to two \ttt{Random} objects, they will generate the same sequence of pseudorandom numbers. If we do not pass a seed to the constructor, then the \ttt{Random} object uses the current time as the seed. In theory we could use a predetermined seed to write JUnit tests for methods that rely on randomness.}

%\enlargethispage{6\baselineskip}
\begin{lstlisting}[language=MyJava]
import static Assertions.assertAll;
import static Assertions.assertEquals;

import java.util.Random;

class DualRandomTester {
  
  @Test
  void testDualRand() {
    Random r1 = new Random(212);
    Random r2 = new Random(212);
    for (int i = 0; i < 1_000_000_000; i++) {
      assertEquals(r1.nextInt(1_000_000_000), r2.nextInt(1_000_000_000));
    }
  }
}
\end{lstlisting}

Admittedly, the above test is somewhat useless since it only tests the efficacy of Java's \ttt{Random} class rather than code that we wrote ourselves. Regardless, it is interesting to observe the behavior of two random number generators to see that, in reality, pseudo-randomness is, as we stated, nothing more than slightly advanced math.

\myexample{Java provides the \ttt{random} method from the \ttt{Math} class, which receives no arguments. To do anything significant, we must understand how this method works, i.e., what values it can return. The \ttt{Math.random()} method returns a random \ttt{double} between $[0, 1)$, where the upper-bound is exclusive. So, we could receive results such as \ttt{0.391283114421}, \ttt{0}, \ttt{0.999999999999}, but never exactly one. We can use basic multiplicative offsets to convert this range into what we might want. For example, to generate a random double value between $[0, 10)$, we can multiply the output by ten, e.g., \ttt{Math.random() * 10}.}

\myexample{To generate a random integer between $-5$ and $15$, inclusive, using the \ttt{Math.random} method, we need to do something similar to our \ttt{Random} example. First, we multiply the result of \ttt{Math.random()} by $21$ to generate a floating-point value between $[0, 21)$. \emph{Casting}\index{casting}\index{cast}\index{type cast} (i.e., explicitly treating the returned expression as another type) this expression to an integer gives us an integer between $[0, 20]$. Finally, subtracting five gets us the desired range.}
\begin{verbnobox}[\small]
int x = ((int) (Math.random() * 21)) - 5;
\end{verbnobox}

% \subsection{Quick Check Questions}

% \begin{enumerate}[label=(\arabic*)]
%   \item \mcq{Which class in Java is commonly used for accessing the standard output stream?}{\texttt{System.out},\texttt{System.err},\texttt{OutputStream},\texttt{System.in}}

%   \item \mcq{What is the purpose of the \texttt{System.err} stream in Java?}{To read input from the console,To write output to the console,To handle exceptions in the program,To write error messages and logs}

%   \item \mcq{Which method is used in Java to format and print output to the standard output stream?}{\texttt{print()},\texttt{println()},\texttt{printf()},\texttt{format()}}

%   \item \mcq{How do you read an integer from the standard input stream in Java?}{Using \texttt{System.in.readInt()},Using \texttt{BufferedReader.readInt()},Using \texttt{InputStream.readInt()},Using \texttt{Scanner.nextInt()}}

%   \item \mcq{What is the result of using \texttt{nextInt()} followed by \texttt{nextLine()} on a \texttt{Scanner} object when reading from standard input?}{The \texttt{nextLine()} method reads the remaining part of the line after the integer.,The \texttt{nextLine()} method reads the next line of input.,The \texttt{nextLine()} method is skipped.,The \texttt{nextLine()} method throws an exception.}
  
%   \item \mcq{Which class in Java is used for generating pseudorandom numbers?}{\texttt{Math},\texttt{Random},\texttt{Pseudorandom},\texttt{Generator}}

%   \item \mcq{How do you generate a random integer within a specific range using the \texttt{Random} class in Java?}{By using the \texttt{nextInt()} method without arguments.,By using the \texttt{nextInt(range)} method.,By multiplying \texttt{nextDouble()} with the range.,By using \texttt{nextInt()} and adding a minimum value.}

%   \item \mcq{What is the purpose of providing a seed to a \texttt{Random} object in Java?}{To specify the range of random numbers generated.,To generate only positive random numbers.,To ensure the sequence of random numbers is unpredictable.,To determine the sequence of pseudorandom numbers generated.}

%   \item \mcq{Which method from the \texttt{Math} class is used to generate a random double between 0 (inclusive) and 1 (exclusive)?}{\texttt{randomInt()},\texttt{randomDouble()},\texttt{nextInt()},\texttt{random()}}

%   \item \mcq{How can you generate a random integer between two specific values using \texttt{Math.random()}?}{By adding the minimum value to the result of \texttt{Math.random()},By multiplying \texttt{Math.random()} with the maximum value,By casting the result of \texttt{Math.random()} multiplied by the range to an integer and adding the minimum value,By using the \texttt{Random} class instead of \texttt{Math.random()}}
% \end{enumerate}



\newpage
\section{Exercises}

\myexercise{1}{chapter-testingandjava}{Design the \ttt{double celsiusToFahrenheit(double c)} method, which converts a given temperature from Celsius to Fahrenheit.}

\myexercise{1}{chapter-testingandjava}{Design the \ttt{double fiToCm(double f, double in)} method, which receives two quantities in feet and inches respectively, and returns the combined amount in centimeters.}

\myexercise{1}{chapter-testingandjava}{Design the \ttt{int combineDigits(int a, int b)} method, which receives two \ttt{int} values between 0 and 9, and combines them into a single two-digit number.}

\myexercise{1}{chapter-testingandjava}{Design the \ttt{double gigameterToLightsecond(double gm)} method, which converts a distance in gigameters to light seconds (i.e., the distance that light travels in one second). There are $1,000,000,000$ meters in a gigameter, and light travels $299,792,458$ meters per second.}

\myexercise{1}{chapter-testingandjava}{Design the \ttt{double billTotal(double t)} method, which computes the total for a bill. The total is the given subtotal $t$, plus $6.75\%$ of $t$ for the tax, and $20\%$ of the taxed total for the tip.}

\myexercise{1}{chapter-testingandjava}{Design the \ttt{double grocery(int a, int b, int o, int g, int p)} method, which receives five integers representing the number of apples, bananas, oranges, bunches of grapes, and pineapples purchased at a store. Use the following table to compute the total purchase cost in US dollars.}

\begin{center}
\begin{tabular}{c|c}
    Item & Price Per Item\\
    \hline
    Apple & \$0.59\\
    Banana & \$0.99\\
    Orange & \$0.45\\
    Bunch of Grapes & \$1.39\\
    Pineapple & \$2.24
\end{tabular}
\end{center}

\myexercise{1}{chapter-testingandjava}{Design the \ttt{double pointDistance(double px, double py, double qx, double qy)} method, which receives four double values representing two Cartesian coordinates. The method should return the distance between these points.}

\myexercise{1}{chapter-testingandjava}{Design the \ttt{int sumOfSquares(int x, int y)} method, which computes and returns the sum of the squares of two integers $x$ and $y$.}

\myexercise{1}{chapter-testingandjava}{Design the \ttt{double octagonArea(double s)} method, which computes the area of an octagon with a given side length $s$. The formula is}
\[
A = 2(1 + \sqrt{2})s^2
\]

\myexercise{1}{chapter-testingandjava}{Design the \ttt{double pyramidSurfaceArea(double l, double w, double h)} method, which computes the surface area of a pyramid with a given base length $l$, base width $w$, and height $h$. The formula is}
\[
A = lw + l\sqrt{\left(\dfrac{w}{2}\right)^2 + h^2} + w\sqrt{\left(\dfrac{l}{2}\right)^2 + h^2}
\]

\myexercise{1}{chapter-testingandjava}{Design the \ttt{double crazyMath(double x)} method, which receives a value of $x$ and computes the value of the following expression:}

\[
    (1 - e^{-x})^{xe^{-x}} \cdot \dfrac{x\pi\cdot\cos{(4\pi{x})}}{\log_2{|x|}\cdot{\log_4{|x|} \cdot \ln{|x|}}}
\]

\noindent Below are some test cases. Hint: when testing this method, you may want to use the \ttt{delta} parameter of \ttt{assertEquals}!

\begin{verbnobox}[\small]
crazyMath(0)  => -0.0 
crazyMath(1)  => Infinity
crazyMath(2)  => 17.429741427952166
crazyMath(3)  => 6.778069159471912
crazyMath(10) => 2.4727699557822547
\end{verbnobox}

\myexercise{1}{chapter-testingandjava}{The \emph{z-score} is a measure of how far a given data point is away from the mean of a normally-distributed sample. In essence, roughly $68\%$ of data falls between z-scores of $[-1, 1]$, roughly $95\%$ falls between $[-2, 2]$, and $99.7\%$ falls between $[-3, 3]$. This means that extreme outliers have z-scores of either less than $-3$ or greater than $3$.}
 
Design the \ttt{boolean isExtremeOutlier(double x, double avg, double stddev)} me\-thod that, when given a data point $x$, a mean $\mu$, and a standard deviation $\sigma$, computes the corresponding z-score of $x$ and returns whether it is an ``extreme'' outlier. Use the following formula:

\[
Z = \dfrac{x - \mu}{\sigma}
\]

\myexercise{1}{chapter-testingandjava}{Design the \ttt{double logBase(double n, double b)} that, when given a number $n$ and a base $b$, returns $\log_{b}(n)$. You will need to make use of the change-of-base formula, which we provide below ($n$ is the number to compute the logarithm of, $b$ is the old base, and $b'$ is the new base).}

\[
\log_b(n) = \dfrac{\log_{b'}(n)}{\log_{b'}(b)}
\]

\myexercise{1}{chapter-testingandjava}{Design the \ttt{double pdf(double mean, double stddev, double x)} that calculates the probability density of a given value~$x$ in a normal distribution with mean~$\mu$ and standard deviation~$\sigma$. The formula for the probability density function (PDF) of a normal distribution is as follows:}

\[
f(x) = \dfrac{1}{\sigma\sqrt{2\pi}}e^{-\frac{1}{2}\left(\frac{x-\mu}{\sigma}\right)^2}    
\]

\myexercise{1}{chapter-testingandjava}{Design the \ttt{double lawOfCosines(double a, double b, double th)} method that, when given two side lengths of a triangle~$a, b$ and the angle between those two sides~$\theta$ in degrees, returns the length of the third side~$c$. The formula, which is the law of cosines, is listed below. Hint: \ttt{Math.cos} receives a value in radians; we can convert a value from degrees to radians using \ttt{Math.toRadians}.}

\[
c = \sqrt{a^2+b^2-2ab\cos{\theta}}    
\]

\myexercise{1}{chapter-testingandjava}{Design the \ttt{double angle(double a, double b, double c)} method that, when given three side lengths of a triangle~$a$,~$b$, and~$c$, returns the angle $\theta$ opposite to that of~$c$ in degrees. The formula for this computation is listed below. Hint: arccosine in Java is \ttt{Math.acos}.}

\[
\theta = \cos^{-1}\left(\dfrac{a^2+b^2-c^2}{2ab}\right)  
\]

\myexercise{1}{chapter-testingandjava}{A physics formula for computing object distance displacement is}
\[
d = v_{i}t + 1/2at^2    
\]
where $d$ is the final distance traveled in meters, $v_i$ is the initial velocity, $t$ is the time in seconds, and $a$ is the acceleration in meters per second. Design the \ttt{double distanceTraveled(double vi, double a, double t)} method that, when given these variables as parameters, returns the distance that the object in question traveled.

\myexercise{1}{chapter-testingandjava}{Design the \ttt{String weekday(int d)} method that, when given an integer $d$ from $1$ to $7$, returns the corresponding day of the week, with $1$ corresponding to \ttt{"Monday"} and \ttt{"Sunday"} corresponding to $7$. You \textbf{cannot} use any conditionals or data structures to solve this problem. Hint: declare a string containing each day of the week, with spaces to pad the days, and use \ttt{indexOf} and \ttt{substring}.}

\myexercise{1}{chapter-testingandjava}{Design the \ttt{String flStrip(String s)} method that, when given a string, returns a new string with the first and last characters stripped. You may assume that the input string contains at least two characters.}

%\enlargethispage{2\baselineskip}
\myexercise{1}{chapter-testingandjava}{Design the \ttt{double stats(double x, double y)} method that receives two \ttt{double} parameters, and returns a \ttt{String} containing the following information: the sum, product, difference, the average, the maximum, and the minimum. The string should be formatted as follows, where each category is separated by a newline \ttt{'\textbackslash n'} character. Assume that \ttt{XX} is a placeholder for the calculated result.}
\begin{verbnobox}[\small]
"sum=XX
product=XX
difference=XX
average=XX
max=XX
min=XX"
\end{verbnobox}

\myexercise{1}{chapter-testingandjava}{Design the \ttt{String userId(String f, String l, int y)} method that computes a user ID based on three given values: a first name, a last name, and a birth year. A user ID is calculated by taking the the first five letters of their last name, the first letter of their first name, and the last two digits of their birth year, and combining the result. Your method should, therefore, receive two \ttt{String} parameters and an \ttt{int}. Do not convert the year to a \ttt{String}. Below are some test cases.}

\begin{verbnobox}[\small]
userId("Joshua", "Crotts", 1999)     => "CrottJ99"
userId("Katherine", "Johnson", 1918) => "JohnsK18"
userId("Fred", "Fu", 1957)           => "FuF57"
\end{verbnobox}

\myexercise{1}{chapter-testingandjava}{Design the \ttt{String cutUsername(String email)} method that receives an email address of the form \ttt{x@y.z} and returns the username. The username of an email address is \ttt{x}.}

\myexercise{1}{chapter-testingandjava}{Design the \ttt{String cutDomain(String url)} method that returns the domain name of a website URL of the form \ttt{www.x.Z}, where \ttt{X} is the domain name and \ttt{Z} is the top-level domain.}

\myexercise{1}{chapter-testingandjava}{Design the \ttt{int nextClosest(int m, int n)} method that, when given two positive (non-zero) integers $m$ and $n$, finds the closest positive integer $z$ to $m$ such that $z$ is a multiple of $n$ and $z \leq m$. For example, if $m=67$ and $n=15$, then $z=60$.}
