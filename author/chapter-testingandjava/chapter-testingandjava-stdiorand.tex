\section{Standard I/O}
Early on in a Java programmer's career, they encounter the issue of reading from the ``console,'' or standard input, as well as the dubiously useful act of debugging by printing data to standard output. 
Many programmers are aptly familiar with both of these topics when coming from other programming languages.

First, we need to discuss \emph{standard data streams}\index{standard data streams}. Java (and the operating system in general) utilizes three standard data streams: standard input, standard output, and standard error. 
We can think of these as sources for reading data from and writing data to. 
The \emph{standard output stream}\index{standard output stream} is often accessed using the \ttt{System.out} class and through its various methods, e.g., \ttt{println}, \ttt{print}, and \ttt{printf}. 
To output a line of data to standard output, we invoke \ttt{System.out.println} with a string (or some other datatype that is coerced into a string). 
For relaying messages to the user in a terminal-based application, or even when debugging, outputting information to standard output is a good idea. 
On the other hand, sometimes a program fails or the programmer wants to output an error message. 
It is possible to output error messages to standard output, since they are otherwise indistinguishable. 
Though, Java has a dedicated \emph{standard error stream}\index{standard error stream} for outputting error messages and logs via \ttt{System.err}. 
Now, let's discuss \ttt{printf} in more detail due to its inherent power.

The \ttt{printf} method originates from C, and is handy for printing multiple values at once without resorting to unnecessary string concatenation. 
In addition, it preserves the formatting of the data, which is useful when wanting to treat a \ttt{double} as a floating-point number in a string representation. 
It receives at least one argument: a \emph{format string}\index{format string}, and is one of several \emph{variadic methods} that we will discuss. A format string contains special \emph{format characters}\index{format characters} and possibly other characters. 

\myexample{To output an \ttt{int} or \ttt{long} using \ttt{printf}, we use the \ttt{\%d} format specifier.}
\begin{verbnobox}[\small]
int x = 42;
System.out.printf("The value of x is %d\n", x);
System.out.printf("We can inline ints 42 or as literals %d\n", 42);
\end{verbnobox}

\myexample{To output a \ttt{double} using \ttt{printf}, we use the \ttt{\%f} format specifier.} 
We can also specify the number of digits \ttt{n} to print after the decimal point by using the format specifier \ttt{.\%nf}. 
Note that floating a decimal to~$n$ digits does not change the value of that variable; rather, it only changes its string/output representation.

\begin{verbnobox}[\small]
double x = 42.0;
System.out.printf("The value of x is %f\n", x);
System.out.printf("PI to 2 decimals is %.2f\n", Math.PI);
System.out.printf("PI with all decimals is %f\n", Math.PI);
\end{verbnobox}

There are many ways to get creative with \ttt{printf},\index{\ttt{printf}} including space padding, number formatting, left/right-alignment, and more. 
We will not discuss these in detail, but instead we provide Table~\ref{fig:format-specifiers} of the most common format specifiers\index{format specifiers}.

\begin{figure}[htbp!]
\centering
\begin{tabular}{|c|c|}
\hline
Format Specifier & Description \\
\hline
\hline
\ttt{\%d} & Integer (\ttt{int}/\ttt{long})\\
\hline
\ttt{\%nf} & Floating-point number to $n$ decimals (\ttt{float}/\ttt{double})\\
\hline
\ttt{\%s} & String \\
\hline
\ttt{\%c} & Character (\ttt{char})\\
\hline
\ttt{\%b} & Boolean (\ttt{boolean})\\
\hline
\end{tabular}
\caption{Common Format Specifiers}
\label{fig:format-specifiers}
\end{figure}

The \emph{standard input stream}\index{standard input stream}\index{standard input} allows us to ``read data from the console.'' 
We place this phrase in quotes because the standard input stream is not necessarily the console/terminal; it simply refers to reading characters from the keyboard that are then added into the data stream.

\myexample{Suppose we want to read an integer from the standard input stream.} 
To do so, we first need to instantiate a \ttt{Scanner}\index{\ttt{Scanner}}, which declares a ``pipe,'' so to speak, from which information is read. 
It is important to state that, while a \ttt{Scanner} may read from the standard input stream, it can also read from other input streams, e.g., files or network connections. 
We will explore this further in subsequent chapters, but for now, let's declare a \ttt{Scanner} object to read from standard input.

\begin{verbnobox}[\small]
Scanner in = new Scanner(System.in);
\end{verbnobox}

The \ttt{Scanner} class has handy methods for retrieving data from the stream it is scanning (which we will dub the \emph{scannee}). 
As we said in the example prompt, to read an integer from the scannee, we use \ttt{nextInt}, which retrieves and removes the next-available integer from the scannee data stream. 
Note that the \ttt{Scanner} class, by default, is line-buffered\index{line buffered}, meaning that the data will not be processed by the ``retriever methods,'' e.g., \ttt{nextInt}, until there is a new-line character in the input stream. To force a new-line, we press the ``Enter''/``Return'' key.

\begin{verbnobox}[\small]
Scanner in = new Scanner(System.in);
int x = in.nextInt();
System.out.println(x);
\end{verbnobox}

Running the program and typing in any~$32$-bit integer feeds it into the standard input stream, then echos it to standard output. 
Entering any non-integer value crashes the program with an \ttt{InputMismatchException} exception. 
So, what if we want to read in a \ttt{String} from the scannee; would we use \ttt{nextString}? 
Unfortunately, this is not correct. 
We need to instead use \ttt{nextLine}.
The \ttt{nextLine}\index{\ttt{nextLine}} method reads a ``line'' of text, as a string, from the scannee. 
We define a ``line'' as all characters until the first occurrence of a line break. 
Invoking \ttt{nextLine} consumes these characters, including the newline, from the input stream, and stores them into a variable, if requested. 
It does not, however, store the new-line character in the returned string. 

\begin{verbnobox}[\small]
Scanner in = new Scanner(System.in);
String line = in.nextLine();
System.out.println(line);
\end{verbnobox}

Typing in some characters, which may or may not be numbers, followed by a new-line, stores them in the \ttt{line} variable, excluding said new-line. 
Though, what happens if we prompt for an integer, \emph{then} a string? 
Our program behaves quite strangely. 
We type the integer, hit ``Return,'' and the program terminates as if it did not prompt for a string. 
This is because of how both \ttt{nextInt} and \ttt{nextLine} work: \ttt{nextInt} consumes all data up to but excluding an integer from the input stream; ignoring any preceding whitespace characters. 
So, after consuming the integer, a new-line character remains in the input stream buffer. 
Then, \ttt{nextLine} intends to wait until a newline is in the buffer. 
Because the input stream buffer presently contains a new-line, it consumes everything before the new-line, which comprises the empty string, meaning that both the empty string and the new-line are removed the buffer. 
To circumvent this problem, we insert a call to \ttt{nextLine} in between the calls to \ttt{nextInt} and \ttt{nextLine}, thereby consuming the lone new-line character and, in effect, clearing the buffer. 
Notice that we do not put a return value on the left-hand side of this intermediate \ttt{nextLine} invocation; this is because such a variable would hold the empty string, which for the purposes of this program is a meaningless (variable) assignment.

\begin{verbnobox}[\small]
Scanner in = new Scanner(System.in);
int x = in.nextInt();
in.nextLine();
String line = in.nextLine();
\end{verbnobox}

\myexample{Let's reimplement Python's \ttt{input} function, which receives a \ttt{String} serving as a prompt for the user to enter data.} 
To make it a bit more user-friendly and elegant, we will add a colon and a space after the given prompt. 
Because we open a \ttt{Scanner} to read from the standard input stream, there is no need to worry about, say, calling \ttt{nextInt} prior to invoking \ttt{input}.
If, on the other hand, we declared a static global \ttt{Scanner} that reads standard input, and we use that to read an integer \emph{inside} \ttt{input}, we would be in trouble. 

\begin{verbnobox}[\small]
static String input(String prompt) {
  Scanner in = new Scanner(System.in);
  System.out.printf("%s: ", prompt);
  return in.nextLine();
}
\end{verbnobox}

\myexample{Suppose we want to design a method that reads three Cartesian points, as integers, from standard input, and computes the area of the triangle that comprises these points.} 
We can type all integers on the same line, as separated by spaces, because \ttt{nextInt} only parses the \emph{next} integer in the input stream, delimited by whitespace. 
And, as we said before, \ttt{nextInt} skips over existing trailing spaces in the input stream buffer, so those spaces are omitted. 
From there, we use the formula for computing the area of the triangle from those points.
\[
\dfrac{x_1(y_2 - y_3) + x_2(y_3 - y_1) + x_3 (y_1 - y_2)}{2}
\]
\enlargethispage{2\baselineskip}
\begin{lstlisting}[language=MyJava]
import java.util.Scanner;

class ThreePointArea {

  /**
   * Computes the area of a triangle given three Cartesian points via standard input.
   * @return the area of the triangle.
   */
  static double computeThreePointArea() {
    Scanner in = new Scanner(System.in);
    int x1 = in.nextInt(); int y1 = in.nextInt();
    int x2 = in.nextInt(); int y2 = in.nextInt();
    int x3 = in.nextInt(); int y3 = in.nextInt();
    return (x1 * (y2 - y3) + x2 * (y3 - y1) + x3 * (y1 - y2)) / 2.0;
  }
}
\end{lstlisting}

We make a note that reading data from a scanner inside a static method to compute some value is not a very good idea. 
A better alternative solution is to read the data inside a different, unrelated method, e.g., \ttt{main}, then call \ttt{computeThreePointArea} with the six arguments representing each point.

%\enlargethispage{2\baselineskip}
\begin{lstlisting}[language=MyJava]
import java.util.Scanner;

class ThreePointArea {
  
  /**
   * Computes the area of a triangle given three 
   * Cartesian points as parameters.
   * @param x1 x coordinate of first point.
   * @param y1 y coordinate of first point.
   * @param x2 x coordinate of second point.
   * @param y2 y coordinate of second point.
   * @param x3 x coordinate of third point.
   * @param y3 y coordinate of third point.
   * @return the area of the triangle.
   */
  static double computeThreePointArea(double x1, double y1,
                                      double x2, double y2,
                                      double x3, double y3) {
    return (x1 * (y2 - y3) + x2 * (y3 - y1) + x3 * (y1 - y2)) / 2.0;
  }

  public static void main(String[] args) {
    Scanner in = new Scanner(System.in);
    int x1 = in.nextInt();
    int y1 = in.nextInt();
    ...
    System.out.println(computeThreePointArea(x1, y1, x2, y2, x3, y3));
  }
}
\end{lstlisting}

\section{Randomness}
So-called ``true'' randomness is difficult to implement from a computability standpoint. 
Thus, for most intents and purposes (i.e., all of those described in this textbook), it is sufficient to use \emph{pseudorandomess}\index{pseudorandom}\index{pseudorandomness} when generating random values. 
A \emph{pseudorandom number generator} (PRNG) computes apparently random values using a deterministic algorithm, which means that the output values from the generator are predictable. 
Although it might be incredibly difficult to predict pseudorandomly-generated numbers, it is theoretically possible, which makes those numbers insufficient and insecure for cryptographic schemata and algorithms. 
For designing, perhaps, a word-guessing game that randomly chooses words from a list, it is perfectly reasonable to use a pseudorandom number generator.

Well, how do we generate pseudorandom numbers in Java? 
There are a few approaches, and many textbooks opt to use \ttt{Math.random},\index{\ttt{Math.random()}} which we will explain, but our examples will largely constitute the use of the \ttt{Random}\index{\ttt{Random}} class. 
Testing methods that rely on randomness is difficult, so our following code snippets do not come with testing suites. 

\newpage %ugh
\myexample{Using \ttt{Random}, let's generate an integer between~$0$ and~$9$, inclusive on both bounds.} 
To do so, we first need to instantiate a \ttt{Random} object, which we will call \ttt{random}. 
Then, we should invoke \ttt{nextInt} on the \ttt{random} object with an argument of~$10$. Passing the argument~$n$ to \ttt{nextInt} returns an integer $x \in [0, n-1]$.\footnote{Readers should be aware that this version of \ttt{nextInt} is different from the one provided by the \ttt{Scanner} class and cannot be used interchangeably.}

\begin{verbnobox}[\small]
Random random = new Random();
int x = random.nextInt(10); // x in [0, 9]
\end{verbnobox}

\myexample{Imagine we want to generate an integer between~$-50$ and~$50$, inclusive on both bounds.} 
The idea is to generate an integer between~$0$ and~$100$, inclusive, then subtract~$50$ from the result.

\begin{verbnobox}[\small]
Random random = new Random();
int x = random.nextInt(101) - 50; // x in [-50, 50]
\end{verbnobox}

\myexample{When creating a \ttt{Random} object, we can pass a \emph{seed} to the constructor, which is an integer that determines the sequence of pseudorandom numbers generated by our \ttt{Random} instance.} 
Therefore, if we pass the same seed to two distinct \ttt{Random} objects, they will generate the same sequence of pseudorandom numbers. 
If we do not pass a seed to the constructor, then the \ttt{Random} object will use the current system time as the seed. 
In theory, we could select a predetermined seed to write JUnit tests for methods that rely on randomness.

%\enlargethispage{6\baselineskip}
\begin{lstlisting}[language=MyJava]
import static Assertions.assertAll;
import static Assertions.assertEquals;

import java.util.Random;

class DualRandomTester {
  
  @Test
  void testDualRand() {
    Random r1 = new Random(212);
    Random r2 = new Random(212);
    for (int i = 0; i < 1_000_000_000; i++) {
      assertEquals(r1.nextInt(1_000_000_000), r2.nextInt(1_000_000_000));
    }
  }
}
\end{lstlisting}

Admittedly, the above test is somewhat useless since it only tests the efficacy of Java's \ttt{Random} class, rather than code that we wrote ourselves. 
Regardless, it is interesting to observe the behavior of two random number generators to see that, in reality, pseudo-randomness is, as we stated, nothing more than slightly advanced math.

\myexample{Java's \ttt{Math} class provides a \ttt{random} method, which receives no arguments. 
To do anything significant, we must understand how this method works and what kinds of values it can return. 
The \ttt{Math.random()} method returns a random \ttt{double} between $[0, 1)$, where the upper-bound is exclusive. 
So, we could see numbers such as \ttt{0.391283114421}, \ttt{0}, \ttt{0.999999999999}, but never exactly one. 
We can use basic multiplicative offsets to convert this range into one that we might want. 
For example, to generate a random double value between $[0, 10)$, we multiply the output by ten, e.g., \ttt{Math.random() * 10}.}

\myexample{To generate a random integer between~$-5$ and~$15$, inclusive, using the \ttt{Math.random} method, we need to create an offset similar to what we did in the \ttt{Random} example.} 
First, we multiply the result of \ttt{Math.random()} by~$21$ to generate a floating-point value between $[0, 21)$. \emph{Casting}\index{casting}\index{cast}\index{type cast} (i.e., explicitly treating the returned expression as another type) this expression to an integer produces an integer between $[0, 20]$. Finally, subtracting five therefrom gets us the desired range.

\begin{verbnobox}[\small]
int x = ((int) (Math.random() * 21)) - 5;
\end{verbnobox}

% \subsection{Quick Check Questions}

% \begin{enumerate}[label=(\arabic*)]
%   \item \mcq{Which class in Java is commonly used for accessing the standard output stream?}{\texttt{System.out},\texttt{System.err},\texttt{OutputStream},\texttt{System.in}}

%   \item \mcq{What is the purpose of the \texttt{System.err} stream in Java?}{To read input from the console,To write output to the console,To handle exceptions in the program,To write error messages and logs}

%   \item \mcq{Which method is used in Java to format and print output to the standard output stream?}{\texttt{print()},\texttt{println()},\texttt{printf()},\texttt{format()}}

%   \item \mcq{How do you read an integer from the standard input stream in Java?}{Using \texttt{System.in.readInt()},Using \texttt{BufferedReader.readInt()},Using \texttt{InputStream.readInt()},Using \texttt{Scanner.nextInt()}}

%   \item \mcq{What is the result of using \texttt{nextInt()} followed by \texttt{nextLine()} on a \texttt{Scanner} object when reading from standard input?}{The \texttt{nextLine()} method reads the remaining part of the line after the integer.,The \texttt{nextLine()} method reads the next line of input.,The \texttt{nextLine()} method is skipped.,The \texttt{nextLine()} method throws an exception.}
  
%   \item \mcq{Which class in Java is used for generating pseudorandom numbers?}{\texttt{Math},\texttt{Random},\texttt{Pseudorandom},\texttt{Generator}}

%   \item \mcq{How do you generate a random integer within a specific range using the \texttt{Random} class in Java?}{By using the \texttt{nextInt()} method without arguments.,By using the \texttt{nextInt(range)} method.,By multiplying \texttt{nextDouble()} with the range.,By using \texttt{nextInt()} and adding a minimum value.}

%   \item \mcq{What is the purpose of providing a seed to a \texttt{Random} object in Java?}{To specify the range of random numbers generated.,To generate only positive random numbers.,To ensure the sequence of random numbers is unpredictable.,To determine the sequence of pseudorandom numbers generated.}

%   \item \mcq{Which method from the \texttt{Math} class is used to generate a random double between 0 (inclusive) and 1 (exclusive)?}{\texttt{randomInt()},\texttt{randomDouble()},\texttt{nextInt()},\texttt{random()}}

%   \item \mcq{How can you generate a random integer between two specific values using \texttt{Math.random()}?}{By adding the minimum value to the result of \texttt{Math.random()},By multiplying \texttt{Math.random()} with the maximum value,By casting the result of \texttt{Math.random()} multiplied by the range to an integer and adding the minimum value,By using the \texttt{Random} class instead of \texttt{Math.random()}}
% \end{enumerate}


