\section{Interfaces}
\label{chapter-classes-interfaces}

Interfaces are a way of grouping classes together by a ubiquitous behavior. 
We have worked with interfaces before without acknowledging their properties as an interface.
For example, the \ttt{Comparable} interface is implemented by classes that we want to be able to inhibit ``comparable'' behavior. 
In particular, there is a single method that must be implemented by any class that implements the \ttt{Comparable} interface: \ttt{compareTo}. 
The \ttt{compareTo} method receives a single parameter of the same type as the class that implements the \ttt{Comparable} interface, and returns an integer. 
Said integer is negative if \ttt{this} object instance is less than the passed argument, zero if \ttt{this} object instance is equal to the passed argument, and positive if \ttt{this} object instance is greater than the passed argument.

So, by having a class implement the \ttt{Comparable} interface, we group it into that subset of classes that are, indeed, comparable. 
Doing so implies that these classes have an ordering and are sortable in, for example, a Java collection. 

In addition to the \ttt{Comparable} interface, we have worked with the \ttt{List}, \ttt{Queue}, and \ttt{Map} interfaces, which all have a set of methods that must be implemented by any class that implements the interface. 
Recall that \ttt{ArrayList} and \ttt{LinkedList} are ``kinds of'' \ttt{List} objects, and this interface describes several methods that all lists, by definition, must override.\footnote{Here we clarify that ``kind of,'' in this context, means to implement the \ttt{List} interface.} 
To \emph{override} a method means that we provide a new implementation of the method that is different from the default implementation provided by the interface.

\subsection*{Defining an Interface}

\myexample{Imagine that we want to design an interface that describes a shape.} 
All (two-dimensional) shapes have an area and a perimeter, so we can define an interface that, when implemented by a class, requires that the class provide an implementation of the \ttt{area} and \ttt{perimeter} methods. 
A common convention for user-defined interfaces is to prefix their names with \ttt{I} to distinguish them from classes. 
Moreover, the names of interfaces are either adjectives or, more traditionally, verbs, since they describe behaviors or characteristics of a class.\footnote{We do not add the \ttt{public} keyword to the interface definition nor any methods within because all interface methods are implicitly public.}

\begin{lstlisting}[language=MyJava]
interface IShape {

  double area();

  double perimeter();
}
\end{lstlisting}

We cannot write any tests for the \ttt{IShape} interface directly, because it is impossible to instantiate an interface. 
As defined, interfaces are a way of grouping classes by behavior. 
It, therefore, does not make sense to be able to instantiate an interface, because that would suggest that the interface in and of itself is an object.
We can, however, write two different classes that implement \ttt{IShape}, and test those classes. 
To demonstrate this concept, we will design and test the \ttt{Pentagon} and \ttt{Octagon} classes whose constructors receive (and then store as instance variables) the side length of the shape. 
Fortunately, the definitions thereof are trivial because they are nothing more than regurgitations of the mathematical formulae. 
Notice that, when testing, we initialize the object instance to be of type \ttt{IShape} instead of \ttt{Pentagon} or \ttt{Octagon}. 
We want to be able to categorize these classes as types of \ttt{IShape} instances rather than solely instances of \ttt{Pentagon} or \ttt{Octagon} respectively. 
Instantiating a variable as an interface type, then instantiating it as a subtype is a form of \emph{polymorphism}. 
Polymorphism is the ability of an object to take on many forms. 
In this case, the \ttt{IShape} interface is the form that the \ttt{Pentagon} and \ttt{Octagon} classes use to take on the form of a shape as we described. 

When implementing the methods of an interface in a class, we must mark those methods as \ttt{public} because all interface methods are \ttt{public}, either explicitly or implicitly. 
In this context, the \ttt{area} and \ttt{perimeter} methods are overridden in the \ttt{Octagon} and \ttt{Pentagon} classes.

%\enlargethispage{-2\baselineskip}
\begin{lstlisting}[language=MyJava]
import static Assertions.assertAll;
import static Assertions.assertEquals;

class IShapeTester {

  private static final DELTA = 0.01;
  
  @Test
  void testPentagon() {
    IShape p1 = new Pentagon(1);
    IShape p2 = new Pentagon(7.25);
    assertAll(
      () -> assertEquals(1.72, p1.area(), DELTA),
      () -> assertEquals(90.43, p2.area(), DELTA),
      () -> assertEquals(5, p1.perimeter(), DELTA),
      () -> assertEquals(36.25, p2.perimeter(), DELTA));
  }

  @Test
  void testOctagon() {
    IShape o1 = new Octagon(1);
    IShape o2 = new Octagon(7.25);
    assertAll(
      () -> assertEquals(4.83, o1.area(), DELTA),
      () -> assertEquals(253.79, o2.area(), DELTA),
      () -> assertEquals(8, o1.perimeter(), DELTA),
      () -> assertEquals(58, o2.perimeter(), DELTA));
  }
}
\end{lstlisting}

\begin{lstlisting}[language=MyJava]
class Pentagon implements IShape {
  
  private final double SIDE_LENGTH;

  Pentagon(double sideLength) { 
    this.SIDE_LENGTH = sideLength; 
  }

  @Override
  public double area() {
    return 0.25 * Math.sqrt(5 * (5 + 2 * Math.sqrt(5))) 
                * Math.pow(this.SIDE_LENGTH, 2);
  }

  @Override
  public double perimeter() {
    return 5 * this.SIDE_LENGTH;
  }
}
\end{lstlisting}

%\enlargethispage{-2\baselineskip}
\begin{lstlisting}[language=MyJava]
class Octagon implements IShape {

  private final double SIDE_LENGTH;

  Octagon(double sideLength) { 
    this.SIDE_LENGTH = sideLength; 
  }

  @Override
  public double area() {
    return 2 * (1 + Math.sqrt(2)) * Math.pow(this.SIDE_LENGTH, 2);
  }

  @Override
  public double perimeter() {
    return 8 * this.SIDE_LENGTH;
  }
}
\end{lstlisting}

\myexample{Recall from the previous chapter our ``Twenty-one'' card game example.} 
In that small project, we designed the \ttt{Suit} class, which encapsulated four static instances of \ttt{Suit}, where each represented one of the four valid card suits. 
Even though this design works as intended, it fails to be elegant and demonstrate how the suits are all the same, but differ only in their string representation. 
Let's now design the \ttt{ISuit} interface, thereby requiring any implementing class to override the \ttt{stringify} method.

\begin{lstlisting}[language=MyJava]
interface ISuit {

  /**
   * Returns the string representation of the suit.
   */
  String stringify();
}
\end{lstlisting}

From here, we define four separate classes, all of which implement \ttt{ISuit} and override the \ttt{stringify} method. 
These classes are incredibly simple, and as such, we show only the \ttt{Diamond} and \ttt{Heart} classes.

\begin{lstlisting}[language=MyJava]
class Diamond implements ISuit {
  
  Diamond() {}

  @Override
  public String stringify() { return "(*;$\diamondsuit$;*)"; }
}
\end{lstlisting}

%\enlargethispage{\baselineskip}
\begin{lstlisting}[language=MyJava]
class Heart implements ISuit {
    
  Heart() {}

  @Override
  public String stringify() { return "(*;$\heartsuit$;*)"; }
}
\end{lstlisting}

As shown, both \ttt{Diamond} and \ttt{Heart} implement \ttt{ISuit} and handle ``stringification'' differently. We can test these definitions by storing a list of \ttt{ISuit} instances and ensuring that the correct character is returned.

\begin{lstlisting}[language=MyJava]
import static Assertions.assertAll;
import static Assertions.assertEquals;

import java.util.List;
import java.util.ArrayList;

class ISuitTester {

  @Test
  void suitTester() {
    List<ISuit> suit = new ArrayList<>();
    // Add diamonds at even indices, hearts at odd indices.
    for (int i = 0; i < 100; i++) {
      if (i % 2 == 0) { suit.add(new Diamond()); }
      else { suit.add(new Heart()); }
    }

    // Now check to verify that the stringification works.
    for (int i = 0; i < suit.size(); i++) {
      if (i % 2 == 0) { assertEquals("(*;$\diamondsuit$;*)", suit.get(i).stringify()); }
      else { assertEquals("(*;$\heartsuit$;*)", suit.get(i).stringify()); }
    }
  }
}
\end{lstlisting}

One extra piece of information that we should share is that we can instantiate objects in different ways. 
To demonstrate why this is significant, suppose we initialize an object~$s_1$ to be of type \ttt{ISuit}, but instantiate it as type \ttt{Diamond}. 
Then, we initialize another object~$s_2$ to be of type \ttt{Diamond} and instantiate it as type \ttt{Diamond}. 
We would expect that~$s_1$ and~$s_2$ are equivalent, but this is not the case. 
Suppose \ttt{Diamond} contains a method \ttt{diamondCount} that does something irrelevant, but belongs solely to the \ttt{Diamond} class. 
Because~$s_1$ is of type \ttt{ISuit}, we cannot invoke the \ttt{diamondCount} method, since \ttt{ISuit} knows nothing about said method. 
On the contrary, $s_2$ can certainly invoke \ttt{diamondCount}, but it is not polymorphic, since it is not of type \ttt{ISuit}. 
Should we want to invoke \ttt{diamondCount} on the ~$s_1$ object, we need to \emph{downcast}\index{downcast} $s_1$ to type \ttt{Diamond}.

\begin{verbnobox}[\small]
ISuit s1 = new Diamond();
s1.diamondCount();             // Compile-time error!
Diamond s2 = new Diamond();
s2.diamondCount();             // Works but not polymorphic.
((Diamond) s1).diamondCount(); // Works but downcasts.
\end{verbnobox}

\myexample{Animals are a common example of an interface.} 
Imagine that, in our domain of animals, every animal can speak one way or another. 
Speaking involves returning a string representing the sound that the animal makes. 
By designing the \ttt{IAnimal} interface, we can group all animals that have the capability of ``speaking'' together. 
We can follow this by designing classes to implement the \ttt{IAnimal} interface, which provide an implementation of the \ttt{speak} method. 
When testing these classes, we can instantiate a collection of \ttt{IAnimal} instances, and invoke \ttt{speak} on each of them polymorphically. 
In doing so, we debut a refresher of the stream API.

\begin{lstlisting}[language=MyJava]
interface IAnimal {

  /**
   * Returns the sound that the animal makes.
   */
  String speak();
}
\end{lstlisting}

\begin{lstlisting}[language=MyJava]
import static Assertions.assertAll;
import static Assertions.assertEquals;

import java.util.List;
import java.util.ArrayList;

class IAnimalTester {
  
  @Test
  void testCat() {
    IAnimal cat = new Cat();
    assertEquals("Meow!", cat.speak());
  }

  @Test
  void testDog() {
    IAnimal dog = new Dog();
    assertEquals("Woof!", dog.speak());
  }

  @Test
  void testListOfAnimals() {
    List<IAnimal> animals = new ArrayList<>();
    animals.add(new Cat());
    animals.add(new Dog());
    animals.add(new Cat());
    assertEquals("[Meow!, Wolf! Meow!]", 
                 animals.stream()
                        .map(IAnimal::speak)
                        .collect(Collectors.toList()));
  }
}
\end{lstlisting}

\begin{lstlisting}[language=MyJava]
class Cat implements IAnimal {

  @Override
  public String speak() { return "Meow!"; }
}
\end{lstlisting}

%\enlargethispage{3\baselineskip}
\begin{lstlisting}[language=MyJava]
class Dog implements IAnimal {

  @Override
  public String speak() { return "Woof!"; }
}
\end{lstlisting}

\myexample{Suppose we want to design an interface that ``boxes'' an arbitrary value.} 
We have seen this idea through Java's autoboxing and autounboxing mechanisms of the primitive datatypes via the wrapper classes. 
Our interface, however, extends the concept to any type. 
We can define an interface that requires any class to implement it provide an implementation of the \ttt{box}, \ttt{get}, and \ttt{set} methods. 
Boxing a value means that we can pass it around as a \emph{reference} rather than as a raw value. 
Recall that, in Java, we pass primitives to methods by value and, therefore, any changes to the argument are not preserved outside the method body. 
If, however, we box the primitive, the box is passed by reference, and it is then possible to manipulate the contents of the box. 
We will first design the generic \ttt{IBox} interface, then we will design a class that implements the methods specified by the interface.

Interestingly, interfaces can contain static methods. 
Our \ttt{IBox} interface has the static \ttt{box} method, which returns a box that encapsulates the provided value. 
The \ttt{box} method can be used without having to instantiate a class that implements the \ttt{IBox} interface. 
From there, we can write the \ttt{get} and \ttt{set} methods to retrieve and change the value of the box.

\begin{lstlisting}[language=MyJava]
interface IBox<T> {
  
  /**
   * Boxes the value of type T.
   */
  static IBox<T> box(T t);

  /**
   * Returns the boxed value of type T.
   */
  T get();

  /**
   * Sets the boxed value of type T.
   */
  void set(T t);
}
\end{lstlisting}

%\enlargethispage{1\baselineskip}
\begin{lstlisting}[language=MyJava]
import static Assertions.assertAll;
import static Assertions.assertEquals;

class IBoxTester {

  private static <T> void modifyBox(IBox<T> box, T t) {
    box.set(t);
  }
  
  @Test
  void testIntegerBox() {
    IBox<Integer> box = IntegerBox.box(5);
    assertAll(
      () -> assertEquals(5, box.get()),
      () -> modifyBox(box, 10),
      () -> assertEquals(10, box.get()));
  }
}
\end{lstlisting}

\begin{lstlisting}[language=MyJava]
class IntegerBox implements IBox<Integer> {
  
  private Integer value;
  
  private IntegerBox(Integer value) { 
    this.value = value; 
  }
  
  @Override
  public static IBox<Integer> box(Integer value) { 
    return new IntegerBox(value); 
  }
  
  @Override
  public Integer get() { return this.value; }
  
  @Override
  public void set(Integer value) { this.value = value; }
}
\end{lstlisting}

\myexample{The Java Swing API is a graphics framework for designing graphical interfaces and drawing shapes/images.} 
In addition to these capabilities, it also supports user input through the keyboard, mouse, and other means. 
Compared to a class such as \ttt{Scanner}, which waits for the user to press ``Enter'' when they are finished typing input, the Swing API allows for dynamic input and whose events are processed as they occur.
We call the part of the program that listens and processes events an \emph{event listener}. 
A popular example in Java is the \ttt{ActionListener} interface, which is used to listen for a broad classification of events, ranging from button clicks to menu selections. 
When an event occurs, the \ttt{ActionListener} interface is notified and can then respond to the event however the programmer desires. 
The \ttt{ActionListener} interface has a single method, \ttt{actionPerformed}, that is invoked when an event occurs. 
The \ttt{actionPerformed} method receives an \ttt{ActionEvent} object that contains information about the event that occurred, which is then usable by the method to determine what to do in response to the event. 
Because graphical interface design goes beyond the scope of this textbook, we will omit a code example, but we mention action listeners to demonstrate that interfaces are not limited to the examples we have seen thus far. 
Moreover, the Swing API provides more specific listeners for processing keyboard and mouse events, e.g., \ttt{KeyListener}, \ttt{MouseListener}, \ttt{MouseMotionListener}, and so forth. 
We could, for instance, design a class that implements the \ttt{MouseListener} interface and provides an overriding implementation of the \ttt{mouseClicked} method. 
Then, inside a Java Swing graphical component, we might hook the class as a mouse listener and, when the user clicks the mouse, the \ttt{mouseClicked} method is invoked.

\myexample{An amazing insight into the power of interfaces is already present in Java, but deriving it ourselves is useful.} 
Consider the notion of first-class methods: the concept in which methods and data are equivalent, wherein both can be passed to and returned from methods. In Java, we can pass methods around as arguments, mimicing first-class methods, by designing a \emph{functional interface}.

Let's design the generic \ttt{Function<T, V>} interface, which quantifies over two types \ttt{T}, representing the input type, and \ttt{V}, representing the output type. 
The \ttt{Function<T, V>} interface has a single static method, \ttt{apply}, which receives an argument of type \ttt{T} and returns a value of type \ttt{V}. 
We can then design a class that implements the \ttt{Function<T, V>} interface and provides an implementation of the \ttt{apply} method. 
Then, by passing the class around as an argument to other methods, we can invoke the \ttt{apply} method on the class to get the method result.
An incredibly simple example is \ttt{AddOne}, which implements the \ttt{Function<Integer, Integer>} interface and adds one to its input. 
We mark the constructor of the implementing class as private to prevent any unnecessary instantiations; the class itself should only ever be utilized as a first-class citizen rather than an object.

\begin{lstlisting}[language=MyJava]
interface Function<T, V> {

  static V apply(T t); 
}
\end{lstlisting}

\begin{lstlisting}[language=MyJava]
import static Assertions.assertAll;
import static Assertions.assertEquals;

class AddOneTester {

  @Test
  void addOneTester() {
    assertAll(
      () -> assertEquals(0, AddOne.apply(1)),
      () -> assertEquals(3, AddOne.apply(2)),
      () -> assertEquals(30001, AddOne.apply(30000)));
  }
}
\end{lstlisting}
\begin{lstlisting}[language=MyJava]
class AddOne implements Function<Integer, Integer> {

  private AddOne() {}

  @Override
  public static Integer apply(Integer i) { return i + 1; }
}
\end{lstlisting}

So far, we have not demonstrated the potential of first-class methods in Java with our design. Suppose that~$l$ is a list of \ttt{Integer} values $v_1, v_2, ..., v_n$ and $f : \ttt{Integer} \to \ttt{Integer}$. 
We want to apply~$f$ to each element of the list~$l$ and produce a new list that is the result of mapping~$f$ over~$l$. 
That is, we will create a new list $l'=f(v_1), f(v_2), ..., f(v_n)$. 
By passing a class that implements a functional interface, we can design a single method that receives a list and a function~$f$, rather than having to redundantly design several methods to work over multiple variants of~$f$. 
This operation, in the functional programming domain, is called \ttt{map}, which we saw during our discussion on streams in Chapter~\ref{chapter-arrays-collections}.\footnote{Do not confuse this with the concept of a map/dictionary from our data structures/collections discussion.}

\begin{lstlisting}[language=MyJava]
import static Assertions.assertAll;
import static Assertions.assertEquals;

import java.util.List;

class FunctionMapTester {

  @Test
  void testMap() {
    List<Integer> l = List.of(1, 2, 3, 4, 5);
    Function<Integer, Integer> addOne = new AddOne();
    assertAll(
      () -> assertEquals(List.of(2, 3, 4, 5, 6), 
                         FunctionMap.map(l, addOne)),
      () -> assertEquals(List.of(), 
                         FunctionMap.map(List.of(), addOne)));
  }
}
\end{lstlisting}

\begin{lstlisting}[language=MyJava]
import java.util.List;
import java.util.ArrayList;

class FunctionMap {

  /**
   * Applies the function f to each element of the list l.
   * @param l - the list of elements.
   * @param f - the function to apply to each element.
   * @return the list of elements after applying f to each element.
   */
  static <T, V> List<V> map(List<T> l, Function<T, V> f) {
    return l.stream()
            .map(t -> f.apply(t))
            .collect(Collectors.toList());
  }
}
\end{lstlisting}

\myexample{Java 8 introduced the \ttt{Function} interface, so we do not have to design our own version.} 
When using it, we do not need to design a separate \ttt{AddOne} class to implement the interface; we can instead opt to use method referencing via the \ttt{::} operator. 
Let's rewrite the \ttt{addOne} example doing so. 
Concurrently, we will show off the fact that Java autoboxes and unboxes primitives into wrapper class counterparts in the functional interface, meaning that our \ttt{addOne} method does not need to receive and return objects, but rather primitives, which are easier to work with. 
Moreover, lambda expressions are passable to methods that receive \ttt{Function} arguments, since Java automatically converts them into \ttt{Function} objects, mimicing the autoboxing treatment of primitive datatypes.\footnote{In the tester code snippet below, we could omit the \ttt{FunctionMapTester::} type qualification because the method is defined inside the same class that it is used.}

%\enlargethispage{-1\baselineskip}
\begin{lstlisting}[language=MyJava]
import static Assertions.assertAll;
import static Assertions.assertEquals;

import java.util.List;

class FunctionMapTester {

  static int addOne(int i) { return i + 1; }

  @Test
  void testMap() {
    List<Integer> l = List.of(1, 2, 3, 4, 5);
    assertAll(
      () -> assertEquals(List.of(2, 3, 4, 5, 6), 
                         map(l, FunctionMapTester::addOne)),
      () -> assertEquals(List.of(),   
                         map(List.of(), FunctionMapTester::addOne)),
      () -> assertEquals(List.of(2, 3, 4, 5, 6), 
                         map(l, i -> i + 1)),
      () -> assertEquals(List.of(2, 3, 4, 5, 6), 
                         map(List.of(), i -> i + 1)));
  }
}
\end{lstlisting}

\myexample{Now that we have interfaces, we can write a very simple expression tree interpreter!} 
What do we mean by this? 
Consider the arithmatic expression `$5 + (3 + 4)$'. 
According to the standard order-of-operations, we evaluate the parenthesized sub-expressions first, then reduce outer expressions. 
So, in our case, we add~$3$ and~$4$ to get~$7$, followed by an addition of~$5$. 
We can represent this idea as an evaluation tree, where we travel from bottom-up, evaluating sub-expressions as they are encountered. 
How does the notion of evaluation trees relate to interfaces? 
Suppose we create the \ttt{IExpr} interface, which encompasses the \ttt{int value} method to resolve to the value of an expression.

\begin{lstlisting}[language=MyJava]
interface IExpr {

  /**
   * Returns the value of the expression.
   */
  int value();
}
\end{lstlisting}

The simplest (atomic) values in our language are numbers, or numeric literals. 
A \ttt{Lit} stores a single integer as an instance variable, and returns this instance variable upon a \ttt{value} invocation, which means \ttt{Lit} should implement the \ttt{IExpr} interface. 
Testing this class is trivial, so we will only write two tests.

%\enlargethispage{5\baselineskip}
\begin{lstlisting}[language=MyJava]
import static Assertions.assertAll;
import static Assertions.assertEquals;

class LitTester {

  @Test
  void testLit() {
    assertAll(
      () -> assertEquals(0, new Lit(0).value()),      
      () -> assertEquals(42, new Lit(42).value()));
  }
}
\end{lstlisting}

\begin{lstlisting}[language=MyJava]
class Lit implements IExpr {

  private final int N;

  Lit(int n) { this.N = n; }

  @Override
  public int value() { return this.N; }
}
\end{lstlisting}

How do we add two numbers? 
Or, rather, how do we represent the addition of two (literal) numbers? 
This question comes via the answer to our other question of representing literal values. 
Additive expressions store two \ttt{IExpr} objects as instance variables, and (mutually) recursively calls their \ttt{value} methods, followed by a summation to those results. 
Note the parallelism to how we do this when manually evaluating, say, parenthesized addition expressions.

\begin{lstlisting}[language=MyJava]
import static Assertions.assertAll;
import static Assertions.assertEquals;

class AddTester {

  @Test
  void testAdd() {
    assertAll(
      () -> assertEquals(12, new Add(new Lit(5), 
                                     new Add(new Lit(3), new Lit(4)))),
      () -> assertEquals(42, new Add(new Lit(41), 
                                     new Lit(1))),
      () -> assertEquals(101, new Add(new Add(new Lit(123), new Lit(-43)),
                                      new Add(new Lit(2), new Lit(19)))));
  }
}
\end{lstlisting}

\begin{lstlisting}[language=MyJava]
class Add implements IExpr {

  private final IExpr LHS;
  private final IExpr RHS;

  Add(IExpr lhs, IExpr rhs) {
    this.LHS = lhs;
    this.RHS = rhs;
  }

  @Override
  public int value() { 
    return this.LHS.value() + this.RHS.value(); 
  }
}
\end{lstlisting}
%\enlargethispage{1\baselineskip}
Thus, we now have a programming language that interprets numbers and addition expressions! 
We could add more elements/operators to this language, and we encourage the readers to get creative.

\myexample{Symbolic differentiators are programs that take a mathematical expression and compute its derivative, but non-numerically.} 
That is, symbolic differentiators examine and interpret the structure of an expression to calculate its derivative. 
In this example, we will write a symbolic differentiator in Java using interfaces and classes. Note that you do not need any calculus knowledge to reasonably follow along and understand the high-level and pertinent object-oriented details.

The formal definition of the derivative of a function is not a necessary detail to concern ourselves of; but in short, it measures the instantaneous rate-of-change at a given point of the function, i.e., the slope of the line tangent to the point. 
There are several rules for computing derivatives of functions, all of which are served as common exercises in an introductory calculus course. 
We want to be able to construct expressions in such a way that it is trivial to differentiate their(sub-)components. 
As an example, the derivative of the expression $3x^2 - 16x + 100$ is $6x - 6$ due to specific rules that we will explain shortly. 
The idea, however, is that we have a large expression to find the derivative of, and by differentiating its sub-components, we obtain the derivative of the larger, similar to our arithmetic expression resolver. 
Let's see what all we need to do.

First, let's design the \ttt{Expression} interface, which houses one method, to compute the derivative of an \ttt{Expression}: \ttt{Expression derivative(String v)}. Any class that implements \ttt{Expression} must override \ttt{derivative}. 
We differentiate expressions with respect to a given variable, e.g.,~$x$, so we need to pass that variable to any expression that we wish to differentiate. 

Using some basic calculus derivative shortcuts/rules, we can easily think of two more types of expressions: numeric constants (e.g., $3$, $0$, $27$) and monomials (e.g., $ax^n$ where $a, n$ are integers). 
So, let's design the \ttt{ConstantExpression} and \ttt{MonomialExpression} classes, the former of which has a constructor that receives a single integer~$c$, whereas the latter stores the variable~$v$, the coefficient~$a$, and finally the exponent~$n$. 
To make working with these expressions easier, as well as ensuring testibility, we will override the \ttt{equals}, \ttt{hashCode}, and \ttt{toString} methods.

The derivative of a constant~$c$ is always zero, because the slope of a straight line, namely $f(x) = c$ is zero, i.e., non-changing. 
On the other hand, a monomial follows a different rule based off its coefficient and exponent: the derivative of $ax^n$ is $anx^{n-1}$ for any $n > 1$. 
If $n=1$, then this trivially becomes a constant. 
There is one more edge-case to consider: if the given variable~$v$ does not match the variable of the monomial, then the derivative is zero because the monomial does not depend on the variable~$v$.

\begin{lstlisting}[language=MyJava]
import static Assertions.assertAll;
import static Assertions.assertEquals;

class DerivativeTester {

  @Test
  void testNumberExpressionDerivative() {
    assertAll(
      () -> assertEquals(new NumberExpression(0),
                         new NumberExpression(0).derivative("x")),
      () -> assertEquals(new NumberExpression(0),
                         new NumberExpression(10).derivative("x")));
  }

  @Test
  void testMonomialExpressionDerivative() {
    assertAll(
      () -> assertEquals(new ConstantExpression(3),
              new MonomialExpression("x", 3, 1).derivative("x")),
      () -> assertEquals(new ConstantExpression(0),
              new MonomialExpression("x", 3, 10).derivative("y")),
      () -> assertEquals(new MonomialExpression("x", 6, 1),
              new MonomialExpression("x", 3, 2).derivative("x")));
  }
}
\end{lstlisting}


\begin{lstlisting}[language=MyJava]
import java.util.Objects;

class ConstantExpression implements Expression {

  private final int CONSTANT;

  ConstantExpression(int c) { 
    this.CONSTANT = c; 
  }

  @Override
  public Expression derivative(String v) { 
    return new ConstantExpression(0); 
  }

  @Override
  public boolean equals(Object obj) {
    if (!(obj instanceof ConstantExpression)) { 
      return false; 
    } else { 
      ConstantExpression cons = (ConstantExpression) obj;
      return cons.CONSTANT == this.CONSTANT; 
    }
  }

  @Override
  public int hashCode() { 
    return Objects.hash(this.CONSTANT); 
  }

  @Override
  public String toString() { 
    return String.format("%d", this.CONSTANT); 
  }
}
\end{lstlisting}

%\enlargethispage{-5\baselineskip}
\begin{lstlisting}[language=MyJavaNF]
import java.util.Objects;

class MonomialExpression implements Expression {

  private final int COEFFICIENT;
  private final int EXPT;
  private final String VAR;

  MonomialExpression(String v, int a, int n) { 
    this.VAR = v;
    this.COEFFICIENT = a;
    this.EXPT = n;
  }

  @Override
  public Expression derivative(String v) { 
    if (this.VAR.equals(v)) {
      if (this.EXPT == 1) {
        return new ConstantExpression(this.COEFFICIENT);
      } else {
        return new MonomialExpression(this.VAR, 
                                      this.COEFFICIENT * this.EXPT, 
                                      this.EXPT - 1);
      }
    } else {
      return new ConstantExpression(0);
    } 
  }

  @Override
  public boolean equals(Object obj) {
    if (!(obj instanceof MonomialExpression)) { 
      return false; 
    } else { 
      MonomialExpression expr = (MonomialExpression) obj;
      return this.VAR.equals(expr.VAR)
          && this.COEFFICIENT == expr.COEFFICIENT 
          && this.EXPT == expr.EXPT;
    }
  }

  @Override
  public int hashCode() { 
    return Objects.hash(this.VAR, this.COEFFICIENT, this.EXPT); 
  }
\end{lstlisting}
%\enlargethispage{4\baselineskip}
\begin{lstlisting}[language=MyJavaNF]
  @Override
  public String toString() { 
    return String.format("%d%s^%d", this.COEFFICIENT, 
                                    this.VAR, 
                                    this.EXPT); 
  }
}
\end{lstlisting}

Let's move into compositional expressions, i.e., those that contain expressions as instance variables. 
Such an example is an additive operator: the derivative of the expression $(f(x) + g(x))' = f'(x)~+~g'(x)$, where~$f'$ is the derivative of~$f$. 
In summary, the derivative of a sum is the sum of the derivatives of its operands. 
To represent sequential operands, e.g., $x + y + z + \cdots + w$, we will store the expressions in a list. 
Note that our symbolic differentiator neither simplifies expressions nor combines like terms.

\begin{lstlisting}[language=MyJava]
import static Assertions.assertAll;
import static Assertions.assertEquals;
  
class DerivativeTester {
  // ... previous methods not shown.

  @Test
  void testAddExpressionDerivative() {
    assertAll(
      () -> assertEquals(
              new AddExpression(
                new MonomialExpression("x", 3, 2),
                new MonomialExpression("x", 6, 5)),
              new AddExpression(
                new MonomialExpression("x", 1, 3),
                new MonomialExpression("x", 1, 6)).derivative("x")),
      () -> assertEquals(
              new AddExpression(
                new MonomialExpression("x", 10, 4,
                new MonomialExpression("x", 12, 2),
                new MonomialExpression("x", -14, 1),
                new NumberExpression(6),
                new NumberExpression(0))),
              new AddExpression(
                new MonomialExpression("x", 2, 5),
                new MonomialExpression("x", 4, 3),
                new MonomialExpression("x", -7, 2),
                new MonomialExpression("x", 6, 1),
                new NumberExpression(9)).derivative("x")));
  }
}
\end{lstlisting}

\begin{lstlisting}[language=MyJava]
import java.util.ArrayList;
import java.util.Arrays;
import java.util.List;
import java.util.Objects;
import java.util.stream.Collectors;

class AddExpression implements Expression {

  private final List<Expression> EXPR_LIST;

  AddExpression(Expression... exprs) {
    this.EXPR_LIST = Arrays.asList(exprs);
  }

  AddExpression(List<Expression> exprs) { 
    this.EXPR_LIST = exprs; 
  }

  @Override
  public Expression derivative(String v) {
    List<Expression> exprs = new ArrayList<>();
    this.EXPR_LIST.forEach(e -> exprs.add(e.derivative(v)));
    return new AddExpression(exprs);
  }

  @Override
  public boolean equals(Object obj) {
    if (!(obj instanceof AddExpression)) { 
      return false; 
    } else { 
      AddExpression expr = (AddExpression) obj;
      for (int i = 0; i < this.EXPR_LIST.size(); i++) {
        if (!this.EXPR_LIST.get(i).equals(expr.EXPR_LIST.get(i))) { 
          return false; 
        }
      }
      return true;
    }
  }

  @Override
  public int hashCode() { this.EXPR_LIST.hashCode(); }

  @Override
  public String toString() { 
    return this.EXPR_LIST.stream()
                         .map(Object::toString)
                         .collect(Collectors.joining(" + "));
  }
}
\end{lstlisting}

\myexample{Let's clarify the distinction between the \ttt{Comparable} and \ttt{Comparator} interfaces.}
Consider an \ttt{Employee} class, which stores an employee's first and last name, as well as their salary. 
Suppose we want to be able to compare \ttt{Employee} instances. 
One option to do so is to declare \ttt{Employee} to implement the \ttt{Comparable<Employee>} interface. 
Therefore, \ttt{Employee} must override the \ttt{public int compareTo(Employee e)} method. 
Further suppose that our method will return the result of lexicographically comparing the employee's last name.

\begin{lstlisting}[language=MyJava]
class Employee implements Comparable<Employee> {

  private double salary;
  private String firstName;
  private String lastName;

  Employee(String firstName, String lastName, double salary) {
    this.firstName = firstName;
    this.lastName = lastName;
    this.salary = salary;
  }

  @Override
  public int compareTo(Employee o) {
    return this.lastName.compareTo(o.lastName);
  }

  // Getters and setters omitted.
}
\end{lstlisting}

Now, if we want to create a list of employees and sort them, we can use the static \ttt{sort} method from the \ttt{Collections} class. 

\begin{lstlisting}[language=MyJava]
import static Assertions.assertAll;
import static Assertions.assertEquals;

import java.util.ArrayList;
import java.util.Collections;
import java.util.Collectors;
import java.util.List;

class EmployeeTester {

  @Test
  void testEmployeeComparable() {
    List<Employee> loe1 = new ArrayList<>();
    loe1.add(new Employee("John", "Doe", 100000));
    loe1.add(new Employee("Alex", "Smith", 120000));
    loe1.add(new Employee("Barbara", "Jones", 140000));
    loe1.add(new Employee("Cliff", "Brown", 160000));
    loe1.add(new Employee("Jane", "Davis", 180000));
    loe1.add(new Employee("Trevor", "Wilson", 200000));
    loe1.add(new Employee("Peter", "Taylor", 220000));
    loe1.add(new Employee("Jennifer", "Clark", 240000));
    Collections.sort(loe1);

    assertEquals("Brown, Clark, Davis, Doe, Jones, Smith, Taylor, Wilson",
                 loe1.stream()
                     .map(e -> e.getLastName())
                     .collect(Collectors.join(", ")));
  }
}
\end{lstlisting}

By default, \ttt{Collections.sort} will sort the provided collection using the object's \ttt{compareTo} method. So, in this case, the employees are sorted based on their last name. Note that \ttt{Collections.sort} uses an in-place\index{in-place sorting} sorting algorithm, which means that the original list is modified.

Now, suppose that we want to compare employees using a \emph{different} metric. 
For instance, what if we want to sort the employees based on their first name, or perhaps their salary? 
One approach would be to change how \ttt{compareTo} is implemented in the \ttt{Employee} class. 
The problem with this is that any code that relies on the last name ordering is now broken. 
Plus, it's possible that the source code of the \ttt{Employee} class is immutable, a commonality when working with third-party libraries or legacy codebases.

A solution to this predicament is to use a \ttt{Comparator} object. 
Comparators, as their name suggests, compare instances of a class. 
The essential difference between a \ttt{Comparator} and \ttt{Comparable} is that the class of interest, e.g., \ttt{Employee}, should \emph{not} implement \ttt{Comparator}. 
Rather, we want to create a separate class that represents a comparison between \ttt{Employee} objects by an arbitrary metric. 
For example, in the following listing is a class \ttt{EmployeeFirstNameComparator}, which compares employees by their first name. 
Notice that the \ttt{Comparator} class provides the \ttt{public int compare(T o1, T o2)} method instead of \ttt{compareTo}. 
Another difference is that \ttt{compare} receives two arguments rather than one, because \ttt{compareTo} relies on \ttt{this} and its other argument to perform the relevant comparison. 
That is, \ttt{compareTo} returns a comparison result based on \ttt{this} and its argument. 
By contrast, \ttt{compare} receives \emph{two} instances of the class of interest, and the returned value should be the result of however we choose to compare those two objects.

\begin{lstlisting}[language=MyJava]
import java.util.Comparator;

class EmployeeFirstNameComparator implements Comparator<Employee> {

  @Override
  public int compare(Employee o1, Employee o2) {
    return o1.getFirstName().compareTo(o2.getFirstName());
  }
}
\end{lstlisting}

We can then pass an instance of this comparator as a second argument to the \ttt{Collections.sort} method:

\begin{lstlisting}[language=MyJava]
import static Assertions.assertAll;
import static Assertions.assertEquals;

import java.util.ArrayList;
import java.util.Collections;
import java.util.Collectors;
import java.util.List;

class EmployeeTester {

  @Test
  void testEmployeeComparable() {
    List<Employee> loe1 = new ArrayList<>();
    // ... assume the same list as before.
    Collections.sort(loe1, new EmployeeFirstNameComparator());
    assertEquals("Alex, Barbara, Cliff, Jane, Jennifer, John, Peter, Trevor",
                 loe1.stream()
                     .map(e -> e.getFirstName())
                     .collect(Collectors.joining(", ")));
  }
}
\end{lstlisting}

Let's create another comparator for comparing employees based on their salaries. 
A lower salary indicates a lower index in the ordering relation.

\begin{lstlisting}[language=MyJava]
import java.util.Comparator;

class EmployeeSalaryComparator implements Comparator<Employee> {

  @Override
  public int compare(Employee o1, Employee o2) {
    return (int) Math.signum(o1.getSalary() - o2.getSalary());
  }
}
\end{lstlisting}
