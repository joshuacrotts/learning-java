%%%%%%%%%%%%%%%%%%%%%%acknow.tex%%%%%%%%%%%%%%%%%%%%%%%%%%%%%%%%%%%%%%%%%
% sample acknowledgement chapter
%
% Use this file as a template for your own input.
%
%%%%%%%%%%%%%%%%%%%%%%%% Springer %%%%%%%%%%%%%%%%%%%%%%%%%%

\extrachap{Acknowledgements}

I piloted \emph{Learning Java} on the students in the Fall 2023 and Spring 2024 semesters of CSCI-C 212 (Introduction to Software Systems) at Indiana University. The inspiration came from those in the former, which drove me to continue writing and designing exercises. I sincerely appreciate all of the comments, suggestions, and corrections made. The following students and course staff members found mistakes: Ashley Won, Daniel Yang, Jack Liang, Shyam Makwana, and Muazzam Siddiqui.

Amr Sabry of Indiana University provided the suggestion to add streams to the Java curriculum, and as a consequence I now fully embrace and use them constantly.

I appreciate the help provided by my representative at Springer, Ralf Gerstner, and the fact that Springer took a bold chance to publish the work of a computer science PhD student.

Chandana Ariyawansa of UNC Greensboro and Nathaniel (Nat) Kell assigned brutal problem sets that shaped me into the skilled programmer and educator that I am today. 

Steve Tate of UNC Greensboro put me through the ringer, figuratively, in his graduate algorithms, compilers, system programming, computer security, and software security classes. His mentorship through it all, and the fact that he was the chair of my master's thesis committee, is forever appreciated.

I attribute all of my love for Java (and computer science in general) to my former AP Computer Science A teacher: Tony Smith. Thanks for everything.
